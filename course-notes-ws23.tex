\documentclass[12pt]{report}
\title{Wahrscheinlichkeitstheorie 1}
\author{Cornelius Hanel}
\usepackage{geometry}
\geometry{margin=2.51cm} % 2.51 default
\usepackage{amssymb}
\usepackage{amsmath}
\usepackage{amsthm}
\usepackage{dsfont}
\usepackage[makeroom]{cancel}
\usepackage{enumitem}
\usepackage{pgfplots}
\usepackage{multicol}
\usepackage{ulem}
\pgfplotsset{compat=1.15}
\usepackage[hidelinks]{hyperref}
\hypersetup{
    colorlinks,
    citecolor=black,
    filecolor=black,
    linkcolor=black,
    urlcolor=black,
    linktoc=all
}

\begin{document}
\maketitle

\chapter*{Preface}
\addcontentsline{toc}{chapter}{Preface}
Fehler oder Erg\"anzungen bitte an
\newline
\begin{center}
    \texttt{corneliush99@univie.ac.at}
\end{center}
oder auf \href{https://github.com/cornelius-qq/probtheory1}{GitHub}.

\tableofcontents

%\makeatletter
%\renewcommand{\paragraph}{
%    \@startsection {paragraph}{4}
%    {\z@ }{3.25ex \@plus 1ex \@minus .2ex}{-1em} % change the 3.25ex to some different value maybe
%    {\normalfont \normalsize \bfseries }
%}
%\makeatother

\newcommand{\E}{\mathbb{E}}
\newcommand{\cR}{\mathcal{R}}
\newcommand{\cB}{\mathcal{B}}
\newcommand{\C}{\mathbb{C}}
\newcommand{\del}{\partial}
\newcommand{\A}{\mathcal{A}}
\newcommand{\M}{\mathcal{M}}
\newcommand{\G}{\mathcal{G}}
\renewcommand{\geq}{\geqslant}
\renewcommand{\leq}{\leqslant}
\newcommand{\eps}{\varepsilon}
\newcommand{\Pp}{\mathbb{P}}
\newcommand{\R}{\mathbb{R}}
\newcommand{\Var}{\operatorname{Var}}
\newcommand{\pspace}{(\Omega, \mathcal{A}, \Pp)}
\newcommand{\borel}{\mathcal{B}(\R)}
\newcommand{\ind}[1]{\mathds{1}_{#1}}
\newcommand{\nto}[2]{\xrightarrow[#2]{\makebox[1.5em][c]{$\scriptstyle#1$}}}‌
\newcommand{\krestr}[1]{%
  \sbox{0}{\raisebox{\dimexpr\fontcharht\font`A-\height\relax}{$\big|$}}%
  \usebox{0}%
  \raisebox{\dimexpr-\dp0+\depth\relax}{$\scriptstyle#1$}%
}

\chapter*{1. Messbare R\"aume und Ma\ss{}e}
\addcontentsline{toc}{chapter}{1. Messbare R\"aume und Ma\ss{}e}

\section*{Algebra, $\sigma$-Algebra und Ma\ss{}}
\addcontentsline{toc}{section}{Algebra, $\sigma$-Algebra und Ma\ss{}}

Sei im folgenden Kapitel $\Omega$ jeweils eine nicht-leere Menge. Die Komplementbildung erfolgt jeweils bez\"uglich $\Omega$, also $A^c=\{\omega\in\Omega:\omega\notin A\}$.

\paragraph{1.1. Definition:} Eine Familie von Teilmengen $\A\subseteq\mathcal{P}(\Omega)$ ist eine Algebra (Englisch \textit{field of sets}), wenn folgendes gilt:
\begin{enumerate}[label=(\roman*)]
    \item $\Omega\in\A$ 
    \item $A\in\A\implies A^c\in\A$ (Abgeschlossenheit bzgl. Komplementbildung)
    \item $A,B\in\A\implies A\cup B\in\A$ (Abgeschlossenheit bzgl. endlichen Vereinigungen)
\end{enumerate}

\paragraph{Bemerkung:} (iii) ist \"aquivalent zu $A,B\in\A\implies A\cap B\in\A$ (Abgeschlossenheit bzgl. endlichen Durchschnitten).

\paragraph{1.2. Definition:} Eine Familie von Teilmengen $\A\subseteq\mathcal{P}(\Omega)$ ist eine $\sigma$-Algebra (Englisch auch \textit{$\sigma$-field}), wenn folgendes gilt:
\begin{enumerate}[label=(\roman*)]
    \item $\Omega\in\A$
    \item $A\in\A\implies A^c\in\A$ (Abgeschlossenheit bzgl. Komplementbildung)
    \item $A_n\in\A,n\geq1\implies\displaystyle\bigcup_{n\geq1}A_n\in\A$ (Abgeschlossenheit bzgl. abz\"ahlbaren Vereinigungen)
\end{enumerate}

\paragraph{Bemerkung:} (iii) ist \"aquivalent zu $A_n\in\A,n\geq1\implies\displaystyle\bigcap_{n\geq1}A_n\in\A$ (Abgeschlossenheit bzgl. abz\"ahlbaren Durchschnitten).

\paragraph{1.3. Definition:} Ein messbarer Raum ist ein Paar $(\Omega,\A)$, wobei $\A$ eine $\sigma$-Algebra auf $\Omega$. Eine Menge $A\in\A$ hei\ss{}t messbar. 

\paragraph{1.4. Beispiel:}
\begin{itemize}
    \item $\A:=\{\emptyset,\Omega\}$ ist die kleinste (triviale) $\sigma$-Algebra auf $\Omega$.
    \item $\A:=\mathcal{P}(\Omega)$ ist die gr\"o\ss{}te $\sigma$-Algebra auf $\Omega$.
    \item $\A:=\{A\in\mathcal{P}(\Omega):A\text{ oder }A^c\text{ endlich}\}$ ist eine $\sigma$-Algebra falls $\Omega$ endlich ist, aber nur eine Algebra falls $\Omega$ unendlich ist. Sei $\{\omega_1,\omega_2,\hdots\}\subseteq\Omega$ mit $\omega_i\neq\omega_j$ f\"ur $i\neq j$. Definiere $A_i:=\{\omega_{2i}\}$ f\"ur alle $i\geq1$. Dann gilt $A_i\in\A$ f\"ur alle $i\geq1$, aber $\bigcup_{i\geq1}A_i=\{\omega_2,\omega_4,\hdots\}$ und $\left(\bigcup_{i\geq1}A_i\right)^c=\{\omega_1,\omega_3,\hdots\}$ sind beide unendlich und damit nicht in $\A$.
\end{itemize}

\paragraph{1.5. Definition:}Sei $(\Omega,\A)$ ein messbarer Raum.
Ein Ma\ss{} auf $(\Omega,\A)$ ist eine Abbildung $\mu:\A\to[0,\infty]$, sodass 
    \begin{enumerate}[label=(\roman*)]
        \item $\mu(\emptyset)=0$
        \item F\"ur $A_n\in\A,n\geq1$ paarwise disjunkt gilt ($\sigma$-Additivit\"at)
        $$\mu\left(\bigcup_{n\geq1}A_n\right)=\sum_{n\geq1}\mu(A_n)$$
    \end{enumerate}
Ein Ma\ss{} $\mu:\A\to[0,\infty]$ ist $\sigma$-endlich, falls es $A_n\in\A,n\geq1$ gibt, sodass $\Omega=\displaystyle\bigcup_{n\geq1}A_n$ und $\mu(A_n)<\infty$ f\"ur alle $n\geq1$. Ein Ma\ss{} $\mu:\A\to[0,\infty]$ ist endlich, falls $\mu(\Omega)<\infty$ (damit folgt $\mu:\A\to[0,\infty)$). Ein Warhscheinlichkeitsma\ss{} ist eine Abbildung $\Pp:\A\to[0,1]$ mit $\Pp(\Omega)=1$. 

\paragraph{1.6. Definition:} Sei $(\Omega,\A)$ ein messbarer Raum und $\mu:\A\to[0,\infty]$ ein Ma\ss{} auf $(\Omega,\A)$. Dann nennt man $(\Omega,\A,\mu)$ einen Ma\ss{}raum. Falls $\mu=\Pp$ ein Wahrscheinlichkeitsma\ss{} ist, nennt man $\pspace$ einen Wahrscheinlichkeitsraum.

\section*{Eigenschaften von Ma\ss{}en}
\addcontentsline{toc}{section}{Eigenschaften von Ma\ss{}en}

\paragraph{1.7. Satz:} Sei $(\Omega,\A,\mu)$ ein Ma\ss{}raum. Dann gilt
\begin{enumerate}[label=(\roman*)]
    \item F\"ur $A_i\in\A,1\leq i\leq n$ paarweise disjunkt gilt (endliche Additivit\"at)
    $$\mu\left(\bigcup_{i=1}^nA_i\right)=\sum_{i=1}^n\mu(A_i)$$
    \item F\"ur $A,B\in\A$ mit $A\subseteq B$ gilt $\mu(A)\leq\mu(B)$ (Monotonie). 
    \item F\"ur $A_n\in A,n\geq1$ gilt ($\sigma$-Subadditivit\"at)
    $$\mu\left(\bigcup_{n\geq1}A_n\right)\leq\sum_{n\geq1}\mu(A_n)$$
    \item Falls $\mu$ endlich ist gilt f\"ur $A,B\in\A$, dass $\mu(A\cup B)=\mu(A)+\mu(B)-\mu(A\cap B)$ (Einschluss-Ausschluss-Prinzip)
\end{enumerate}

\paragraph{Beweis:}
\begin{enumerate}[label=(\roman*)]
    \item Setze $A_i:=\emptyset\in\A$ f\"ur $i>n$. Damit folgt die Aussage aus der $\sigma$-Additivit\"at.
    \item Schreibe $B=(B\setminus A)\cup A$ als Vereinigung disjunkter Mengen. Damit folgt mit (i), dass $\mu(B)=\mu(A)+\mu(B\setminus A)\geq\mu(A)$.
    \item Setze hier $B_1:=A_1$ und $B_k:=A_k\setminus\left(\bigcup_{j=1}^{k-1}A_j\right)$ f\"ur $k\geq2$. Dann gilt $B_k\in\A$ f\"ur alle $k\geq1$, $B_k$ sind paarweise disjunkt, $\bigcup_{k\geq1}B_k=\bigcup_{n\geq1}A_n$ und $B_k\subseteq A_k$ f\"ur alle $k\geq1$. Es folgt mit (ii)
    $$\mu\left(\bigcup_{n\geq1}A_n\right)=\mu\left(\bigcup_{k\geq1}B_k\right)=\sum_{k\geq1}\mu(B_k)\leq\sum_{m\geq1}\mu(A_n)$$
    \item Schreibe $A\cup B=(A\setminus B)\cup(A\cap B)\cup(B\setminus A)$. Dann gilt mit (i)
    \begin{align*}
        \mu(A\cup B)&=\mu(A\setminus B)+\mu(A\cap B)+\mu(B\setminus A)\\
        &=\mu(A)+\mu(B\setminus A)\\
        &=\mu(A)+\mu(B)-\mu(A\cap B)
    \end{align*}
    \qed
\end{enumerate}

\paragraph{Bemerkung:}In der Literatur ist das Einschluss-Ausschluss-Prinzip meist in der allgemeineren Form f\"ur endliche Vereinigungen $\mu\left(\bigcup_{i=1}^nA_n\right)$ zu finden.

\paragraph{1.8. Korollar:}F\"ur $A,B\in/A$ mit $A\subseteq B$ und $\mu(B)<\infty$ gilt $\mu(B\setminus A)=\mu(B)-\mu(A)$. Damit folgt f\"ur endliche Ma\ss{}e $\mu(A^c)=\mu(\Omega)-\mu(A)$ und insbesondere f\"ur Wahrscheinlichkeitsma\ss{}e $\Pp(A^c)=1-\Pp(A)$.

\paragraph{Beweis:}Folgt sofort aus Satz 1.7 (i) mit $B=A\cup (B\setminus A)$. \qed

\paragraph{1.9. Satz (Stetigkeit von unten/oben):}Sei $(\Omega,\A,\mu)$ ein Ma\ss{}raum und $A_n\in\A,n\geq1$.
\begin{enumerate}[label=(\roman*)]
    \item Falls $A_1\subseteq A_2\subseteq\hdots\subseteq\bigcup_{n\geq1}A_n$, dann gilt $\mu(A_n)\nto{}{n\to\infty}\mu\left(\bigcup_{n\geq1}A_n\right)$ (Stetigkeit von unten).
    \item Falls $A_1\supseteq A_2\supseteq\hdots\supseteq\bigcap_{n\geq1}A_n$ und $\mu(A_1)<\infty$, dann gilt $\mu(A_n)\nto{}{n\to\infty}\mu\left(\bigcap_{n\geq1}A_n\right)$ (Stetigkeit von oben).
\end{enumerate}
In (ii) gen\"ugt es $\mu(A_j)<\infty$ f\"ur ein $j\geq1$ vorauszusetzen, also $\limsup_{n\to\infty}\mu(A_n)<\infty$ (Monotonie).

\paragraph{Beweis:}
\begin{enumerate}[label=(\roman*)]
    \item Setze $B_1:=A_1$ und $B_k:=A_k\setminus\left(\bigcup_{j=1}^{k-1}A_j\right)$. Damit ist $B_k$ f\"ur alle $k\geq1$ messbar und $B_k$ sind paarweise disjunkt. Weiters gilt $A_n=\bigcup_{k=1}^n B_k$, $B_n\subseteq A_n$ und $\bigcup_{n\geq1}A_n=\bigcup_{k\geq1}B_k$ (leicht nachzupr\"ufen). Es folgt
    \begin{align*}
        \mu\left(\bigcup_{n\geq1}A_n\right)&=\mu\left(\bigcup_{k\geq1}B_k\right)\\
        &=\sum_{k\geq1}\mu(B_k)\\
        &=\lim_{K\to\infty}\sum_{k=1}^K\mu(B_k)\\
        &=\lim_{K\to\infty}\mu\left(\bigcup_{k=1}^K B_k\right)\\
        &=\lim_{N\to\infty}\mu(A_N)
    \end{align*}
    \item Setze $B_k:=A_1\setminus A_k$. Dann gilt $B_1\subseteq B_2\subseteq\hdots\subseteq\bigcup_{k\geq1}B_k=A_1\setminus\left(\bigcap_{n\geq1}A_n\right)$. Mit (i) folgt
    $$\mu(B_k)\nto{}{k\to\infty}\mu\left(A_1\setminus\left(\bigcap_{n\geq1}A_n\right)\right)$$
    Es gilt $A_n\subseteq A_1$, $\bigcap_{n\geq1}A_n\subseteq A_1$ und damit $\mu\left(\bigcap_{n\geq1}A_n\right)<\infty$. Es folgt 
    $$\lim_{k\to\infty}\mu(B_k)=\mu(A_1)-\lim_{n\to\infty}\mu(A_n)=\mu(A_1)-\lim_{n\to\infty}\mu\left(\bigcap_{n\geq1}A_n\right)$$
    und damit die Aussage. \qed
\end{enumerate}

\paragraph{1.10. Definition:}Sei $(\Omega,\A,\mu)$ ein Ma\ss{}raum. Falls f\"ur $\omega\in\Omega$ $\{\omega\}\in\A$ und $\mu(\{\omega\})>0$ gilt, dann nennt man $\omega$ ein Atom von $\mu$ (bez\"uglich dem Ma\ss{}raum).

\paragraph{1.11. Proposition:}Sei $(\Omega,\A,\mu)$ ein $\sigma$-endlicher Ma\ss{}raum. Dann ist die Menge der Atome $A:=\{\omega\in\Omega:\{\omega\}\in\A,\mu(\{\omega\})>0\}$ h\"ochstens abz\"ahlbar.

\paragraph{Beweis:}Schreibe
$$A=\bigcup_{n\geq1}\left\{\omega\in\Omega:\{\omega\}\in\A,\mu(\{\omega\})>\frac{1}{n}\right\}$$
Es gen\"ugt zu zeigen, dass jedes Element der Vereinigung h\"ochstens abz\"ahlbar ist (abz\"ahlbare Vereinigung abz\"ahlbarer Mengen ist abz\"ahlbar [ben\"otigt das Auswahlaxiom]). Sei also $A_n:=\left\{\omega\in\Omega:\{\omega\}\in\A,\mu(\{\omega\})>\frac{1}{n}\right\}$. W\"ahle au\ss{}erdem $B_n\in\A,n\geq1$, sodass $\bigcup_{n\geq1}B_n=\Omega$ und $\mu(B_n)<\infty$ ($\sigma$-Endlichkeit). Dann gilt
$$A_n=\bigcup_{k\geq1}(B_k\cap A_n)$$
Es gen\"ugt also zu zeigen, dass $A_n\cap B_k$ f\"ur alle $k,n\geq1$ h\"ochstens abz\"ahlbar ist. Wir zeigen sogar, dass $A_n\cap B_k$ f\"ur alle $k,n\geq1$ endlich ist:\newline
 Angenommen $A_n\cap B_k$ ist abz\"ahlbar unendlich und schreibe $A_n\cap B_k=\{\omega_1,\omega_2,\hdots\}$ mit $\omega_i\neq\omega_j$ f\"ur $i\neq j$. Damit folgt
 $$\mu(A_n\cap B_k)=\mu\left(\bigcup_{j\geq1}\{\omega_j\}\right)=\sum_{j\geq1}\mu(\{\omega_j\})=\infty$$
Aber $\mu(A_n\cap B_k)\leq\mu(B_k)<\infty$ f\"ur alle $n,k\geq1$, ein Widerspruch. Es verbleibt noch zu zeigen, dass $A_n\cap B_k$ nicht \"uberabz\"ahlbar sein kann (einfache \"Uberlegung). \qed

\section*{Erzeugung von $\sigma$-Algebren}
\addcontentsline{toc}{section}{Erzeugung von $\sigma$-Algebren}
\paragraph{1.12. Lemma:}Sei $I$ eine beliebige Indexmenge. Sei $\A_i$ f\"ur jedes $i\in I$ eine $\sigma$-Algebra auf $\Omega$. Dann ist $\A:=\bigcap_{i\in I}\A_i$ wieder eine $\sigma$-Algebra auf $\Omega$.

\paragraph{Beweis:}Es gilt drei Eigenschaften zu zeigen:
\begin{enumerate}[label=(\roman*)]
    \item $\Omega\in\A$\newline
    Es gilt laut Annahme $\Omega\in\A_i$ f\"ur alle $i\in I$, womit die Behauptung sofort folgt.
    \item $A\in\A\implies A^c\in\A$\newline
    $A\in\A\iff\forall i\in I:A\in\A_i\implies\forall i\in I:A^c\in\A_i\iff A^c\in\A$
    \item $A_n\in\A,n\geq1\implies\bigcup_{n\geq1}A_n\in\A$\newline
    wie (ii). \qed
\end{enumerate}

\paragraph{1.13. Definition:} Sei $\M\subseteq\mathcal{P}(\Omega)$. Dann definiert man die von $\M$ erzeute $\sigma$-Algebra als
$$\sigma(\M):=\bigcap_{\substack{\A\ \sigma\text{-Algebra}\\\M\subseteq\A}}\A$$
Mit Lemma 1.12 folgt sofort, dass $\sigma(\M)$ eine $\sigma$-Algebra bez\"uglich $\Omega$ ist. Weiters ist $\sigma(\M)$ die kleinste $\sigma$-Algebra, die $\M$ enth\"alt (i.e. ist $\mathcal{E}$ eine $\sigma$-Algebra mit $\M\subseteq\mathcal{E}$, dann folgt $\sigma(\M)\subseteq\mathcal{E}$). 

\paragraph{1.14. Lemma:}Sei $\M_1\subseteq\M_2\subseteq\mathcal{P}(\Omega)$. Dann folgt $\sigma(\M_1)\subseteq\sigma(\M_2)$.

\paragraph{Beweis:}$\sigma(\M_2)$ ist eine $\sigma$-Algebra, die $\M_2$ enth\"alt, und damit auch $\M_1$. Mit der Bemerkung in Definition 1.13 folgt die Aussage. \qed

% Nicht nummeriert, aber sollte es mMn sein
\paragraph{1.14.$\frac{1}{2}$. Definition:}Sei $K\subseteq\Omega$ und $\M\subseteq\mathcal{P}(\Omega)$. Dann definiert man die Spur (Englisch \textit{trace}) von $\M$ auf $K$ als
$$\M\krestr{K}:=\{M\cap K:M\in\M\}$$

\paragraph{1.15. Proposition:} Sei $\A$ eine $\sigma$-Algebra auf $\Omega$ und $K\subseteq\Omega$. Dann ist $\A\krestr{K}$ eine $\sigma$-Algebra auf $K$. Man nennt $\A\krestr{K}$ die Spur-$\sigma$-Algebra von $\A$ auf $K$. 

\paragraph{Beweis:}Es gilt drei Eigenschaften zu zeigen:
\begin{enumerate}[label=(\roman*)]
    \item $K\in\A\krestr{K}$\newline
    Es gilt $\Omega\in\A$ und damit $K=K\cap\Omega\in \A\krestr{K}$.
    \item $A\in\A\krestr{K}\implies K\setminus A\in\A\krestr{K}$\newline
    $A\in\A\krestr{K}\implies A=K\cap B,B\in\A$. Nun gilt aber 
    \begin{align*}
        K\setminus A&=(K\setminus A)\cap K=(K\setminus(K\cap B))\cap K\\
        &=(K\cap(K\cap B)^c)\cap K\\
        &=(K\setminus B)\cup(K\cap K^c)\\
        &=K\cap B^c\in \A\krestr{K}
    \end{align*}
    da $B^c\in\A$.
    \item $A_n\in\A\krestr{K},n\geq1\implies\bigcup_{n\geq1}A_n\in\A\krestr{K}$\newline
    Es gilt $A_n=B_n\cap K$ f\"ur $B_n\in\A, n\geq1$ und damit
    $$\bigcup_{n\geq1}A_n=\bigcup_{n\geq1}(B_n\cap K)=K\cap\bigcup_{n\geq1}B_n\in\A\krestr{K}$$
    da $\bigcup_{n\geq1}B_n\in\A$. \qed
\end{enumerate}

\paragraph{1.16. Lemma:}Sei $\M\subseteq\mathcal{P}(\Omega)$ und $K\subseteq\Omega$. Dann gilt 
$$\sigma(\M\krestr{K})=\sigma(\M)\krestr{K}$$

\paragraph{Beweis:}
Wir verwenden hier einige Ergebnisse aus der \"Ubung, z.B. $K\subseteq L\implies\M\krestr{K}\subseteq\M\krestr{L}$, $\M\subseteq\sigma(\M)$ und $\A\ \sigma\text{-Algebra}\implies\sigma(\A)=\A$ 
\begin{enumerate}[label=\Roman*.]
    \item \underline{$\sigma(\M\krestr{K})\subseteq\sigma(\M)\krestr{K}$}\newline
    Es gilt $\M\subseteq\sigma(\M)$ und damit $\M\krestr{K}\subseteq\sigma(\M)\krestr{K}$. Damit folgt
    $$\sigma(\M\krestr{K})\subseteq\sigma\left(\sigma(\M)\krestr{K}\right)=\sigma(\M)\krestr{K}$$ 
    da $\sigma(\M)\krestr{K}$ mit Proposition 1.15 eine $\sigma$-Algebra ist.
    
    \item \underline{$\sigma(\M\krestr{K})\supseteq\sigma(\M)\krestr{K}$}\newline
    Definiere hierf\"ur
    $$\G:=\{A\in\sigma(\M):A\cap K\in\sigma(\M\krestr{K})\}\subseteq\sigma(\M)$$
    Falls $\G$ eine $\sigma$-Algebra ist, die $\M$ enth\"alt, dann folgt $\sigma(\M)\subseteq\G$ und damit $\G=\sigma(\M)$. Daraus folgt schlie\ss{}lich
    $$\forall A\in\sigma(\M):A\cap K\in\sigma(\M\krestr{K})$$
    und damit $\sigma(\M)\krestr{K}\subseteq\sigma(\M\krestr{K})$.
    
    \item \underline{$\G$ ist eine $\sigma$-Algebra und $\M\subseteq\G$}\newline
    $\M\subseteq\G$ folgt sofort aus $\M\subseteq\sigma(\M)$ und $\M\krestr{K}\subseteq\sigma(\M\krestr{K})$. Zeige also, dass $\G$ eine $\sigma$-Algebra ist.
    \begin{enumerate}[label=(\roman*)]
        \item $\Omega\in\sigma(\M)$ und $\Omega\cap K=K\in\sigma(\M\krestr{K})$.
        \item $A\in\G\iff A\in\sigma(\M)\land A\cap K\in\sigma(\M\krestr{K})$\newline
            $\implies A^c\in\sigma(\M)\land K\setminus(A\cap K)=K\cap A^c\in\sigma(\M\krestr{K})$
        \item $A_n\in\G,n\geq1\iff \forall n\geq1:A_n\in\sigma(\M)\land A_n\cap K\in\sigma(\M\krestr{K})$\newline$\implies\bigcup_{n\geq1}A_n\in\sigma(\M)\land\bigcup_{n\geq1}(A_n\cap K)=K\cap\bigcup_{n\geq1}A_n\in\sigma(\M\krestr{K})$ \qed
    \end{enumerate}
\end{enumerate}

\chapter*{2. \"Au\ss{}ere und Innere Ma\ss{}e}
\addcontentsline{toc}{chapter}{2. \"Au\ss{}ere und Innere Ma\ss{}e}

\section*{$\lambda$- und $\pi$-Systeme}
\addcontentsline{toc}{section}{$\lambda$- und $\pi$-Systeme}

\paragraph{2.1. Definition:}Eine Mengenfamilie $\mathcal{D}\subseteq\mathcal{P}(\Omega)$ ist ein $\lambda$-System (auch Dynkin-System oder d-System), wenn gilt
\begin{enumerate}[label=(\roman*)]
    \item $\Omega\in\mathcal{D}$
    \item $A,B\in\mathcal{D},A\subseteq B\implies B\setminus A\in\mathcal{D}$
    \item $\forall n\geq1:A_n\in\mathcal{D},A_1\subseteq A_2\subseteq\hdots\implies\displaystyle\bigcup_{n\geq1}A_n\in\mathcal{D}$
\end{enumerate}

\paragraph{2.2. Lemma:}Sei $(\Omega,\A)$ ein messbarer Raum und seien $\mu,\nu$ endliche Ma\ss{}e auf $(\Omega,\A)$, sodass $\mu(\Omega)=\nu(\Omega)$. Dann ist 
$$\mathcal{D}:=\{A\in\A:\mu(A)=\nu(A)\}$$
ein $\lambda$-System.

\paragraph{Beweis:}
\begin{enumerate}[label=(\roman*)]
    \item Laut Voraussetzung gilt $\mu(\Omega)=\nu(\Omega)$.
    \item Seien $A,B\in\mathcal{D}$. Dann gilt $\mu(A)=\nu(A)$ und $\mu(B)=\nu(B)$ und es folgt
    $$\mu(B\setminus A)=\mu(B)-\mu(A)=\nu(B)-\nu(A)=\nu(B\setminus A)$$
    \item Seien $A_n\in\mathcal{D},n\geq1$ und $A_1\subseteq A_2\subseteq\hdots$. Dann gilt $\mu(A_n)=\nu(A_n)$ f\"ur alle $n\geq1$. Mit der Stetigkeit von unten folgt 
    $$\mu\left(\bigcup_{n\geq1}A_n\right)=\lim_{n\to\infty}\mu(A_n)=\lim_{n\to\infty}\nu(A_n)=\nu\left(\bigcup_{n\geq1}A_n\right)$$
    \qed
\end{enumerate}

\paragraph{2.3. Definition:}Eine durchschnittsstabile Mengenfamilie $\M\subseteq\mathcal{P}(\Omega)$ nennt man $\pi$-System.

\paragraph{Bemerkung:}Jede $\sigma$-Algebra ist damit sowohl ein $\lambda$-System, als auch ein $\pi$-System.

\paragraph{2.4. Lemma:}Sei $\M$ sowohl ein $\lambda$-System, als auch ein $\pi$-System. Dann ist $\M$ eine $\sigma$-Algebra.

\paragraph{Beweis:}Die ersten beiden Eigenschaften einer $\sigma$-Algebra folgen sofort. Mit de Morgan folgt au\ss{}erdem die Abgeschlossenheit bez\"uglich endlicher Vereinigungen. Seien also $A_n\in\M,n\geq1$. Setze $B_N:=\displaystyle\bigcup_{n=1}^NA_n\in\M$ f\"ur alle $N\geq1$. Dann gilt $B_1\subseteq B_2\subseteq\hdots$ und damit $\displaystyle\bigcup_{N\geq1}B_N=\bigcup_{n\geq1}A_n\in\M$. \qed

\paragraph{2.5. Definition:}F\"ur $\M\subseteq\mathcal{P}(\Omega)$ definiere das von $\M$ erzeugte $\lambda$-System als
$$\lambda(\M):=\bigcap_{\substack{\mathcal{D}\ \lambda\text{-System}\\ \M\subseteq\mathcal{D}}}\mathcal{D}$$
$\lambda(\M)$ ist wohldefiniert, da z.B. $\mathcal{P}(\Omega)$ oder $\sigma(\M)$ $\lambda$-Systeme sind, die $\M$ enthalten. 

\paragraph{2.6. Proposition:}$\lambda(\M)$ ist das kleinste $\lambda$-System auf $\Omega$, das $\M$ enth\"alt. 

\paragraph{Beweis:}Da der Durchschnitt beliebig vieler $\lambda$-Systeme wieder ein $\lambda$-System ist, ist $\lambda(\M)$ ein $\lambda$-System. Sei $\mathcal{D}$ ein $\lambda$-System, das $\M$ enth\"alt. Dann gilt $\lambda(\M)\subseteq\mathcal{D}$. \qed

\paragraph{2.7. Satz (Sierpi\'{n}ski\textendash Dynkin's $\lambda\textendash\pi$ Theorem):}Ist $\M$ ein $\pi$-System, dann gilt 
$$\lambda(\M)=\sigma(\M)$$

\paragraph{Beweis:}Die Inklusion $\lambda(\M)\subseteq\sigma(\M)$ folgt sofort aus der Bemerkung zu Definition 2.3. Es verbleibt also die Inklusion $\sigma(\M)\subseteq\lambda(\M)$ zu zeigen. Dazu gen\"ugt es zu zeige, dass $\lambda(\M)$ ein $\pi$-System ist. 
\begin{enumerate}[label=\Roman*.]
    \item Definiere $\mathcal{D}_1:=\left\{A\in\lambda(\M):\forall M\in\M:A\cap M\in\lambda(\M)\right\}$. Dann gilt $\M\subseteq\mathcal{D}_1$ (leicht nachzupr\"ufen) und per Konstruktion $\mathcal{D}_1\subseteq\lambda(\M)$. Falls $\mathcal{D}_1$ ein $\lambda$-System ist (Beweis \"Ubung), gilt $\lambda(\M)\subseteq\mathcal{D}_1$ und somit $\mathcal{D}_1=\lambda(\M)$.
    \item Definiere $\mathcal{D}_2:=\left\{A\in\lambda(\M):\forall B\in\lambda(\M):A\cap B\in\lambda(\M)\right\}$. Es gilt $\M\subseteq\mathcal{D}_1\subseteq\mathcal{D}_2$ und $\mathcal{D}_2\subseteq\lambda(\M)$. Falls $\mathcal{D}_2$ ein $\lambda$-System ist (Beweis \"Ubung), gilt $\lambda(\M)\subseteq\mathcal{D}_2$ und damit $\mathcal{D}_2=\lambda(\M)$. 
\end{enumerate}
Damit folgt, dass $\lambda(\M)$ durchschnittsstabil ist. \qed

\paragraph{2.8. Korollar:}Seien $\mu$ und $\nu$ endliche Ma\ss{}e auf einem messbaren Raum $(\Omega,\A)$, die auf einem $\pi$-System $\mathcal{M}\subseteq\A$ \"ubereinstimmen (i.e. $\forall M\in\M:\mu(M)=\nu(M)$). Falls $\mu(\Omega)=\nu(\Omega)$, dann stimmen $\mu$ und $\nu$ auch auf $\sigma(\M)$ \"uberein. 

\paragraph{Beweis:}Sei $\mathcal{D}$ wie in Lemma 2.2. Dann gilt $\M\subseteq\mathcal{D}$ und mit Proposition 2.6 folgt $\lambda(\M)\subseteq\mathcal{D}$. Mit Satz 2.7 folgt schlie\ss{}lich $\lambda(\M)=\sigma(\M)\subseteq\mathcal{D}$. \qed

\paragraph{2.9. Korollar:}Seien $\mu$ und $\nu$ Ma\ss{}e auf einem messbaren Raum $(\Omega,\A)$, die auf einem $\pi$-System $\M\subseteq\A$ \"ubereinstimmen und sei $\mu$ $\sigma$-endlich auf $\M$. Dann stimmen $\mu$ und $\nu$ auch auf $\sigma(\M)$ \"uberein. 

\paragraph{Beweis:}Trivial, falls $\mu(\Omega)=0$. Sei also $\mu(\Omega)>0$. Mit der $\sigma$-Endlichkeit von $\mu$ auf $\M$, gibt es $A_n\in\M$, sodass $\mu(A_n)$ f\"ur $n$ hinreichend gro\ss{} ($\exists N\geq1,\forall n\geq N$, etc.). Sei o.B.d.A. $\mu(A_n)>0$ f\"ur alle $n\geq1$ und definiere
$$\mu_n(\cdot):=\dfrac{\mu(A_n\cap\cdot)}{\mu(A_n)}\text{ und }\nu_n(\cdot):=\dfrac{\nu(A_n\cap\cdot)}{\nu(A_n)}$$
Dann sind $\mu_n$ und $\nu_n$ endliche Ma\ss{}e auf $(\Omega,\A)$ und laut Voraussetzung stimmen $\mu_n$ und $\nu_n$ auf $\M$ \"uberein, da $A_n\cap M\in\M$ f\"ur alle $M\in\M$. Au\ss{}erdem gilt $\mu_n(\Omega)=\nu_n(\Omega)=1$. Mit Korollar 2.8 folgt, dass $\mu_n$ und $\nu_n$ auch auf $\sigma(\M)$ \"ubereinstimmen. Nun gilt f\"ur $A\in\sigma(\M)$
\begin{align*}
    \mu(A)&=\mu\left(A\cap\bigcup_{n\geq1}A_n\right)=\mu\left(\bigcup_{n\geq1}(A_n\cap A)\right)\\
    &=\lim_{n\to\infty}\mu(A_n\cap A)=\lim_{n\to\infty}\mu_n(A)\cdot\mu(A_n)\\
    &=\lim_{n\to\infty}\nu_n(A)\cdot\nu(A_n)=\nu(A)
\end{align*}
\qed

\section*{Pr\"ama\ss{}e und \"au\ss{}ere Ma\ss{}e}
\addcontentsline{toc}{section}{Pr\"ama\ss{}e und \"au\ss{}ere Ma\ss{}e}

\paragraph{2.10. Definition:}Sei $\A_0$ eine Algebra auf $\Omega$. Ein Pr\"ama\ss{} auf $(\Omega,\A_0)$ ist eine Abbildung $\mu:\nobreak\A_0\to\nobreak [0,\infty]$ mit
\begin{enumerate}[label=(\roman*)]
    \item $\mu(\emptyset)=0$
    \item F\"ur $A_i\in\A_0,i\geq1$ disjunkt mit $\displaystyle\bigcup_{i\geq1}A_i\in\A_0$ gilt $\displaystyle\mu\left(\bigcup_{i\geq1}A_i\right)=\sum_{i\geq1}\mu(A_i)$
\end{enumerate}

\paragraph{2.11. Definition:}Sei $\mu$ ein Pr\"ama\ss{} auf $(\Omega,\A_0)$. Das entsprechende \"au\ss{}ere Ma\ss{} ist die Abbildung $\mu^*:\mathcal{P}(\Omega)\to[0,\infty]$ mit 
$$\mu^*(A):=\inf_{\substack{A_i\in\A_0,i\geq1\\ A\subseteq\bigcup_{i\geq1}A_i}}\sum_{i\geq1}\mu(A_i)$$

\paragraph{Bemerkung:}$\mu^*$ ist wohldefiniert, da $\Omega,\emptyset\in\A_0$ und der $\inf$ damit von unten durch $0$ beschr\"ankt ist. 

\paragraph{2.12. Definition:}Sei $\mu$ ein Pr\"ama\ss{} auf $(\Omega,\A_0)$ und $\mu^*$ das entsprechende \"au\ss{}ere Ma\ss{}. Eine Menge $A\subseteq\Omega$ ist von au\ss{}en messbar (bzgl. $\mu^*$), falls
$$\forall M\in\mathcal{P}(\Omega):\mu^*(M\cap A)+\mu^*(M\cap A^c)=\mu^*(M)$$
Sei au\ss{}erdem $\A^*$ die Familie der von au\ss{}en messbaren Mengen.\newline \newline
Betrachte im Folgenden jeweils ein Pr\"ama\ss{} $\mu$ auf $(\Omega,\A_0)$, das entsprechende \"au\ss{}ere Ma\ss{} $\mu^*$ und die von au\ss{}en messbaren Mengen $\A^*$.

\paragraph{2.13. Lemma:}
\begin{enumerate}[label=(\roman*)]
    \item $\mu^*(\emptyset)=0$
    \item $\forall A\in\mathcal{P}(\Omega):\mu^*(A)\geq0$
    \item $\forall A,B\in\mathcal{P}(\Omega),A\subseteq B:\mu^*(A)\leq\mu^*(B)$
\end{enumerate}

\paragraph{Beweis:}
\begin{enumerate}[label=(\roman*)]
    \item $\mu^*(\emptyset)=\sum_{i\geq1}\mu(\emptyset)=0$, da $\emptyset\in\A_0$. Insbesondere ist $\emptyset$ von au\ss{}en messbar (einfach zu pr\"ufen).
    \item siehe Bemerkung zu Definition 2.11. 
    \item Zeige hierzu die $\sigma$-Subadditivit\"at, also 
    $$\mu^*\left(\bigcup_{n\geq1}A_n\right)\leq\sum_{n\geq1}\mu^*(A_n)$$
    f\"ur alle $A_n\in\mathcal{P}(\Omega),n\ge1$. Sei dazu $\eps>0$ und w\"ahle $B_{n,i}\in\A_0,i\geq1$, sodass $A_n\subseteq\bigcup_{i\geq1}B_{n,i}$. Dann gilt 
    $$\bigcup_{n\geq1}A_n\subseteq\bigcup_{n\geq1}\bigcup_{i\geq1}B_{n,i}\text{ und }\sum_{i\geq1}\mu(B_{n,i})\leq\mu^*(A_n)+\dfrac{\eps}{2^n}$$
Hinweis: $a\leq\left(\displaystyle\inf_{a\in A}a\right)+\eps$ f\"ur alle $a\in A$ und $\eps>0$. Es folgt
$$\mu^*\left(\bigcup_{n\geq1}A_n\right)\leq\sum_{n\geq1}\sum_{i\geq1}\mu(B_{n,i})\leq\sum_{n\geq1}\mu^*(A_n)+\dfrac{\eps}{2^n}=\eps+\sum_{n\geq1}\mu^*(A_n)$$
Die $\sigma$-Subadditivit\"at (und damit die Monotonie) folgt f\"ur $\eps\searrow0$. \qed
\end{enumerate}

\paragraph{2.14. Korollar:}$\mu^*$ ist damit auch endlich subadditiv, also
$$\forall A,B\in\mathcal{P}(\Omega):\mu^*(A\cup B)\leq\mu^*(A)+\mu^*(B)$$
Insbesondere ist eine Menge $A\in\mathcal{P}(\Omega)$ genau dann von au\ss{}en messbar, wenn
$$\forall M\in\mathcal{P}(\Omega):\mu^*(M\cap A)+ \mu^*(M\cap A^c)\leq\mu^*(M)$$

\paragraph{2.15. Lemma:}$A^*$ ist eine Algebra. 

\paragraph{Beweis:}
\begin{enumerate}[label=(\roman*)]
    \item $\Omega\in\A^*$: folgt \"ahnlich wie der Beweis von Lemma 2.13 (i) oder aus (ii) unten.
    \item $A\in\A^*\implies A^c\in\A^*$: trivial.
    \item $A,B\in\A^*\implies A\cap B\in\A^*$: Es gilt $\forall M\in\mathcal{P}(\Omega)$
    \begin{align*}
        \mu^*(M)&=\mu^*(M\cap A)+\mu^*(M\cap A^c)\\
        &=\mu^*(M\cap A\cap B)+\mu^*(M\cap A^c\cap B)+\mu^*(M\cap A\cap B^c)+\mu^*(M\cap A^c\cap B^c)\\
        &=\mu^*(M\cap A\cap B)+\mu^*\left((M\cap A\cap B^c)\cup(M\cap A^c\cap B^c)\cup(M\cap A^c\cap B)\right)\\
        &=\mu^*(M\cap A\cap B)+\mu^*\left((M\cap A^c)\cup(M\cap A\cap B^c)\right)\\
        &=\mu^*(M\cap(A\cap B))+\mu^*(M\cap(A\cap B)^c)
    \end{align*}
    Die Aussage folgt mit Korollar 2.14. \qed
\end{enumerate}

\paragraph{2.16. Lemma:}Seien $A_i,\in\A^*,i\geq1$ disjunkt. Dann gilt f\"ur alle $M\in\mathcal{P}(\Omega)$
$$\mu^*\left(M\cap\bigcup_{i\geq1}A_i\right)=\sum_{i\geq1}\mu^*(M\cap A_i)$$

\paragraph{Beweis:}Sei $N\in\mathbb{N}\cup\{\infty\}$ und sei $A_n=\emptyset$ f\"ur alle $n>N$. Induktion in $N$.
\begin{itemize}
    \item $N=1$: trivial
    \item $N=2$: Hier gilt 
    \begin{align*}
        \mu^*(M\cap(A_1\cup A_2))&=\mu^*(M\cap(A_1\cup A_2)\cap A_1)+\mu^*(M\cap(A_1\cup A_2)\cap A_1^c)\\
        &=\mu^*(M\cap A_1)+\mu^*(M\cap A_2)
    \end{align*}
    da $A_1\in\A^*$.
    \item $P(N)\implies P(N+1)$: Da $\A^*$ eine Algebra ist, gilt $\displaystyle\bigcup_{i=1}^NA_i\in\A^*$ und damit
    \begin{align*}
        \mu^*\left(M\cap\bigcup_{i=1}^{N+1}A_i\right)&=\mu^*\left(M\cap\left(\bigcup_{i=1}^{N}A_i\cup A_{N+1}\right)\right)\\
        &=\mu^*\left(M\cap \bigcup_{i=1}^N A_i\right)+\mu^*(M\cap A_{N+1})\\
        &=\sum_{i=1}^N\mu^*(M\cap A_i)+\mu^*(M\cap A_{N+1})\\
        &=\sum_{i=1}^{N+1}\mu^*(M\cap A_i)
    \end{align*}
    \item $N=\infty$: Mit der $\sigma$-Subadditivit\"at gilt
    $$\mu^*\left(M\cap\bigcup_{i\geq1}A_i\right)=\mu^*\left(\bigcup_{i\geq1}(M\cap A_i)\right)\leq\sum_{i\geq1}\mu^*(M\cap A_i)$$
    Aber f\"ur alle $m\geq1$ gilt mit der Monotonie
    \begin{align*}
        \mu^*\left(M\cap\bigcup_{i\geq1}A_i\right)&\geq\mu^*\left(M\cap\bigcup_{i=1}^mA_i\right)\\
        &=\sum_{i=1}^m\mu^*(M\cap A_i)
    \end{align*}
    und die Aussage folgt f\"ur $m\to\infty$. \qed
\end{itemize}

\paragraph{2.17. Lemma:}$\A^*$ ist eine $\sigma$-Algebra. 

\paragraph{Beweis:}Mit Lemma 2.15 gen\"ugt es die Abgeschlossenheit bzgl. abz\"ahlbarer Durchschnitte zu zeigen. Seien dazu zun\"achst $A_i\in\A^*,i\geq1$ disjunkt. Setze $B_n:=\displaystyle\bigcup_{i=1}^nA_i$ und $\displaystyle B:=\bigcup_{i\geq1}A_i$. Dann gilt f\"ur $M\in\mathcal{P}(\Omega)$ und alle $n\geq1$
\begin{align*}
    \mu^*(M)&=\mu^*(M\cap B_n)+\mu^*(M\cap B_n^c)\\
    &=\mu^*\left(M\cap\bigcup_{i=1}^nA_i\right)+\mu^*\left(M\cap\left(\bigcup_{i=1}^nA_i\right)^c\right)\\
    &=\sum_{i=1}^n\mu^*(M\cap A_i)+\mu^*\left(M\cap\left(\bigcup_{i=1}^nA_i\right)^c\right)\\
    &\geq\sum_{i=1}^n\mu^*(M\cap A_i)+\mu^*\left(M\cap\left(\bigcup_{i\geq1}A_i\right)^c\right)
\end{align*}
da $B_n\in\A^*$ (Algebra). F\"ur $n\to\infty$ folgt 
\begin{align*}
    \mu^*(M)&\geq\sum_{i\geq1}\mu^*(M\cap A_i)+\mu^*\left(M\cap\left(\bigcup_{i\geq1}A_i\right)^c\right)\\
    &\overset{2.16.}{=}\mu^*\left(M\cap\bigcup_{i\geq1}A_i\right)+\mu^*\left(M\cap\left(\bigcup_{i\geq1}A_i\right)^c\right)
\end{align*}
und damit schlie\ss{}lich $\displaystyle\bigcup_{i\geq1}A_i\in\A^*$.\newline\newline
Seien nun $A_i\in\A^*,i\geq1$ allgemein (also nicht unbedingt disjunkt). Setze $B_1:=A_1$ und $\displaystyle B_n:=A_n\setminus\left(\bigcup_{i=1}^{n-1}A_i\right)$ f\"ur alle $n\geq2$. Mit Lemma 2.15 gilt $B_n\in\A^*$ f\"ur alle $n\geq1$ und insbesondere sind die $B_n,n\geq1$ disjunkt und $\displaystyle\bigcup_{n\geq1}B_n=\bigcup_{i\geq1}A_i$. Die Aussage folgt mit der ersten Fall. \qed

\paragraph{2.18. Lemma:}F\"ur $A\in\A_0$ gilt $\mu(A)=\mu^*(A)$.

\paragraph{Beweis:}Laut Konstruktion (siehe 2.11) gilt $\mu^*(A)\leq\mu(A)$. Sei also $\eps>0$ und w\"ahle $A_i\in\A_0,i\geq1$, sodass $A\displaystyle\subseteq\bigcup_{i\geq1}A_i$ und
$$\sum_{i\geq1}\mu(A_i)\leq\mu^*(A)+\eps$$
Da $\mu$ als Pr\"ama\ss{} monoton und $\sigma$-additiv (und damit auch $\sigma$-subadditiv, siehe \"Ubung) auf $\A_0$ ist, folgt
\begin{align*}
    \mu(A)&=\mu\left(A\cap\bigcup_{i\geq1}A_i\right)\\&=\mu\left(\bigcup_{i\geq1}(A\cap A_i)\right)\\&\leq\sum_{i\geq1}\mu(A\cap A_i)\\
    &\leq\sum_{i=1}\mu(A_i)\leq\mu^*(A)+\eps
\end{align*}
und die Aussage folgt f\"ur $\eps\searrow0$. \qed

\paragraph{2.19. Lemma:}$\A_0\subseteq \A^*$, also ist jede Menge in der Algebra $\A_0$ auch von au\ss{}en messbar. 

\paragraph{Beweis:}Sei $A\in\A_0$ und $M\in\mathcal{P}(\Omega)$. Zu zeigen ist
$$\mu^*(M\cap A)+\mu^*(M\cap A^c)\leq\mu^*(M)$$
Sei dazu $\eps>0$ und w\"ahle $A_i\in\A_0,i\geq1$, sodass $M\subseteq\displaystyle\bigcup_{i\geq1}A_i$ und $\displaystyle\sum_{i\geq1}\mu(A_i)\leq\mu^*(M)+\eps$. Setze $B_i:=A\cap A_i$ und $C_i:=A^c\cap A_i$. Dann sind $B_i$ und $C_i$ disjunkt und $B_i,C_i\in\A_0$ f\"ur $i\geq1$. Au\ss{}erdem gilt $\displaystyle\bigcup_{i\geq1}B_i=A\cap\bigcup_{i\geq1}A_i$ und damit $A\cap M\subseteq \displaystyle\bigcup_{i\geq1}B_i$. \"ahnliches gilt f\"ur $C_i$. Damit gilt
$$\mu^*(M\cap A)+\mu^*(M\cap A^c)\leq\sum_{i\geq1}\mu(B_i)+\mu(C_i)=\sum_{i\geq1}\mu(A_i)\leq\mu^*(M)+\eps$$
und die Aussage folgt f\"ur $\eps\searrow 0$. \qed

\paragraph{2.20. Satz:}Sei $\mu$ ein Pr\"ama\ss{} auf $(\Omega,\A_0)$ und seien $\mu^*$ das entsprechende \"au\ss{}ere Ma\ss{} und $\A^*$ die Familie der von au\ss{}en messbaren Mengen. Dann ist $(\Omega,\A^*,\mu^*)$ ein Ma\ss{}raum, $\A_0\subseteq\A^*$ und es gilt $\forall A\in\A_0:\mu(A)=\mu^*(A)$. 

\paragraph{Beweis:}Mit Lemma 2.17 ist $\A^*$ eine $\sigma$-Algebra. Mit Lemma 2.13 und Lemma 2.16 ($M=\Omega$) folgt, dass $\mu^*$ alle Eigenschaften eines Ma\ss{}es auf $(\Omega,\A^*)$ erf\"ullt. Mit Lemma 2.19 gilt $\A_0\subseteq\A^*$ und mit Lemma 2.18 gilt $\mu=\mu^*$ auf $\A_0$. \qed

\paragraph{2.21. Satz (Ma\ss{}erweiterungssatz von Carath\'eodory):}Sei $\mu$ ein Pr\"ama\ss{} auf einer Algebra $\A_0$, sodass $\exists A_n\in\A_0,n\geq1:\Omega=\displaystyle\bigcup_{n\geq1}A_n\text{ und }\forall n\geq1:\mu(A_n)<\infty$. Dann gibt es eine eindeutige Erweiterung $\mu^*$ auf $\sigma(\A_0)$ mit $\mu=\mu^*$ auf $\A_0$.

\paragraph{Beweis:}Es gilt $\sigma(\A_0)\subseteq\A^*$. Mit Satz 2.20 ist $\mu^*$ ein Ma\ss{} auf $(\Omega,\A^*)$ und damit auch auf $(\Omega,\sigma(\A_0))$, das auf $\A_0$ mit $\mu$ \"ubereinstimmt. Sei $\nu$ ein weiteres Ma\ss{}, das auf $\A_0$ mit $\mu$ \"ubereinstimmt. Dann stimmt $\nu$ auf $\A_0$ auch mit $\mu^*$ \"uberein. Beachte, dass $\A_0$ durchschnittsstabil ist, und damit die Voraussetzungen von Korollar 2.9 erf\"ullt sind. Damit gilt $\forall A\in\sigma(\A_0):\nu(A)=\mu^*(A)$. \qed

\chapter*{3. Ma\ss{} auf $\R$}
\addcontentsline{toc}{chapter}{3. Ma\ss{} auf $\R$}

\section*{Motivation}
\addcontentsline{toc}{section}{Motivation}

\paragraph{3.1. Beispiel:}Betrachte den messbaren Raum $(\mathbb{N},\mathcal{P}(\mathbb{N}))$ und eine Folge nicht-negativer reeller Zahlen $p_i\geq0,i\geq1$ mit $\sum_{i\geq1}p_i=1$. Definiere die Abbildung $\Pp:\mathcal{P}(\mathbb{N})\to[0,1]$ mit $A\mapsto\sum_{i\in A}p_i$. Dann ist $\Pp$ ein Wahrscheinlichkeitsma\ss{}, da $\Pp(\emptyset)=0$ (leere Summe), $\Pp(\mathbb{N})=1$ (per Konstruktion), und f\"ur $A_n\in\mathcal{P}(\mathbb{N}), n\geq1$ disjunkt
$$\Pp\left(\bigcup_{n\geq1}A_n\right)=\sum_{i\in\bigcup_{n\geq1}A_n}p_i\overset{\dagger}{=}\sum_{n\geq1}\sum_{i\in A_n}p_i=\sum_{n\geq1}\Pp(A_n)$$
wobei der Schritt in $\dagger$ aus dem folgenden Satz (cf. Analysis I?) folgt. Dieses Beispiel deckt alle diskreten Verteilungen auf $\mathbb{N}$ ab. Unser Ziel ist es, dieses Beispiel auf stetige Verteilungen auf $\R$ zu erweitern.

\paragraph{3.2. Satz (Umordnung absolut konvergenter Reihen):}Sei die Reihe von $a_n,n\geq1$ absolut konvergent und sei $b_n,n\geq1$ eine Umordnung der $a_n,n\geq1$ (i.e. es gibt eine Bijektion $f:\mathbb{N}\to\mathbb{N}$ mit $b_n=a_{f(n)}$). Dann ist die Reihe von $b_n,n\geq1$ absolut konvergent und es gilt
$$\sum_{n\geq1}a_n=\sum_{n\geq1}b_n$$

\paragraph{Beweis:} Sei $\eps>0$ und definiere
$$s_N:=\sum_{n=1}^Na_n\text{ und }t_N:=\sum_{n=1}^Nb_n$$
Dann gibt es $N\geq1$, sodass
$$\left|\sum_{n\geq1}a_n-s_N\right|<\eps$$
Da die Reihe von $a_n,n\geq1$ auch absolut konvergiert k\"onnen wir $N\geq1$ so w\"ahlen, dass auch
$$\left|\sum_{n\geq1}|a_n|-(|a_1|+\hdots+|a_N|)\right|=\sum_{n\geq N+1}|a_n|<\eps$$
W\"ahle nun $M\geq1$ gro\ss{} genug, dass unter den $b_n,n=1,\hdots,M$ die Werte $a_1,\hdots,a_N$ alle vorkommen. F\"ur alle $m\geq M$ ist $t_m-s_N$ damit eine Summe, in der die Werte $a_1,\hdots,a_N$ nicht vorkommen und mit der Dreiecksungleichung folgt 
$$|t_m-s_N|\leq\sum_{n\geq N+1}a_n<\eps$$
Damit gilt $\forall m\geq M$
\begin{align*}
    \left|\sum_{n\geq1}a_n-t_m\right|&=\left|\sum_{n\geq1}a_n-s_N+s_N-t_m\right|\\
    &\leq\left|\sum_{n\geq1}a_n-s_N\right|+\left|s_N-t_m\right|<2\eps
\end{align*}
f\"ur $N$ hinreichend gro\ss{}. Da $\eps>0$ beliebig war, folgt die Aussage. \qed

\paragraph{3.2.$\frac{1}{2}$. Satz (Riemann'scher Umordnungssatz):}Sei die Reihe von $a_n,n\geq1$ konvergent, aber nicht absolut konvergent. Dann gibt es f\"ur jede Zahl $a\in\R$ eine Umordnung $b_n,n\geq1$, sodass
$$\sum_{n\geq1}b_n=a$$

\paragraph{Beweis:}siehe z.B. Theorem 22.7, Spivak, M. \textit{Calculus}. ? edn., pp. ?. \qed

\section*{Messen von Intervallen}
\addcontentsline{toc}{section}{Messen von Intervallen}
Sei im Folgenden $F:\R\to\R$ eine beliebige monoton-nichtfallende und rechtsseitig stetige Funktion. Beachte, dass damit linksseitige Grenzwerte f\"ur $F$ existieren. G\"angige Beispiele sind z.B.: $F(x)=x$ oder $F(x)=\int_{-\infty}^xe^{-t^2/2}$. Wir suchen nun eine m\"oglichst "kleine" $\sigma$-Algebra $\A$ auf $\R$, die alle Intervalle der Form $[a,b]$ enth\"alt und ein Ma\ss{} $\mu:\A\to[0,\infty]$, sodass $\mu([a,b])=F(b)-F(a)$.

\paragraph{3.3. Definition:}Sei $-\infty\leq a\leq b\leq\infty$. Definiere das halboffene Interval
\begin{align*}
    (a,b\rangle:=
\begin{cases}
    (a,b] &\text{falls }b<\infty \\
    (a,\infty)&\text{falls }b=\infty
\end{cases}
\end{align*}
sowie die Familie der halboffenen Intervalle
$$\mathcal{J}:=\left\{(a,b\rangle:-\infty\leq a\leq b\leq\infty\right\}
$$
und die Mengenfunktion $\phi:\mathcal{J}\to[0,\infty]$ mit
\begin{align*}
    (a,b\rangle\mapsto
    \begin{cases}
        F(b)-F(a)&\text{falls }a<b\\
        0&\text{falls }a=b
    \end{cases}
\end{align*}

\paragraph{3.4. Lemma:}$\phi$ ist $\sigma$-additiv auf $\mathcal{J}$.

\paragraph{Beweis:}Seien $J_n\in\mathcal{J},n\geq1$, disjunkt, sodass auch $\bigcup_{n\geq1}J_n\in\mathcal{J}$. Zeige 
$$\phi\left(\bigcup_{n\geq1}J_n\right)=\sum_{n\geq1}\phi(J_n)$$
Da $\bigcup_{n\geq1}J_n\in\mathcal{J}$, k\"onnen wir $\bigcup_{n\geq1}J_n=(a,b\rangle$ schreiben. Seien o.B.d.A. alle $J_n=(a_i,b_i\rangle$ nicht-leer und aufsteigend geordnet, sodass
$$a=a_1<b_1=a_2<\hdots<b_{n-1}=a_n<b_n=b$$ 
Dann gilt 
$$\sum_{n\geq1}\phi(J_n)=\sum_{n\geq1}[F(b_n)-F(a_n)]=F(b_n)-F(a_1)=F(b)-F(a)=\phi\left(\bigcup_{n\geq1}J_n\right)$$
\qed

\paragraph{Bemerkung:}Beachte, dass $\bigcup_{n\geq1}J_n\in\mathcal{J}$ hier eine notwendige Voraussetzung ist, da $\mathcal{J}$ keine Algebra (bzw. $\sigma$-Algebra) ist. Wir erweitern $\mathcal{J}$ zun\"achst konstruktiv zu einer Algebra $\mathcal{J}^*$ und die Mengenfunktion $\phi:\mathcal{J}\to[0,\infty]$ zu einem Pr\"ama\ss{} $\phi^*:\mathcal{J}^*\to[0,\infty]$. Sp\"ater liefert uns dann ein (nicht-konstruktiver) Satz (Ma\ss{}erweiterungssatz von Carath\'eodory) eine Erweiterung von $\phi^*$ zu einem Ma\ss{} $\mu:\sigma(\mathcal{J}^*)\to[0,\infty]$.

\paragraph{3.5. Definition:}Definiere die Mengenfamilie
$$\mathcal{J}^*:=\left\{\bigcup_{i=1}^nJ_i:n\in\mathbb{N},J_i\in\mathcal{J},J_i\text{ disjunkt f\"ur }i=1,\hdots,n\right\}$$
aller endlichen disjunkten Vereinigungen von halboffenen Intervallen. 

\paragraph{3.6. Lemma:}$\mathcal{J}^*$ ist eine Algebra auf $\R$ mit $\mathcal{J}\subseteq\mathcal{J}^*$.

\paragraph{Beweis:}Die Eigenschaften $\mathcal{J}\subseteq\mathcal{J}^*$ und $\R\in\mathcal{J}^*$ sind trivial. Zeige also die Abgeschlossenheit bez\"uglich Komplementbildung und endlichen Durchschnitten. \newline\newline
F\"ur $(a,b\rangle\in\mathcal{J}$ gilt $(a,b\rangle^c=(-\infty,a\rangle\cup(b,\infty\rangle\in\mathcal{J}^*$. Sei also $A=\bigcup_{i=1}^n(a_i,b_i\rangle\in\mathcal{J}^*$ mit $(a_i,b_i\rangle$ disjunkt, nicht-leer und aufsteigend geordnet, d.h.
$$-\infty\leq a_i<b_1\leq a_2<\hdots\leq a_n<b\leq\infty$$
Dann gilt $A^c=(-\infty,a_1\rangle\cup(b_1,a_2\rangle\cup\hdots\cup(b_{n-1},a_n\rangle\cup(b_n,\infty\rangle\in\mathcal{J}^*$ per Definition von $\mathcal{J}^*$.\newline\newline
Seien nun $A,B\in\mathcal{J}^*, A=\bigcup_{i=1}^n(a_i,b_i\rangle,B=\bigcup_{j=1}^m(c_j,d_j\rangle$ jeweils endliche Vereinigungen disjunkter, halboffener Intervalle, d.h. $(a_i,b_i\rangle$ paarweise disjunkt f\"ur $i=1,\hdots,n$ und $(c_j,d_j\rangle$ paarweise disjunkt f\"ur $j=1\hdots,m$. Dann gilt
$$A\cap B=\bigcup_{i=1}^n\bigcup_{j=1}^m(a_i,b_i\rangle\cap (\alpha_i,d_i\rangle=\bigcup_{i=1}^n\bigcup_{j=1}^m\left(\max\{a_i,c_j\},\max\{b_i,d_j\}\right\rangle$$
Die Disjunktheit der Intervalle $\left(\max\{a_i,c_j\},\max\{b_i,d_j\}\right\rangle$ f\"ur $i=1,\hdots,n$ und $j=1,\hdots,m$ ist leicht nachzupr\"ufen (Widerspruchsargument). \qed

\paragraph{3.7. Definition:}Sei $A=\bigcup_{i=1}^n(a_i,b_i\rangle\in\mathcal{J}^*$ mit $(a_i,b_i\rangle$ disjunkt f\"ur $i=1,\hdots,n$. Definiere nun die Erweiterung $\phi^*:\mathcal{J}^*\to[0,\infty]$ mit $A\mapsto\sum_{i=1}^n\phi\left((a_i,b_i\rangle\right)$. Da $\mathcal{J}\subseteq\mathcal{J}^*$ gilt $\phi=\phi^*$ auf $\mathcal{J}$.

\paragraph{3.8. Proposition:}$\phi^*(A)$ ist wohldefiniert f\"ur alle $A\in\mathcal{J}^*$ und insbesondere unabh\"angig von der Darstellung von $A$.
 
 \paragraph{Beweis:}Sei $A=\bigcup_{i=1}^n(a_i,b_i\rangle=\bigcup_{j=1}^m(\alpha_i,d_i\rangle\in\mathcal{J}^*$. Schreibe f\"ur $i=1,\hdots,n$
$$(a_i,b_i\rangle=(a_i,b_i\rangle\cap A=(a_i,b_i\rangle\cap\bigcup_{j=1}^m(c_j,d_j\rangle=\bigcup_{j=1}^m\left[(c_j,d_j\rangle\cap(a_i,b_i\rangle\right]$$
 und analog f\"ur $j=1,\hdots,m$
 $$(c_j,d_j\rangle=\bigcup_{i=1}^n\left[(c_j,d_j\rangle\cap(a_i,b_i\rangle\right]$$
 Es folgt mit der $\sigma$-Additivit\"at von $\phi$ auf $\mathcal{J}$
 \begin{align*}
     \phi^*\left(\bigcup_{i=1}^n(a_i,b_i\rangle\right)&=\sum_{i=1}^n\phi((a_i,b_i\rangle)\\
     &=\sum_{i=1}^n\phi\left(\bigcup_{j=1}^m\left[(c_j,d_j\rangle\cap(a_i,b_i\rangle\right]\right)\\
     &=\sum_{i=1}^n\sum_{j=1}^m\phi((a_i,b_i\rangle\cap(c_j,d_j\rangle)\\
     &=\sum_{j=1}^m\sum_{i=1}^n\phi((a_i,b_i\rangle\cap(c_j,d_j\rangle)\\
     &=\sum_{j=1}^m\phi\left(\bigcup_{i=1}^n\left[(c_j,d_j\rangle\cap(a_i,b_i\rangle\right]\right)\\
     &=\sum_{j=1}^m\phi((c_j,d_j\rangle)=\phi^*\left(\bigcup_{j=1}^m(c_j,d_j\rangle\right)
 \end{align*}
 \qed
 
 \section*{Erzeugung von Ma\ss{}en auf $\R$}
 \addcontentsline{toc}{section}{Erzeugung von Ma\ss{}en auf $\R$}
 Wir m\"ochten $\phi^*:\mathcal{J}^*\to[0,\infty]$ nun zu einem Ma\ss{} $\mu:\sigma(\mathcal{J}^*)\to[0,\infty]$ erweitern. Wir fordern dazu folgende Eigenschaften von $\phi^*$:
 
\paragraph{3.9. Lemma:}
\begin{enumerate}[label=(\roman*)]
    \item $\phi^*(\emptyset)=0$
    \item F\"ur $A_n\in\mathcal{J}^*,n\geq1$ disjunkt, mit $\bigcup_{n\geq1}A_n\in\mathcal{J}^*$ gilt $\phi^*\left(\bigcup_{n\geq1}A_n\right)=\sum_{n\geq1}\phi^*(A_n)$ ($\phi^*$ ist $\sigma$-additiv auf $\mathcal{J}^*$).
    \item $\exists B_n\in\mathcal{J}^*,n\geq1$, sodass $\R=\bigcup_{n\geq1}B_n$ und $\phi^*(B_n)<\infty$ f\"ur $n\geq1$ (Das Pr\"ama\ss{} $\phi^*$ ist $\sigma$-endlich auf $(\R,\mathcal{J}^*) $).
\end{enumerate}

\paragraph{Beweis:}
\begin{enumerate}[label=(\roman*)]
    \item $\phi(\emptyset)=\phi^*((1,1\rangle)=\phi((1,1\rangle)=0$ per Definition von $\phi$.
    \item Setze $A:=\bigcup_{n\geq1}A_n$. Da $A\in\mathcal{J}^*$ k\"onnen wir $A=\bigcup_{j=1}^k(c_j,d_j\rangle$ schreiben mit $(c_j,d_j\rangle$ disjunkt f\"ur $j=1,\hdots,k$. Schreibe weiters $A_n=\bigcup_{i=1}^{m_n}(a_i^{(n)},b_i^{(n)}\rangle$ mit $(a_i^{(n)},b_i^{(n)}\rangle$ disjunkt f\"ur $i=1,\hdots,m_n$ und alle $n\geq1$. Es gilt 
    $$(c_j,d_j\rangle=(c_j,d_j\rangle\cap A=\bigcup_{n\geq1}\left(A_n\cap(c_j,d_j\rangle\right)$$
    und
    $$A_n=A_n\cap A=\bigcup_{i=1}^{m_n}\bigcup_{j=1}^k\left((a_i^{(n)},b_i^{(n)}\rangle\cap(c_j,d_j\rangle\right)$$
    und mit Lemma 3.4 folgt (mit der Eigenschaft, dass $\mathcal{J}$ durchschnittsstabil ist)
    $$\phi((c_j,d_j\rangle)=\sum_{n\geq1}\sum_{i=1}^{m_n}\phi\left((a_i^{(n)},b_i^{(n)}\rangle\cap(c_j,d_j\rangle\right)$$
    und per Definition von $\phi^*$ gilt
    $$\phi^*(A_n)=\sum_{i=1}^{m_n}\sum_{j=1}^k\phi\left((a_i^{(n)},b_i^{(n)}\rangle\cap(c_j,d_j\rangle\right)$$
    Mit Satz 3.2 (??, und $\sigma$-Endlichkeit?) folgt 
    \begin{align*}
        \phi^*(A)&=\sum_{i=1}^k\phi((c_j,d_j\rangle)\\
        &=\sum_{j=1}^k\sum_{n\geq1}\sum_{i=1}^{m_n}\phi\left((a_i^{(n)},b_i^{(n)}\rangle\cap(c_j,d_j\rangle\right)\\
        &\overset{\text{Satz 3.2}}{=}\sum_{n\geq1}\sum_{i=1}^{m_n}\sum_{j=1}^k\phi\left((a_i^{(n)},b_i^{(n)}\rangle\cap(c_j,d_j\rangle\right)\\
        &=\sum_{n\geq1}\phi^*(A_n)
    \end{align*}
    \item W\"ahle hier $B_n:=(-n,n\rangle\in\mathcal{J}^*$. Dann gilt $\R=\bigcup_{n\geq1}(-n,n\rangle$ und $\phi^*(B_n)=F(n)-F(-n)<\infty$ f\"ur alle $n\geq1$. \qed
\end{enumerate}

\paragraph{3.10. Lemma:}Folgende Aussagen sind \"aquivalent:
\begin{enumerate}[label=(\roman*)]
    \item F\"ur $A_n\in\mathcal{J},n\geq1$ disjunkt, mit $A=\bigcup_{n\geq1}A_n\in\mathcal{J}$ gilt $\phi\left(\bigcup_{n\geq1}A_n\right)=\sum_{n\geq1}\phi(A_n)$
    \item F\"ur $A_n\in\mathcal{J}^*,n\geq1$ disjunkt, mit $A=\bigcup_{n\geq1}A_n\in\mathcal{J}^*$ gilt $\phi^*\left(\bigcup_{n\geq1}A_n\right)=\sum_{n\geq1}\phi^*(A_n)$
\end{enumerate}

\paragraph{Beweis:} folgt. Ich denke (ii)$\implies$(i) folgt mit der Inklusion $\mathcal{J}\subseteq\mathcal{J}^*$ und $\phi(A)=\phi^*(A)$ f\"ur $A\in\mathcal{J}$. (i)$\implies$(ii) folgt aus dem Beweis von Lemma 3.9 (insofern dieser stimmt)?

\paragraph{3.11. Lemma:}Sei $I\subseteq\mathbb{N}$ nicht-leer und seien $(a_i,b_i\rangle\in\mathcal{J},i\in I$ nicht-leer und disjunkt, sodass $\bigcup_{i\in I}(a_i,b_i\rangle\subseteq(a,b\rangle\in\mathcal{J}$. Dann folgt 
$$\sum_{i\in I}F(b_i)-F(a_i)\leq F(b)-F(a)$$

\paragraph{Beweis:}
\begin{enumerate}[label=\Roman*.]
    \item $I$ endlich\newline
    Da $I$ in Bijektion zu $\{1,\hdots,n\}$ steht, setze o.B.d.A. $I=\{1,\hdots,n\}$ und ordne die Intervalle aufsteigend, also
    $$a\leq a_1<b_1\leq\hdots\leq a_n<b_n\leq b$$
    Dann gilt
    $$\sum_{i=1}^nF(b_i)-F(a_i)=-F(a_1)+F(b_1)-\hdots-F(a_n)+F(b_n)\leq F(b_n)-F(a_1)\leq F(b)-F(a)$$
    da $F(b_k)\leq F(a_{k+1})$ f\"ur $k=1,\hdots,n-1$. 
    \item $I$ abz\"ahlbar unendlich\newline
    Da $I$ in Bijektion zu $\mathbb{N}$ steht, setze o.B.d.A $I=\mathbb{N}$. Dann gilt
    $$\sum_{i\geq1}F(b_i)-F(a_i)=\lim_{n\to\infty}\sum_{i=1}^nF(b_i)-F(a_i)\leq\lim_{n\to\infty}F(b)-F(a)=F(b)-F(a)$$
    mit dem I. Teil. \qed
\end{enumerate}

\paragraph{3.12. Lemma:} Sei $I\subseteq\mathbb{N}$ nicht-leer und seien $(a_i,b_i\rangle\in\mathcal{J},i\in I$ nicht-leer, sodass $\bigcup_{i\in I}(a_i,b_i\rangle\supseteq(a,b\rangle\in\mathcal{J}$. Dann folgt
$$F(b)-F(a)\leq\sum_{i\in I}F(b_i)-F(a_i)$$

\paragraph{Beweis:}
\begin{enumerate}[label=\Roman*.]
    \item $I$ endlich\newline
    Sei wie im Beweis von Lemma 3.11 o.B.d.A $I=\{1,\hdots,n\}$. Seien au\ss{}erdem die Intervalle $(a_i,b_i\rangle$ disjunkt und aufsteigend geordnet. Dann gilt $a\in(a_k,b_k\rangle$ und $b\in(a_\ell,b_\ell\rangle$ f\"ur $1\leq k\leq\ell\leq n$ und $b_i=a_{i+1}$ f\"ur alle $i=k,\hdots,\ell-1$. Es folgt
    \begin{align*}
        \sum_{i=1}^nF(b_i)-F(a_i)&\geq\sum_{i=k}^\ell F(b_i)-F(a_i)\\
        &=-F(a_k)+F(b_\ell)\\&\geq F(b)-F(a)
    \end{align*}
    \item $I$ abz\"ahlbar unendlich, $a,b<\infty$\newline
    Sei wie im Beweis von Lemma 3.11 o.B.d.A. $I=\mathbb{N}$. Sei $\eps>0$. Da $F$ rechtsstetig ist, gibt es $\delta,\delta_i>0,i\geq1$, sodass
    $$F(a+\delta)<F(a)+\eps,\ F(b_i+\delta_i)<F(b_i)+\frac{\eps}{2^i}$$
    Beachte, dass $(a+\delta,b\rangle\bigcup_{i\geq1}(a_i,b_i+\delta_i\rangle$. Nun existiert eine endliche Menge $J\subseteq\mathbb{N}$ (einfache \"Uberlegung), sodass
    $$(a+\delta,b\rangle\subseteq\bigcup_{j\in J}(a_j,b_j+\delta_j\rangle$$
    Damit folgt
    \begin{align*}
        F(b)-F(a+\delta)&\leq\sum_{j\in J}F(b_j+\delta_j)-F(a_j)\\
        &\leq\eps+\sum_{i\geq1}F(b_i)-F(a_i)
    \end{align*}
    Die Aussage folgt f\"ur $\eps\searrow0$.
    \item $I$ abz\"ahlbar unendlich, $a$ oder $b$ unendlich\newline
    Es gibt hier drei F\"alle:
    $$(a,b\rangle=
    \begin{cases}
        (-\infty,b] \\
        (a,\infty) \\
        (-\infty,\infty)
    \end{cases}$$
    Sei $\eps>0$. Wegen den Eigenschaften von $F$ gibt es $s,t\in\R$, sodass $a\leq s\leq t\leq b$ und 
    $$F(s)\leq F(a)+\eps,\ F(t)\leq F(b)-\eps$$
    Es folgt weiters
    \begin{gather*}
        (s,t\rangle\subseteq(a,b\rangle\subseteq\bigcup_{i\geq1}(a_i,b_i\rangle\\
        F(b)-F(a)\geq F(t)-F(s)\geq F(b)-F(a)-2\eps
    \end{gather*}
    Die Aussage folgt f\"ur $\eps\searrow0$. \qed
\end{enumerate}

\paragraph{Bemerkung:} Damit k\"onnen wir $\phi^*$ eindeutig zu einem Ma\ss{} $\mu$ auf $\sigma(\mathcal{J}^*)$ erweitern. Beachte (einfache \"Uberlegung)
$$\mathcal{J}\subseteq\mathcal{J}^*,\ \mathcal{J}^*\subseteq\sigma(\mathcal{J})\implies\sigma(\mathcal{J})=\sigma(\mathcal{J}^*)$$

\paragraph{3.13. Definition:}Definiere die Borel'sche $\sigma$-Algebra auf $\R$ durch
$$\borel:=\sigma(\mathcal{J}^*)$$

\paragraph{3.14. Proposition:}Es gilt $\borel=\sigma(\mathcal{J}_i)$ f\"ur $i=1,\hdots,n$ und
\begin{align*}
    &\mathcal{J}_1:=\{(a,b):-\infty\leq a\leq b\leq\infty\}\\
    &\mathcal{J}_2:=\{[a,b]:-\infty< a\leq b<\infty\}\\
    &\mathcal{J}_3:=\{(-\infty,b]:b\in\R\}\\
    &\mathcal{J}_4:=\{(-\infty,b):b\in\R\}
\end{align*}

\paragraph{Beweis:} nur f\"ur $\mathcal{J}_1$, Rest \"Ubung!
\begin{enumerate}[label=\Roman*.]
    \item $\mathcal{J}_1\subseteq\sigma(\mathcal{J})$\newline
    $(a,b)=\bigcup_{n\geq1}\left(a,b-\frac{1}{n}\right]\in\sigma(\mathcal{J})$, da $\left(a,b-\frac{1}{n}\right]=\left(a,b-\frac{1}{n}\right\rangle\in\mathcal{J}$ f\"ur alle $n\geq1$.
    \item $\mathcal{J}_1\supseteq\sigma(\mathcal{J})$
    \begin{enumerate}[label=(\alph*)]
        \item $a,b<\infty$: $(a,b\rangle=(a,b]=\bigcap_{n\geq1}\left(a,b+\frac{1}{n}\right)\in\sigma(\mathcal{J}_1)$.
        \item $a=-\infty,b<\infty$: $(a,b\rangle=(-\infty,b]=\bigcap_{n\geq1}\left(-\infty,b+\frac{1}{n}\right)\sigma(\mathcal{J}_1)$
        \item $a<\infty,b=\infty$: $(a,b\rangle=(a,\infty)\in\mathcal{J}_1\subseteq\sigma(\mathcal{J}_1)$
        \item $a=-\infty,b=\infty$: $(a,b\rangle=\R\in\mathcal{J}_1\subseteq\sigma(\mathcal{J}_1)$ \qed
    \end{enumerate} 
\end{enumerate}

\paragraph{3.15. Proposition:}Es gilt $\borel=\sigma(\mathcal{O})$, mit $\mathcal{O}=\{O\subseteq\R:O\text{ offen}\}$.

\paragraph{Beweis:} folgt.

\paragraph{3.16. Korollar:}$\borel$ enth\"alt alle einpunktigen, offenen und abgeschlossenen Teilmengen von $\R$.

\paragraph{Beweis:}Die Inklusion der offenen und abgeschlossenen Teilmengen folgt aus Proposition 3.15 und der Abgeschlossenheit bez\"uglich Komplementbildung. \newline
Au\ss{}erdem gilt 
$$\{x\}=\bigcap_{n\geq1}\left(x-\frac{1}{n},x+\frac{1}{n}\right)\in\borel$$
mit Proposition 3.14. \qed

\paragraph{3.17. Satz:}Sei $F:\R\to\R$ monoton nicht-fallend und rechtsstetig. Dann existiert genau ein $\sigma$-endliches Ma\ss{} $\mu_F:\borel\to[0,\infty]$ mit $\mu_F((a,b\rangle)=F(b)-F(a)$.

\paragraph{Beweis:}Existenz und Eindeutigkeit folgen aus dem Ma\ss{}erweiterungssatz von Carath\'eodory. Au\ss{}erdem gilt $\R=\bigcup_{n\geq1}(-n,n\rangle$ und $\forall n\geq1:\mu_F((-n,n\rangle)=F(n)-F(-n)<\infty$. \qed

\paragraph{3.18. Definition:}Das Lebesgue-Ma\ss{} $\lambda(\cdot)$ (auch $\operatorname{vol}(\cdot)$) auf $\R$ ist definiert durch das von der Funktion $F(x)=x$ induzierte Ma\ss{} (gem\"a\ss{} Satz 3.17).

\paragraph{3.19. Definition:}Sei $F:\R\to[0,1]$ monoton nicht-fallend und rechtsstetig, sodass zus\"atzlich $\lim_{x\to-\infty}F(x)=0$ und $\lim_{x\to\infty}F(x)=1$. Dann nennt man $F$ eine Verteilungsfunktion (cdf). Mit Satz 3.17 induziert jede Verteilungsfunktion ein Wahrscheinlichkeitsma\ss{} $\Pp$ auf $\R$.

\paragraph{3.20. Satz} Sei $\varphi$ ein $\sigma$-endliches Ma\ss{} auf $(\R,\borel)$. Dann existiert eine Funktion $F:\R\to\R$, sodass
\begin{itemize}
    \item $F$ ist monoton nichtfallend.
    \item $F$ ist rechtsseitig stetig.
    \item F\"ur $\displaystyle F(-\infty)=\lim_{x\to-\infty}F(x)$ und $F(\infty)=\displaystyle\lim_{x\to\infty}F(x)$ gilt f\"ur alle $-\infty\leq a<b\leq\infty$.
    $$\varphi\left((a,b\rangle\right)=F(b)-F(a)$$
\end{itemize}
\paragraph{Beweis:} 
\begin{enumerate}[label=\Roman*.]
    \item Fall: $\varphi$ endlich.\newline
    Setze $F(x):=\varphi\left((-\infty,x]\right)$ f\"ur $x\in\R$. Da $\varphi$ endlich ist, ist $F$ reellwertig. Die Monotonie von $F$ folgt aus der Monotonie von $\varphi$ (siehe Satz 1.7). Die Rechtsstetigkeit von $F$ folgt der Stetigkeit von oben (siehe Satz 1.9). Schlie\ss{}lich gilt f\"ur $-\infty<a<b<\infty$
    \begin{align*}
        \varphi\left((a,b]\right)&=\varphi\left((-\infty,b]\setminus(-\infty,a]\right)\\
        &=\varphi((-\infty,b])-\varphi((-\infty,a])=F(b)-F(a)\\ \\
        \varphi((-\infty,b))&\overset{\text{S.V.U.}}{=}\lim_{x\to-\infty}\varphi((x,b])\\
        &=\lim_{x\to-\infty}F(b)-F(x)=F(b)-F(-\infty)\\ \\
        \varphi((a,\infty))&\overset{\text{S.V.U.}}{=}\lim_{x\to\infty}\varphi((a,x])\\
        &=\lim_{x\to\infty}F(x)-F(a)=F(\infty)-F(a) \\ \\
        \varphi((-\infty,\infty))&\overset{\text{S.V.U.}}{=}\lim_{x\to\infty}\varphi((-x,x])\\
        &=\lim_{x\to\infty}F(x)-F(-x)=F(\infty)-F(-\infty)
    \end{align*}
    \item Fall: $\varphi(\R)=\infty$.\newline
    Setze hier
    \begin{align*}
        F(x):=
        \begin{cases}
            \varphi((0,x])&\text{ falls }x\geq0 \\
            -\varphi((x,0])&\text{ falls }x<0
        \end{cases}
    \end{align*}
    Die Monotonie und Rechtssteitgkeit folgen wie im I. Fall. F\"ur die dritte Eigenschaft argumentiere wie im I. Fall und betrachte jeweils die F\"alle
    \begin{align*}
        0\leq a\leq b<\infty \\ 
        -\infty<a<0\leq b<\infty \\ 
        -\infty<a<b\leq 0
    \end{align*}
    \qed
\end{enumerate}

\paragraph{3.21. Korollar:}Sei $(\R,\borel,\mu)$ ein $\sigma$-endlicher Ma\ss{}raum und $\mu((a,b\rangle)=F(b)-F(a)$ f\"ur eine monoton nicht-fallende, rechtsstetige Funktion $F:\R\to\R$. Dann gilt f\"ur $a\in\R$:
$$\mu(\{a\})=F(a)-\lim_{x\nearrow a}F(x)$$
Insbesondere ist $\{a\}$ ein Atom von $\mu$, wenn $F$ eine Sprungstelle bei $a$ hat.

\paragraph{Beweis:}Schreibe $\{a\}=\bigcap_{n\geq1}\left(a-\frac{1}{n},a\right\rangle$ und beachte, dass $\mu((a-1,a\rangle)=F(a)-F(a-1)<\infty$. Mit der Stetigkeit von oben (Satz 1.9) gilt
$$\mu(\{a\})=\lim_{n\to\infty}\mu\left(\left(a-\frac{1}{n},a\right\rangle\right)=\lim_{n\to\infty}\left[F(a)-F\left(a-\frac{1}{n}\right)\right]=F(a)-\lim_{x\nearrow a}F(x)$$
wobei die letze Gleichung aus der Definition eines einseitigen Grenzwertes folgt (cf. Analysis). \qed

\chapter*{4. Messbare Abbildungen und Zufallsvariablen}
\addcontentsline{toc}{chapter}{4. Messbare Abbildungen und Zufallsvariablen}

Betrache im folgenden Kapitel jeweils einen Ma\ss{}raum $(\Omega,\A,\mu)$ und eine Abbildung $f:\Omega\to\Omega'$.

 \section*{Urbildoperator und Messbarkeit}
 \addcontentsline{toc}{section}{Urbildoperator und Messbarkeit}
\paragraph{4.1. Definition:}Sei $f:\Omega\to\Omega'$. F\"ur $A'\subseteq\Omega'$ definiere das Urbild von $A'$ unter $f$ als
$$f^{-1}(A'):=\{\omega\in\Omega:f(\omega)\in A'\}\subseteq\Omega$$
Kurschreibweise: $f^{-1}(A')=f^{-1}A'=\{f\in A'\}$. Beachte, dass der Urbild-Operator nicht die inverse Funktion ist (im Gegensatz zur Inversen ist das Urbild immer definiert).

\paragraph{4.1.$\frac{1}{2}$. Proposition :}Der Urbild-Operator kommutiert mit Mengenoperationen, i.e.
\begin{enumerate}[label=(\roman*)]
    \item $\displaystyle f^{-1}\left(\bigcup_{n\geq1}A_n'\right)=\bigcup_{n\geq1}f^{-1}(A_n')$
    \item $\displaystyle f^{-1}(A'^c)=\left(f^{-1}(A')\right)^c$
    \item $\displaystyle f^{-1}\left(\bigcap_{n\geq1}A_n'\right)=\bigcap_{n\geq1}f^{-1}(A_n')$
\end{enumerate}

\paragraph{Beweis:} \"Ubung! (iii) folgt aus (i), (ii) und de Morgan.

\paragraph{4.2. Definition:}Betrachte zwei messbare R\"aume $(\Omega,\A)$ und $(\Omega',\A')$. Eine Abbildung $f$ hei\ss{}t $\A\textendash\A'$-messbar, falls 
$$\forall A'\in\A':f^{-1}(A')\in\A$$
Falls $X:\pspace\to(\Omega',\A')$ eine messbare Abbildung von einem Wahrscheinlichkeitsraum nach $\Omega'$ ist, nennt man $X$ eine $\Omega'$-wertige Zufallsvariable (e.g. $\Omega'=\R$ oder $\Omega'=\mathbb{C}$). Definiert man eine Abbildung als $\A\textendash\A'$-messbar, so schreibt man oft $f:(\Omega,\A)\to(\Omega',\A')$.

\paragraph{4.3. Definition:}Sei $I$ eine beliebige, nicht-leere Indexmenge und seien $f_i:\Omega\to\Omega',i\in I$. Definiere die von den $f_i,i\in I$ erzeugte $\sigma$-Algebra als
$$\sigma(f_i,i\in I):=\sigma\left(\left\{f_i^{-1}(A'):A'\in\A',i\in I\right\}\right)$$
 
 \paragraph{Bemerkung:}
 \begin{enumerate}[label=(\roman*)]
     \item $\sigma(f_i,i\in I)$ ist die kleinste $\sigma$-Algebra auf $\Omega$ f\"ur die alle $f_i,i\in I$ messbar sind.
     \item Ist $\A$ eine $\sigma$-Algebra auf $\Omega$, dann gilt: $\forall i\in I:f_i$ ist $\A\textendash\A'$-messbar $\iff$ $\sigma(f_i,i\in I)\subseteq\A$
     \item $\sigma(f)=\{f^{-1}(A'):A'\in\A'\}$, da der Urbild-Operator mit Mengenoperationen kommutiert. 
 \end{enumerate}
 
 \paragraph{4.4. Proposition:}Betrachte zwei messbare R\"aume $(\Omega,\A)$, $(\Omega',\A')$, wobei $\A'=\sigma(\mathcal{M'})$ f\"ur eine Mengenfamilie $\mathcal{M'}\subseteq\mathcal{P}(\Omega')$, sowie eine Abbildung $f:\Omega\to\Omega'$. Setze $\M:=\{f^{-1}(M'):M'\in\M'\}$. Dann gilt $\sigma(f)=\sigma(\M)$ und insbesondere ist $f$ genau dann $\A\textendash\A'$-messbar, wenn $\M\subseteq\A$.
 
 \paragraph{Beweis:}
 \begin{enumerate}[label=\Roman*.]
     \item $\sigma(\M)\subseteq\sigma(f)$\newline
     Es gilt $\M'\subseteq\sigma(\M')$ und damit 
     $$\M=\{f^{-1}(M'):M'\in\M'\}\subseteq\{f^{-1}(M'):M'\in\sigma(\M')\}=\sigma(f)$$
     Damit folgt $\sigma(\M)\subseteq\sigma(\sigma(f))=\sigma(f)$.
     \item $\sigma(\M)\supseteq\sigma(f)$\newline
     Setze $\G':=\{M'\in\sigma(\M'):f^{-1}(M')\in\sigma(\M)\}$ und zeige $\G'=\sigma(\M')$. Die Inklusion $\G'\subseteq\nobreak\sigma(\M')$ folgt sofort aus der Konstruktion. Es gen\"ugt also zu zeigen, dass $\G'$ eine $\sigma$-Algebra ist und $\M'\subseteq\G'$.
     \begin{enumerate}[label=(\roman*)]
         \item Sei $M'\in\M'$. Dann gilt $M'\in\sigma(\M')$ und $f^{-1}(M')\in\M$ per Definition von $\M$. Es folgt $f^{-1}(M')\in\sigma(\M')$ und damit $M'\in\G'$.
         \item Zeige, dass $\G'$ eine $\sigma$-Algebra ist.
         \begin{itemize}
             \item Es gilt $\Omega'\in\sigma(\M')$ und $f^{-1}(\Omega')=\Omega\in\sigma(\M)$.
             \item Abgeschlossenheit bez\"uglich Komplementbildung und abz\"ahlbaren Vereinigungen folgt aus Bemerkung (iii) oben.
         \end{itemize}
     \end{enumerate}
     \item Zur Messbarkeit.
     \begin{enumerate}[label=(\roman*)]
        \item Sei $f$ $\A\textendash\A'$-messbar. Es gilt mit Bemerkung (ii) oben
        $$\M=\{f^{-1}(M'):M'\in\M'\}\subseteq\{f^{-1}(M'):M'\in\sigma(\M')\}=\sigma(f)\subseteq\A$$
        \item Sei $\M\subseteq\A$. Damit gilt $\sigma(\M)=\sigma(f)\subseteq\sigma(\A)=\A$ und mit Bemerkung (ii) oben folgt die Aussage. \qed
     \end{enumerate}
 \end{enumerate}
 
 \paragraph{4.5. Lemma:}
 $$\sigma(f_i:i\in I)=\sigma\left(\left\{\bigcap_{j\in J}f_j^{-1}(A_j'):J\subseteq I, J\text{ endlich},A_j'\in\A'\text{ f\"ur }j\in J\right\}\right)$$
 
 \paragraph{Beweis:}
 \begin{enumerate}[label=\Roman*.]
     \item $\subseteq$\newline
     Es gilt $\{f_i^{-1}(A'):A'\in\A, i\in I\}\subseteq\left\{\bigcap_{j\in J}f_j^{-1}(A_j'):J\subseteq I, J\text{ endlich},A_j'\in\A'\text{ f\"ur }j\in J\right\}$.
     \item $\supseteq$\newline
     folgt sofort aus der Abgeschlossenheit bez\"uglich endlicher Durchschnitte und der Kommutativit\"at des Urbildoperators mit dem Durchschnittsoperator. \qed
 \end{enumerate}
 
 \paragraph{4.6. Proposition:}Betrachte messbare R\"aume $(\Omega_i,\A_i),i=1,2,3$ sowie messbare Abbildungen 
 \begin{align*}
     &f:(\Omega_1,\A_1)\to(\Omega_2,\A_2)\\
     &g:(\Omega_2,\A_2)\to(\Omega_3,\A_3)
 \end{align*} 
 Dann ist $g\circ f:\Omega_1\to\Omega_3$ auch $\A_1\textendash\A_3$-messbar.
 
 \paragraph{Beweis:}Sei $A_3\in\A_3$. Dann gilt
 \begin{align*}
     (g\circ f)^{-1}(A_3)&=\{\omega_1\in\Omega_1:(g\circ f)(\omega_1)\in A_3\}\\
     &=\{\omega_1\in\Omega_1:g(f(\omega_1))\in A_3\}\\
     &=\{\omega_1\in\Omega_1:f(\omega_1)\in g^{-1}(A_3)\}
 \end{align*}
 Laut Annahme gilt $g^{-1}(A_3)\in\A_2$ f\"ur alle $A_3\in\A_3$ und mit der Messbarkeit von $f$ folgt die Aussage. \qed
 
 \section*{Messbare Funktionen mit Werten in $\R$}
 \addcontentsline{toc}{section}{Messbare Funktionen mit Werten in $\R$}
 
 \paragraph{4.7. Lemma:}Seien $(X,d_X)$ und $(Y,d_Y)$ jeweils metrische R\"aume. Dann ist eine Abbildung $f:X\to Y$ genau dann stetig (bez\"uglich $d_X,d_Y$), wenn Urbilder offener Mengen offen sind.
 
 \paragraph{Beweis:}cf. H\"ohere Analysis, siehe z.B. Theorem 4.8 in Rudin, W. (1976) \textit{Principles of Mathematical Analysis}. 3rd edn., pp. 86-87. \qed
 
 \paragraph{4.8. Proposition:}Ist $f:\R\to\R$ stetig, dann ist $f$ $\borel\textendash\borel$-messbar. 
 
 \paragraph{Beweis:}folgt sofort aus Lemma 4.7 und Proposition 2.15. \qedsymbol
 
 \paragraph{Bemerkung:}Proposition 4.8 l\"asst sich auf beliebige metrische R\"aume ausweiten.
 
 \paragraph{4.9. Proposition:}Sei $(\Omega,\A)$ ein messbarer Raum und seien $f,g:(\Omega,\A)\to(\R,\borel)$ messbar. Sei $c\in\R$. Dann gilt:
 \begin{enumerate}[label=(\roman*)]
     \item $cf$ mit $(cf)(\omega):=cf(\omega)$ ist messbar.
     \item $f+g$ mit $(f+g)(\omega):=f(\omega)+g(\omega)$ ist messbar.
     \item $fg$ mit $(fg)(\omega):=f(\omega)g(\omega)$ ist messbar.
     \item F\"ur alle $\omega\in\Omega$ mit $g(\omega)\neq0$ ist $\frac{f}{g}$ mit $\left(\frac{f}{g}\right)(\omega):=\frac{f(\omega)}{g(\omega)}$ messbar.
  \end{enumerate}
  
  \paragraph{Beweis:}
  \begin{enumerate}[label=(\roman*)]
      \item Da die konstante Abbildung messbar ist, gilt (iii)$\implies$(i).
      % Messbarkeit im Erzeugendensystem?
      \item Fixiere $t\in\R$. Da der Urbildoperator mit Mengenoperationen kommutiert (und mit Proposition 3.14), gen\"ugt es zu zeigen, dass Mengen der Form
      $$A=\{\omega\in\Omega:f(\omega)+g(\omega)\in(-\infty,t)\}$$
      in $\A$ messbar sind. Dazu gen\"ugt es zu zeigen, dass
      $$A=\bigcup_{\substack{q,r\in\mathbb{Q}\\q+r<t}}\{\omega\in\Omega:f(\omega)<q\}\cap\{\omega\in\Omega:g(\omega)<r\}=:B$$
      Sei $\omega\in B$. Dann gibt es rationale Zahlen $q,r$ mit $q+r<t$, sodass $f(\omega)<q$ und $g(\omega)<r$ und somit $f(\omega)+g(\omega)<q+r<t$, also $\omega\in A$.\newline
      Sei $\omega\in A$. Dann gilt $f(\omega)+g(\omega)<t$ und damit $\delta:=t-(f(\omega)+g(\omega))>0$. Da $\mathbb{Q}$ dicht in $\R$ ist, gibt es $q,r\in\mathbb{Q}$, sodass $f(\omega)<q$, $g(\omega)<r$ und $q+r<f(\omega)+g(\omega)+\delta=t$. Damit folgt $\omega\in B$.
      \item $fg=\frac{1}{2}(f+g)^2-f^2-g^2$ wobei $t\mapsto \frac{1}{2}t^2$ und $t\mapsto -t^2$ stetig und damit messbar sind. Die Aussage folgt mit (ii).
      \item Fixiere $t\in\R$. Mit (iii) gen\"ugt es zu zeigen, dass $\{1/g<t,g\neq0\}$ messbar ist. Setze z.B. 
      \begin{align*}
          (1/g)(\omega):=
        \begin{cases}
          1/g(\omega)&\text{ falls }g(\omega)\neq0\\
          0&\text{ falls }g(\omega)=0
        \end{cases}
      \end{align*}
      Es gilt
      $$\{1/g<t\}=\{1/g<t, g<0\}\cup\{1/g<t,g>0\}\cup\{1/g<t,g=0\}$$
      Die Menge $\{1/g<t,g=0\}$ ist trivial ($\Omega$ oder $\emptyset$) messbar. \newline
      F\"ur $t>0$ gilt 
      $$\{1/g<t,g>0\}=\{g>1/t,g>0\},\ \{1/g<t, g<0\}=\{g<1/t,g<0\} $$
      F\"ur $t<0$ gilt
      $$\{1/g<t,g>0\}=\{g<1/t,g>0\},\ \{1/g<t, g<0\}=\{g>1/t,g<0\} $$
      F\"ur $t=0$ gilt
      $$\{1/g<t,g>0\}=\{g<0,g>0\}=\emptyset,\ \{1/g<t, g<0\}=\{g<0,g<0\}=\{g<0\}$$
      Die Aussage folgt mit Abgeschlossenheit von $\A$ bez\"uglich Vereinigungen, Durchschnitten und Komplementen. \qed
  \end{enumerate}
  
  % Ich h\"atte hier noch einen Extra Abschnitt eingef\"ugt, da die Abschnitts\"uberschrift sonst nicht passt
  \section*{Messbare Funktionen mit Werten in $\overline\R$}
  \addcontentsline{toc}{section}{Messbare Funktionen mit Werten in $\overline{\mathbb{R}}$}
  
  
  \paragraph{4.10. Definition:}Definiere die erweiterten reellen Zahlen als
  $$\overline\R:=\R\cup\{-\infty,\infty\}$$
  und analog zum reellen Fall die Mengenfamilie
  $$\mathcal{K}:=\{[-\infty,t]:t\in\overline\R\}$$
  sowie die Borel-$\sigma$-Algebra
  $$\cB(\overline\R):=\sigma(\mathcal{K})$$
  
  \paragraph{4.11. Definition (Rechenregeln in $\overline\R$):}
  \begin{align*}
      &a+\infty=\infty+a:=\infty\text{ f\"ur }a>-\infty\\
      &a-\infty=-\infty+a:=-\infty\text{ f\"ur }a<\infty\\
      &a\cdot\infty=\infty\cdot a:=\infty\text{ f\"ur }a>0\\
      &a\cdot\infty=\infty\cdot a:=-\infty\text{ f\"ur }a<0\\
      &a\cdot(-\infty)=(-\infty)\cdot a:=-\infty\text{ f\"ur }a>0\\
      &a\cdot(-\infty)=(-\infty)\cdot a:=\infty\text{ f\"ur }a<0\\
      &0\cdot\infty=0\cdot(-\infty):=0
  \end{align*}
  Beachte, dass $\infty-\infty$ nicht definiert ist.
  
  \paragraph{4.12. Lemma:} Es gilt $\cB(\overline\R)\krestr{\R}=\borel$. Au\ss{}erdem ist $\R\in\cB(\overline\R)$, sodass insbesondere $\borel\subseteq\cB(\overline\R)$.
 
 \paragraph{Beweis:}
 $$\cB(\overline\R)\krestr{\R}=\sigma(\mathcal{K})\krestr{\R}=\sigma(\mathcal{K}\krestr{\R})\overset{1.16}{=}\sigma(\mathcal{J})=\borel$$
 Au\ss{}erdem gilt $\{-\infty\}=\bigcap_{n\geq1}[-\infty,-n]\in\cB(\overline\R)$ und $\{\infty\}=\bigcap_{n\geq1}[n,\infty]\in\cB(\overline\R)$ und damit 
 $$\R=\overline\R\setminus(\{-\infty\}\cup\{\infty\})\in\cB(\overline\R)$$
 \qed
 
 \paragraph{4.13. Korollar:}Sei $(\Omega,\A)$ ein messbarer Raum und $f:(\Omega,\A)\to(\R,\borel)$ messbar. Dann ist $f$ auch $\A\textendash\cB(\overline\R)$-messbar.
 
 \paragraph{Beweis:}folgt sofort aus Lemma 4.12. \qed
 
 \paragraph{Bemerkung:}Damit gelten alle Aussagen in diesem Abschnitt auch f\"ur $\R$-wertige Funktionen.
 
 \paragraph{4.14. Korollar:}Sei $A\in\cB(\overline\R)$. Dann gilt
 $$A=B\cup\{\infty\} \text{ oder } A=B\cup\{-\infty\} \text{ oder } A=B \text{ oder } A=B\cup\{-\infty,\infty\}$$
 f\"ur ein $B\in\borel$.
 
 \paragraph{Beweis:}$B\in\borel=\cB(\overline\R)\krestr{\R}\implies B=A\cap\R$ mit $A\in\cB(\overline\R)$. Nun gilt $A=(A\setminus\R)\cup(A\cap\R)$, wobei $A\setminus\R=\{\infty\}$ oder $\{-\infty\}$ oder $\{-\infty,\infty\}$. \qed
 
 \paragraph{4.15. Korollar:}Sei $(\Omega,\A)$ ein messbarer Raum. Eine Abbildung $f:\Omega\to\overline\R$ ist $\A\textendash\cB(\overline\R)$-messbar, genau dann wenn:
 \begin{align*}
     \forall B\in\borel:f^{-1}(B)&\in\A\\
     f^{-1}(\{-\infty\})&\in\A\\
     f^{-1}(\{\infty\})&\in\A
 \end{align*}
 
 \paragraph{Beweis:}folgt sofort aus Kommutativit\"at des Urbildoperators mit Mengenoperationen und Korollar 4.14. \qed
 
 \paragraph{4.16. Definition:}Sei $A\subseteq\Omega$. Die Indikatorfunktion $\ind{A}:\Omega\to\{0,1\}$ auf $A$ (auch: charakteristische Funktion) ist definiert als
 \begin{align*}
     \ind{A}(\omega):=
     \begin{cases}
         1&\text{ falls }\omega\in A \\
         0&\text{ falls }\omega\notin A
     \end{cases}
 \end{align*}
 
 \paragraph{Bemerkung:}Jede Funktion $f:\Omega\to\{0,1\}$ mit $f(\Omega)=\{0,1\}$ ist eine Indikatorfunktion auf der Menge $A=\{\omega\in\Omega:f(\omega)=1\}$.
 
 \paragraph{4.17. Lemma:}Sei $(\Omega,\A)$ ein messbarer Raum. F\"ur $A\subseteq\Omega$ ist $\ind{A}$ genau dann $\A\textendash\borel$-messbar, wenn $A\in\A$.
 
 \paragraph{Beweis:}Es gilt
 \begin{align*}
     \ind{A}^{-1}(B)=
     \begin{cases}
         \emptyset&\text{ falls }0,1\notin B\\
         A&\text{ falls }0\notin B, 1\in B \\
         A^c&\text{ falls }0\in B, 1\notin B \\
         \Omega&\text{ falls }0,1\in B
     \end{cases}
 \end{align*}
 f\"ur alle $B\in\borel$. \qed
 
 \paragraph{4.18. Proposition:}Sei $(\Omega,\A)$ ein messbarer Raum und $f_n:\Omega\to\overline\R,n\geq1$ alle $\A\textendash\cB(\overline\R)$-messbar. Dann sind die Funktionen 
 $$\sup_{n\geq1}f_n,\ \inf_{n\geq1}f_n,\ \limsup_{n\to\infty}f_n,\ \liminf_{n\to\infty}f_n$$
 auch $\A\textendash\cB(\overline\R)$-messbar und insbesondere ist die Menge $\{\omega\in\Omega:\lim_{n\to\infty}f_n(\omega)\text{ existiert in }\overline\R\}$ messbar.
 
 \paragraph{Beweis:}Betrachte zun\"achst die Messbarkeitseigenschaften:
 \begin{gather*}
     \left\{\sup_{n\geq1}f_n<c\right\}=\bigcup_{n\geq1}\{f_n>c\}\in\A\implies\forall B\in\borel:\left\{\sup_{n\geq1}f_n\in B\right\}\in\A \\
     \left\{\sup_{n\geq1}f_n=\infty\right\}=\bigcap_{k\geq1}\{\sup_{n\geq1}f_n>k\}\in\A\\
     \left\{\sup_{n\geq1}f_n=-\infty\right\}=\left\{-\sup_{n\geq1}f_n=\infty\right\} \\ 
     \inf_{n\geq1}f_n=-\sup_{n\geq1}(-f_n) \\
     \limsup_{n\to\infty}f_n=\inf_{N\geq1}\left(\sup_{n\geq N}f_n\right) \\
     \liminf_{n\to\infty}f_n=-\limsup_{n\to\infty}(-f_n)
 \end{gather*}
 Die Messbarkeit folgt mit Korollar 4.15 und Proposition 4.9
 Weiters gilt
 $$\left\{\lim_{n\to\infty}f_n\in\overline\R\right\}=\left\{\limsup_{n\to\infty}f_n=\liminf_{n\to\infty}f_n\right\}\cup\left\{\liminf_{n\to\infty}f_n=\infty\right\}\cup\left\{\limsup_{n\to\infty}f_n=-\infty\right\}$$
 \qed
 
 \paragraph{4.19. Definition:}Sei $(\Omega,\A)$ ein messbarer Raum. Seien $A_i\in\A$ und $\alpha_i\in\R$ f\"ur $i=1,\hdots,n$. Dann nennt man $f:\Omega\to\R$ mit $f(\omega)=\sum_{i=1}^n \alpha_i\ind{A_i}(\omega)$ eine einfache Funktion. 
 
 \paragraph{4.20. Korollar:}Jede einfache Funktion ist $\A\textendash\borel$ und $\A\textendash\cB(\overline\R)$-messbar.
 
 \paragraph{Beweis:}Folgt sofort aus Lemma 4.17, Korollar 4.13 und Proposition 4.9. \qed
 
 \paragraph{4.21. Proposition:}Sei $(\Omega,\A)$ ein messbarer Raum.
\begin{enumerate}[label=(\roman*)]
    \item Eine Funktion $f:\Omega\to\R$ ist genau dann einfach, wenn $f$ messbar ist und endlich viele Werte annimmt (i.e. $|f(\Omega)|<\infty$). 
    \item Jede einfache Funktion $f:\Omega\to\R$ l\"asst sich schreiben als $f=\sum_{i=1}^n \alpha_i\ind{A_i}$ mit $A_i,i=1\hdots,n$ disjunkt.
\end{enumerate}

\paragraph{Beweis:}
\begin{enumerate}[label=(\roman*)]
    \item $\implies$: Die Messbarkeit folgt aus Korollar 4.20. Weiters gilt $|f(\Omega)|\leq2^n$.\newline
    $\impliedby$: Sei $f$ messbar und $f(\Omega)=\{\gamma_1,\hdots,\gamma_n\}$ mit $\gamma_i\neq\gamma_j$ f\"ur $i\neq j$. Sei $A_i:=\{f=\gamma_i\}$.
    \item folgt aus dem zweiten Teil im Beweis von (i). \qed
\end{enumerate}
 
 \paragraph{4.22. Satz:}Sei $(\Omega,\A)$ ein messbarer Raum und $f:(\Omega,\A)\to(\overline\R,\cB(\overline\R))$ nicht-negativ und messbar. Dann gibt es eine Folge einfacher Funktionen $f_n,n\geq1$, sodass
 $$\forall\omega\in\Omega:0\leq f_1(\omega)\leq f_2(\omega)\leq\hdots\leq \lim_{n\to\infty}f_n(\omega)=f(\omega)$$
 Wir schreiben oft kurz $0\leq f_n\uparrow f$. Ist $f$ zus\"atzlich beschr\"ankt, i.e. $\exists C\in\R\forall\omega\in\Omega:f(\omega)\leq C$, dann ist die Konvergenz gleichm\"a\ss{}ig, also
 $$\sup_{\omega\in\Omega}\left|f_n(\omega)-f(\omega)\right|\nto{}{n\to\infty}0$$
 
 \paragraph{Beweis:}Setze $\forall n\geq1$
 $$f_n(\omega):=\sum_{k=0}^{n\cdot2^n-1}\dfrac{k}{2^n}\cdot\ind{\left\{f\in\left[\frac{k}{2^n},\frac{k+1}{2^n}\right)\right\}}(\omega)+n\cdot\ind{\{f\geq n\}}$$
 $f_n\geq 0$ und $f_n$ einfach folgen aus der Konstruktion. Zeige also Monotonie und Konvergenz.
 \begin{itemize}
     \item Zeige $\forall\omega\in\Omega:\lim_{n\to\infty}f_n(\omega)=f(\omega)$\newline
        Falls $f(\omega)<\infty$ gibt es $n_0$, sodass $f(\omega)<n_0$ und damit 
        $$\forall n\geq n_0\exists k\in\{0,\hdots, n\cdot2^n-1\}:f(\omega)\in\left[\frac{k}{2^n},\frac{k+1}{2^n}\right)$$
        sodass $f_n(\omega)=\frac{k}{2^n}$. Insbesondere gilt damit
        $$0\leq f(\omega)-f_n(\omega)\leq\frac{1}{2^n}\nto{}{n\to\infty}0$$
        Falls $f(\omega)=\infty$ gilt $\forall n\geq1:f_n(\omega)=n\nto{}{n\to\infty}\infty=f(\omega)$.
        Falls $f$ beschr\"ankt ist, gibt es $n_0\geq1$, sodass $f(\omega)<n_o$ f\"ur alle $\omega\in\Omega$ und damit 
        $$\forall\omega\in\Omega:0\leq f(\omega)-f_(\omega)<\frac{1}{2^n}$$
        Da die obere Schranke unabh\"angig von $\omega$ ist, folgt die gleichm\"a\ss{}ige Konvergenz. 
        \item Zeige $\forall\omega\in\Omega:f_n(\omega)\leq f_{n+1}(\omega)$\newline
        Sei $f_n(\omega)=\frac{k}{2^n}$ f\"ur $k\in\{0,\hdots,n\cdot2^n-1\}$. Dann gilt
        $$f(\omega)\in\left[\frac{k}{2^n},\frac{k+1}{2^n}\right)=\left[\frac{2k}{2^{n+1}},\frac{2k+1}{2^{n+1}}\right)\cup\left[\frac{2k+1}{2^{n+1}},\frac{2k+2}{2^{n+1}}\right)$$
        Falls $f(\omega)\in \left[\frac{2k}{2^{n+1}},\frac{2k+1}{2^{n+1}}\right)$, dann ist $f_{n+1}(\omega)=\frac{2k}{2^{n+1}}=\frac{k}{2^n}=f_n(\omega)$. \newline
        Falls $f(\omega)\in\left[\frac{2k+1}{2^{n+1}},\frac{2k+2}{2^{n+1}}\right)$, dann ist $f_{n+1}(\omega)=\frac{2k+1}{2^{n+1}}>\frac{k}{2^n}=f_n(\omega)$.\newline\newline
        Sei $f_n(\omega)=n$. Dann gilt 
        $$f(\omega)\in [n,\infty]=[n,n+1)\cup[n+1,\infty]$$
        Falls $f(\omega)\in[n+1,\infty]$, dann ist $f_{n+1}(\omega)=n+1>n=f_n(\omega)$.\newline
        Falls $f(\omega)\in[n,n+1)$, dann ist 
        \begin{align*}
            f_{n+1}(\omega)&=\sum_{k=0}^{(n+1)\cdot2^{n+1}-1}\dfrac{k}{2^{n+1}}\cdot\ind{\left\{f\in\left[\frac{k}{2^{n+1}},\frac{k+1}{2^{n+1}}\right)\right\}}+(n+1)\cdot\ind{\{f\geq n+1\}}\\&=\dfrac{(n+1)\cdot2^{n+1}-1}{2^{n+1}}>n=f_n(\omega)\
        \end{align*}
        \qed
 \end{itemize}
 
 \chapter*{5. Lebesgue-Integral}
 \addcontentsline{toc}{chapter}{5. Lebesgue-Integral}
 
 Sei im folgenden Kapitel immer $(\Omega,\A,\mu)$ ein Ma\ss{}raum und $f:(\Omega,\A)\to(\overline\R,\cB(\overline\R))$ messbar.
 
  \section*{Konstruktion des Integrals}
  \addcontentsline{toc}{section}{Konstruktion des Integrals}
 
\paragraph{5.0. Definition (Informelle Definition des Integrals):}
\begin{enumerate}[label=(\roman*)]
    \item F\"ur $f=\sum_{i=1}^n \alpha_i\ind{A_i}$ einfach setzt man 
    $$\displaystyle\int f\ d\mu:=\sum_{i=1}^n\alpha_i\mu(A_i)$$
    \item F\"ur $f\geq0$ messbar w\"ahlt man einfache Funktionen mit $0\leq f_n\uparrow f$ und setzt
    $$\displaystyle\int f\ d\mu:=\lim_{n\to\infty}\int f_n\ d\mu$$
    \item F\"ur $f$ messbar definiert man $f^+:=f\cdot\ind{\{f\geq0\}}$ und $f^-:=f\cdot\ind{\{f< 0\}}$ und setzt
    $$\displaystyle\int f\ d\mu:=\int f^+\ d\mu-\int f^-\ d\mu$$
\end{enumerate}

\paragraph{5.1. Lemma:}F\"ur eine einfache Funktion $f=\sum_{i=1}^n\alpha_i\ind{A_i}$ ist $\displaystyle\int f\ d\mu$ unabh\"angig von der Darstellung von $f$.

% Achtung: Beweis hat einige Fehler!

\paragraph{Beweis:}Sei $m:=|f(\Omega)|$, mit $f(\Omega)=\{\gamma_1,\hdots,\gamma_m\}$ und sei $G_\ell:=\{\omega\in\Omega:f(\omega)=\gamma_\ell\}$ f\"ur $\ell=1,\hdots,m$. Dann sind $G_\ell,\gamma_\ell$ unabh\"angig von der Darstellung von $f$ f\"ur $\ell=1,\hdots,m$. Zeige nun $\sum_{i=1}^n \alpha_i\mu(A_i)=\sum_{\ell=1}^m\gamma_\ell\mu(G_\ell)$. \newline
F\"ur $j\in\{0,1\}^n$, sei 
$$B_j:=\left(\bigcap_{i=1}^nA_{j_i}\right)\cap\left(\bigcap_{i=1}^nA_{j_i}^c\right)=\left(\bigcap_{i:j_i=1}A_i\right)\cap\left(\bigcap_{i:j_i=0}A_i^c\right)$$
und $\beta_j:=\sum_{i=1}^n\alpha_i$. Dann sind die $B_j$ disjunkt und $A_i=\bigcup_{j:j_i=1}B_j$. Au\ss{}erdem gilt $\bigcup_{j\in\{0,1\}^n}B_j=\Omega$ und 
\begin{align*}
    A_i\cap B_j=
    \begin{cases}
        \emptyset&\text{ falls }j_i=0\\
        B_j&\text{ falls }j_i=1
    \end{cases}
\end{align*}
Es folgt 
$$\int f\ d\mu=\sum_{i=1}^n\alpha_i\mu(A_i)=\sum_{i=1}^n\sum_{\substack{j\in\{0,1\}^n\\ j_i=1}}\mu(B_j)=\sum_{j\in\{0,1\}^n}\mu(B_j)\sum_{\substack{i\in\{1,\hdots,n\}\\ j_i=1}}\alpha_i=\sum_{j\in\{0,1\}^n}\beta_j\mu(B_j)$$
\begin{itemize}
    \item Falls $B_j=\emptyset$, dann wird $\beta_j$ "weggeworfen", i.e. setze $\beta_j:=0$.
    \item Falls $\beta_j=\beta_{j'}$, dann werden $B_j$ und $B_{j'}$ vereinigt.
\end{itemize}
Schlie\ss{}lich erh\"alt man Werte $\{\beta_{j^{(1)}},\hdots,\beta_{j^{(m)}}\}=\{\gamma_1,\hdots,\gamma_m\}$ und f\"ur $\gamma_\ell=\beta_{j^{(k)}}$ ist 
$$G_\ell=\{f=\gamma_\ell\}=\{f=\beta_{j^{(k)}}\}$$
 Es folgt
 $$\int f\ d\mu=\sum_{i=1}^n\alpha_i\mu(A_i)=\sum_{k=1}^m\beta_{j^{(k)}}\mu(B_k)=\sum_{\ell=1}^m\gamma_\ell\mu(G_\ell)$$
 \qed
 
 \paragraph{5.2. Definition:}Sei $f=\sum_{i=1}^n\alpha_i\ind{A_i}$ mit $\alpha_i\geq0$ und $A_i\in\A$ f\"ur $i=1,\hdots,n$ einfach. Das Lebesgue-Integral von $f$ bez\"uglich $\mu$ ist definiert als
 $$\int f\ d\mu:=\sum_{i=1}^n\alpha_i\mu(A_i)\in[0,\infty]$$
 Durch Lemma 5.1 ist der Ausdruck wohldefiniert.
 
 \paragraph{5.3. Lemma:}Seien $f,g$ beide nicht-negative, einfache Funktionen sodas $\forall\omega\in\Omega:f(\omega)\leq g(\omega)$. Dann gilt (Monotonie des Integrals f\"ur einfache Funktionen)
 $$\int f\ d\mu\leq\int g\ d\mu$$
 
 \paragraph{Beweis:}Sei $f=\sum_{i=1}^n\alpha_i\ind{A_i}$ und $g=\sum_{j=1}^m\beta_j\ind{B_j}$ in kanonischer Darstellung ($A_i$ disjunkt f\"ur $i=1,\hdots,n$ und $B_j$ disjunkt f\"ur $j=1,\hdots,m$) und sei o.B.d.A. $\Omega=\bigcup_{i=1}^nA_i=\bigcup_{j=1}^mB_j$. Dann gilt 
 \begin{align*}
     f&=\sum_{i=1}^n\sum_{j=1}^m\alpha_i\ind{A_i\cap B_j}\\
     g&=\sum_{i=1}^n\sum_{j=1}^m\beta_j\ind{A_i\cap B_j}
 \end{align*}
 Laut Annahme gilt f\"ur $\omega\in A_i\cap B_j\neq\emptyset$ daher $\alpha_i\leq\beta_j$ und damit
 $$\int f\ d\mu=\sum_{i=1}^n\sum_{j=1}^m\alpha_i\mu(A_i\cap B_j)\leq\sum_{i=1}^n\sum_{j=1}^m\beta_j\mu(A_i\cap B_j)=\int g\ d\mu$$
 \qed
 
 \paragraph{5.4. Lemma:}Seien $f_n,n\geq1$ und $g$ nicht negative, einfache Funktionen sodass
 $$0\leq f_1\leq\hdots\leq\lim_{n\to\infty}f_n\text{ und }g\leq\lim_{n\to\infty}f_n$$
 Dann gilt $\displaystyle\lim_{n\to\infty}\int f_n\ d\mu\in\overline\R$ und $\displaystyle\int g \ d\mu\leq\lim_{n\to\infty}\int f_n\ d\mu$.
 
 \paragraph{Beweis:}Aus Lemma 5.3 folgt 
 $$0\leq\int f_1\ d\mu\leq\int f_2\ d\mu\leq\hdots$$
 Damit ist $\displaystyle\int f_n\ d\mu,n\geq1$ eine monoton nicht-fallende Folge in $\overline\R$ und damit $\displaystyle\lim_{n\to\infty}\int f_n\ d\mu\in\overline\R$. Es verbleibt zu zeigen, dass $\displaystyle\int g\ d\mu\leq\lim_{n\to\infty}\int f_n\ d\mu$. Sei $\alpha>1$ und definiere $A_n:=\{g\leq\alpha \cdot f_n\},n\geq1$. Dann gilt
 $$A_1\subseteq A_2\subseteq\hdots\subseteq\bigcup_{n\geq1}A_n=\Omega$$
 da $f_1\leq f_2\leq\hdots$. Die letze Gleichheit folgt aus folgendem Argument: \newline\newline
 Angenommen $\exists\omega\in\Omega\setminus\left(\bigcup_{n\geq1}A_n\right)$. Dann gilt $\exists\omega\in\Omega\forall n\geq1:g(\omega)\geq\alpha \cdot f_n(\omega)$ und damit f\"ur dieses $\omega\in\Omega$, dass $g(\omega)>\displaystyle\lim_{n\to\infty}f_n(\omega)$, ein Widerspruch zur Annahme.\newline\newline
 Nun ist $g\cdot\ind{A_n}$ f\"ur $n\geq1$ einfach und messbar und $g\cdot\ind{A_n}\leq\alpha\cdot f_n$. Mit Lemma 5.3 folgt f\"ur $\alpha\searrow1$ f\"ur alle $n\geq1$
 $$\int g\cdot\ind{A_n}\ d\mu\leq\int f_n\ d\mu\leq\lim_{n\to\infty}\int f_n\ d\mu$$
 Sei $g=\sum_{i=1}^m\gamma_i\ind{G_i}$ und damit $g\cdot\ind{A_n}=\sum_{i=1}^m\gamma_i\ind{G_i\cap A_n}$. Weiters gilt f\"ur $i=1,\hdots,m$
 $$G_i\cap A_1\subseteq G_i\cap A_2\subseteq\hdots\subseteq\bigcup_{n\geq1}(G_i\cap A_n)=G_i$$
 und mit der Stetigkeit von unten (Satz 1.9) folgt $\displaystyle\lim_{n\to\infty}\mu(G_i\cap A_n)=\mu(G_i)$ und damit
 \begin{align*}
     \lim_{n\to\infty}\int g\cdot\ind{A_n}\ d\mu&=\lim_{n\to\infty}\sum_{i=1}^m\gamma_i\cdot\mu(G_i\cap A_n)\\&=\sum_{i=1}^m\gamma_i\cdot\left(\lim_{n\to\infty}\mu(G_i\cap A_n)\right)\\&=\sum_{i=1}^m\gamma_i\cdot\mu(G_i)=\int g\ d\mu
 \end{align*}
 Ebenfalls ist $\displaystyle\int g\cdot\ind{A_n}\ d\mu\leq\lim_{n\to\infty}\int f_n\ d\mu$ f\"ur alle $n\geq1$ und damit
 $$\int g\ d\mu=\lim_{n\to\infty}\int g\cdot\ind{A_n}\ d\mu\leq\lim_{n\to\infty}\lim_{n\to\infty}\int f_n\ d\mu=\lim_{n\to\infty}\int f_n\ d\mu$$
 \qed
 
 \paragraph{5.5. Korollar:}Betrachte eine nicht-negative, messbare Funktion $f$, sowie einfache Funktionen $f_n,n\geq1$ und $g_m,m\geq1$, sodass $0\leq f_n\uparrow f\text{ und }0\leq g_m\uparrow f$. Dann gilt 
 $$\displaystyle\lim_{n\to\infty}\int f_n\ d\mu=\lim_{m\to\infty}\int g_m\ d\mu$$
 Insbesondere ist damit die Wahl der approxmierenden Funktionen in der Definition des Lebesgue-Integrals sp\"ater egal.
 
 \paragraph{Beweis:}Es gilt $\forall m\geq1:g_m\leq\displaystyle\lim_{n\to\infty}f_n$ und mit Lemma 5.4 $\displaystyle\forall m\geq1:\int g_m\ d\mu\leq\lim_{n\to\infty}\int f_n\ d\mu$. Es folgt 
 $$\lim_{m\to\infty}\int g_m\ d\mu\leq\lim_{m\to\infty}\lim_{n\to\infty}\int f_n\ d\mu=\lim_{n\to\infty}\int f_n\ d\mu$$
 Ebenfalls gilt $\forall n\geq1:f_n\leq\displaystyle\lim_{m\to\infty}g_m$ und mit Lemma 5.4 $\forall n\geq1:\displaystyle\int f_n\ d\mu\leq\lim_{m\to\infty}\int g_m\ d\mu$. Wie oben folgt
 $$\lim_{n\to\infty}\int f_n\ d\mu\leq\lim_{m\to\infty}\int g_m\ d\mu$$
 \qed
 
 \paragraph{5.6. Definition:}Sei $f:(\Omega,\A)\to(\overline\R,\cB(\overline\R))$ nicht-negativ und messbar und $f_n,n\geq1$ eine belibige Folge einfacher Funktionen, sodass $0\leq f_n\uparrow f$. Dann ist das Lebesgue-Integral von $f$ bez\"uglich $\mu$ definiert als
 $$\int f\ d\mu:=\lim_{n\to\infty}\int f_n\ d\mu$$
 Dieser Grenzwert ist wegen der Monotonie (Lemma 5.3) und der Unabh\"angigkeit von den approximierenden Funktionen (Korollar 5.5) wohldefiniert. 
 
 \paragraph{5.7. Definition:}F\"ur eine Funktion $f:\Omega\to\overline\R$ werden der Positivteil $f^+$ und der Negativteil $f^-$ definiert als
 $$f^+:=\max(f,0)\text{ und }f^-:=-\min(f,0)$$
 Es gilt trivial $f=f^+-f^-$. Ist $f$ messbar, dann sind $f^+,f^-$ auch beide messbar, da $f^+=f\cdot\ind{\{f\geq0\}}$ und $f^-=f\cdot\ind{\{f<0\}}$.
 
 \paragraph{5.8. Definition:}Sei $f:(\Omega,\A)\to(\overline\R,\cB(\overline\R))$ messbar.
 \begin{enumerate}[label=(\roman*)]
     \item Falls $\displaystyle\int f^+\ d\mu<\infty$ \underline{und} $\displaystyle\int f^-\ d\mu<\infty$, dann ist $f$ $\mu$-integrierbar.
     \item Falls $\displaystyle\int f^+\ d\mu<\infty$ \underline{oder} $\displaystyle\int f^-\ d\mu<\infty$, dann ist $f$ $\mu$-quasi-integrierbar.
     \item Falls $\displaystyle\int f^+\ d\mu=\displaystyle\int f^-\ d\mu=\infty$, dann ist $f$ nicht $\mu$-integrierbar.
     \item Falls $f$ quasi-integrierbar bez\"uglich $\mu$ ist, dann ist das Lebesgue-Integral von $f$ definiert als 
     $$\int f\ d\mu:=\int f^+\ d\mu-\int f^-\ d\mu$$
 \end{enumerate}
 
 \paragraph{5.9. Definition:}Definiere die folgenden beiden Funktionenr\"aume
 \begin{align*}
     \mathcal{L}^1(\Omega,\A,\mu)&:=\left\{f:(\Omega,\A)\to(\overline\R,\cB(\overline\R)):f\text{ integrierbar bez\"uglich } \mu \right\} \\
     \mathcal{L}(\Omega,\A,\mu)&:=\left\{f:(\Omega,\A)\to(\overline\R,\cB(\overline\R)):f\text{ quasi-integrierbar bez\"uglich } \mu \right\}
 \end{align*}
 Falls der Ma\ss{}raum $(\Omega,\A,\mu)$ bzw. der messbare Raum $(\Omega,\A)$ beliebig oder aus dem Kontext bekannt sind, schreibt man oft kurz $\mathcal{L}^1$ bzw. $\mathcal{L}^1(\mu)$.
 
 \paragraph{Bemerkung:}$\mathcal{L}^1(\Omega,\A,\mu)$ bildet mit skalarweiser Addition und Multiplikation einen Vektorraum und $\Vert f\Vert_1:=\displaystyle\int |f|\ d\mu$ bildet eine Halbnorm auf $\mathcal{L}^1(\Omega,\A,\mu)$. Beachte den Unterschied zwischen $\mathcal{L}^1(\Omega,\A,\mu)$ und dem Quotientenraum $L^1(\Omega,\A,\mu)$ (bzgl. \"Aquivalenz fast \"uberall, siehe Definition 5.12). Details siehe z.B. Teschl, G. (2024) \textit{Topics in Real Analysis}., pp. 73-74. 
 
 \paragraph{5.10. Definition:}F\"ur $f\in\mathcal{L}(\Omega,\A,\mu)$ und $A\in\A$, dann wird das Integral von $f$ bez\"uglich $\mu$ \"uber $A$ definiert als
 $$\int_A f\ d\mu:=\int f\cdot\ind{A}\ d\mu$$
 
 \paragraph{5.11. Definition:}Ist $\pspace$ ein Wahrscheinlichkeitsraum und ist $X\in \mathcal{L}(\Pp)$, dann nennt man 
 $$\E X:=\int X\ d\Pp$$
 den Erwartungswert von $X$ unter $\Pp$. Falls zus\"atzlich $X\in\mathcal{L}^1(\Pp)$, dann nennt man
 $$\operatorname{Var}(X):=\int (X-\E X)^2\ d\Pp$$
 die Varianz von $X$ (unter $\Pp$).
 
 \section*{Eigenschaften des Integrals}
  \addcontentsline{toc}{section}{Eigenschaften des Integrals}
  
  \paragraph{5.12. Definition:}
  \begin{itemize}
      \item Eine Menge $A\in\A$ mit $\mu(A)=0$ nennt man $\mu$-Nullmenge.
      \item Ist $f:(\Omega,\A)\to(\overline\R,\cB(\overline\R))$ messbar und $B\in\cB(\overline\R)$ mit $\mu\left(\{f\in B\}^c\right)=\mu(\{f\in B^c\})=0$, dann sagt man, dass das Ereignis $\{f\in B\}$ $\mu$-fast-\"uberall eintritt. Kurz: $f\in B$ f.\"u. (englisch \textit{a.e., almost everywhere}).
      \item Sei $\Pp$ ein Wahrscheinlichkeitsma\ss{} und $X:(\Omega,\A)\to(\overline\R,\cB(\overline\R))$ eine Zufallsvariable. Ist $B\in\cB(\overline\R)$, sodass $X\in B$ $\Pp$-fast-\"uberall, dann sagt man auch dass $X\in B$ fast sicher. Kurz: $X\in B$ f.s. (englisch \textit{a.s., almost surely}).
  \end{itemize}
  
  \paragraph{Bemerkung:}Die Vereinigung bis zu abz\"ahlbar vieler Nullmengen ist wegen der $\sigma$-Subadditivit\"at (Satz 1.7 (iii)) wieder eine Nullmenge. 
  
  \paragraph{5.13. Lemma:}Sei $f\geq0$ messbar. Dann gilt
  $$\int f \ d\mu=0\iff f = 0\text{ a.e.}$$
  
  \paragraph{Beweis:}
  \begin{enumerate}[label=\Roman*.]
      \item $f$ einfach\newline
      Sei $\displaystyle\int f\ d\mu=\sum_{i=1}^n\alpha_i\cdot\mu(A_i)=0$ in kanonischer Darstellung. Da $f\geq0$ gilt f\"ur $i=1,\hdots,n$, dass $\alpha_i=0$ oder $\mu(A_i)=0$. \goodbreak
      Es gilt mit der $\sigma$-Additivit\"at
      \begin{align*}
          \mu\left(\{f>0\}
          \right)=\mu\left(\bigcup_{\substack{i=1\\\alpha_i>0}}^n A_i\right)=\sum_{\substack{i=1\\ \alpha_i>0}}^n\mu(A_i)=0
      \end{align*}
      Sei $\mu(\{f>0\})=0$. Dann gilt wie im ersten Fall 
      $$\sum_{\substack{i=1\\ \alpha_i>0}}^n\mu(A_i)=0\implies\int f\ d\mu=\sum_{i=1}^n\alpha_i\cdot\mu(A_i)=0$$
      \item allgemeiner Fall\newline
      Sei $\displaystyle\int f\ d\mu=\lim_{n\to\infty}\int f_n\ d\mu=0$. Da $0\leq f_n\uparrow f$, gilt $\int f_n\ d\mu=0$ f\"ur alle $n\geq1$ und mit I. folgt $f_n=0$ a.e. Mit der Stetigkeit von unten folgt weiters
      $$\mu(\{f>0\})=\lim_{n\to\infty}\mu(\{f_n>0\})=0$$
      Sei nun $\mu(\{f>0\})=0$. Dann folgt wegen $f_n\leq f$ f\"ur alle $n\geq1$, dass $\mu(\{f_n>0\})=0$. Mit I. folgt $\displaystyle\int f_n\ d\mu=0$ und schlie\ss{}lich $\displaystyle\int f\ d\mu=\lim_{n\to\infty}\int f_n\ d\mu=0$. \qed
  \end{enumerate}
  
  \paragraph{5.14. Proposition:}Sei $f\in\mathcal{L}(\mu)$ und $g$ messbar, sodass $f=g$ a.e. Dann gilt $g\in\mathcal{L}(\mu)$ und
  $$\int g\ d\mu=\int f\ d\mu$$
  
  \paragraph{Beweis:}
  Seien $f_n,g_n,n\geq1$ wie im Beweis von Satz 3.22, sodass $0\leq f_n\uparrow f$ und $0\leq g_n\uparrow g$.
  \begin{enumerate}[label=\Roman*.]
      \item $f,g$ nicht-negativ\newline
      Es gilt 
      \begin{align*}
          \forall B\in\cB(\overline\R):\mu(\{f\in A\})&=\mu(\{f\in A,f=g\}\cup\{f\in A,f\neq g\})\\
          &=\mu(\{f\in A,f=g\})+\mu(\{f\in A,f\neq g\})\\
          &=\mu(\{g\in A,f=g\})+0\\
          &=\mu(\{g\in A,f=g\})+\mu(\{g\in A,f\neq g\})\\
          &=\mu(\{g\in A\})
      \end{align*}
      Damit gilt per Konstruktion von $f_n,g_n$, dass $f_n=g_n$ a.e. f\"ur alle $n\geq1$ und damit 
      $$\forall n\geq1:\displaystyle\int f_n\ d\mu=\int g_n\ d\mu$$
      Es folgt 
      $$\int f\ d\mu=\lim_{n\to\infty}\int f_n\ d\mu=\lim_{n\to\infty}\int g_n\ d\mu=\int g\ d\mu$$
      \item $f,g$ allgemein\newline
      Sei $f=f^+-f^-$ und $g=g^+-g^-$. Es gilt $f^+,f^-,g^+,g^-\geq0$ und daher $f^+,f^-,g^+,g^-\in\mathcal{L}(\mu)$. Nun gilt $\{f^+\neq g^+\}=\{f\neq g,f\geq0,g\geq0\}\subseteq\{f\neq g\}$ und damit $f^+=g^+$ a.e. und $f^-=g^-$ a.e. Mit I. folgt
      $$\int f^+\ d\mu=\int g^+\ d\mu\text{ und }\int f^-\ d\mu=\int g^-\ d\mu$$
      Die gew\"unschte Aussage folgt mit der Konstruktion des Integrals. \qed
  \end{enumerate}
  
  \paragraph{5.15. Proposition:}Ist $f\in\mathcal{L}^1(\mu)$, dann gilt $|f|<\infty$ a.e. Insbesondere gibt es eine reelwertige, messbare Funktion $g:(\Omega,\A)\to(\R,\borel)$, sodass $\displaystyle\int g\ d\mu=\int f\ d\mu$.
  
  \paragraph{Beweis:}
  \begin{enumerate}[label=\Roman*.]
      \item $f$ nicht-negativ\newline
      Seien $f_n,n\geq1$ wie im Beweis von Satz 3.22. Dann gilt
      $$\infty>\int f\ d\mu\geq\int f_n\ d\mu\geq n\cdot\mu(\{f\geq n\})$$
      f\"ur alle $n\geq1$. F\"ur $n\to\infty$ muss also $\mu(\{f\geq n\})\to0$ gelten. Nun ist
      $$\{f\geq1\}\supseteq\{f\geq2\}\supseteq\hdots\supseteq\bigcap_{n\geq1}\{f\geq n\}=\{f=\infty\}$$
      und $\mu(\{f\geq1\})\cdot 1<\infty$ (s.o.). Mit der Stetigkeit von oben folgt also 
      $$\mu(\{f=\infty\})=\lim_{n\to\infty}\mu(\{f\geq n\})=0$$
      und die Aussage folgt mit $f=|f|$, da $f\geq 0$.
      \item $f$ messbar\newline
      Sei $f=f^+-f^-$. Dann gilt $f^+,f^-\geq0$ und $f^+,f^-\in\mathcal{L}^1(\mu)$. Mit I. folgt $|f^+|,|f^-|<\infty$ a.e. Die Aussage folgt mit $|f|=|f^+-f^-|\leq|f^+|+|f^-|$. Definiere nun $g:=f\cdot\ind{\{|f|<\infty\}}$. Dann ist $f=g$ a.e. mit $f\in\mathcal{L}^1(\mu)\subseteq\mathcal{L}(\mu)$ und $g$ messbar. Die Gleichheit der Integral folgt aus Proposition 5.14. \qed
  \end{enumerate}
  
  \paragraph{5.16. Satz (Linearit\"at und Monotonie des Lebesgue-Integrals):}
  \begin{enumerate}[label=(\roman*)]
      \item Seien $f,g\in\mathcal{L}(\mu)$, sodass $\displaystyle\int f\ d\mu+\int g\ d\mu$ wohldefiniert ist (i.e. nicht $\infty-\infty$, z.B. wenn $f,g\in\mathcal{L}^1(\mu)$). Dann ist auch $(f+g)$ fast \"uberall wohldefiniert und 
      $$\int(f+g)\ d\mu=\int f\ d\mu+\int g\ d\mu$$
      \item Sei $f\in\mathcal{L}(\mu)$ und $\alpha\in\R$. Dann ist $(\alpha\cdot f)$ wohldefiniert, $(\alpha\cdot f)\in\mathcal{L}(\mu)$ und 
      $$\int (\alpha\cdot f)\ d\mu=\alpha\cdot\int f\ d\mu$$
      \item Seien $f,g\in\mathcal{L}(\mu)$, sodass $f\leq g$ a.e. Dann gilt
      $$\int f\ d\mu\leq\int g\ d\mu$$
  \end{enumerate}
 
 \paragraph{Beweis:}
 \begin{enumerate}[label=(\roman*)]
     \item
     \begin{enumerate}[label=\Roman*.]
        \item $f,g\geq 0$ einfach\newline
        Seien $f=\sum_{i=1}^n\alpha_i\cdot\ind{A_i}$ und $g=\sum_{j=1}^m\beta_i\cdot\ind{B_i}$ in kanonischer Darstellung. Dann ist
        $$(f+g)=\sum_{i=1}^n\sum_{j=1}^m(\alpha_i+\beta_j)\cdot\ind{A_i\cap B_j}$$
        und 
        $$\int (f+g)\ d\mu=\sum_{i=1}^n\sum_{j=1}^m(\alpha_i+\beta_j)\cdot\mu(A_i\cap B_j)$$
        Nun gilt (Einschluss-Ausschluss) $\mu(A_i\cap B_j)=\mu(A_i)+\mu(B_j)-\mu(A_i\cup B_j)$. Au\ss{}erdem sind die $A_i,i=1,\hdots,n$ eine Partition von $\Omega$ ($B_j$ genauso) und damit  
        $$\sum_{i=1}^n\mu(A_i\cap B_j)=\mu(B_j)\text{ und } \sum_{j=1}^m\mu(A_i\cap B_j)=\mu(A_i)$$
        Es folgt 
        $$\int(f+g)\ d\mu=\sum_{i=1}^n\sum_{j=1}^m(\alpha_i+\beta_j)\cdot\mu(A_i\cap B_j)=\sum_{i=1}^n\alpha_i\cdot\mu(A_i)+\sum_{j=1}^m\beta_j\cdot\mu(B_j)=\int f\ d \mu+\int g\ d\mu$$
        \item $f,g$ nicht-negativ\newline
        W\"ahle $f_n,g_n,n\geq1$ einfach mit $0\leq f_n\uparrow f$ und $0\leq g_n\uparrow g$. Dann gilt $0\leq f_n+g_n\uparrow f+g$, wobei $f+g$ nicht-negativ und messbar ist (Proposition 3.18). Wegen $f+g\geq0$ gilt auch $f+g\in\mathcal{L}(\mu)$. Mit I. folgt
        $$\int (f+g)\ d\mu=\lim_{n\to\infty}\int (f_n+g_n)\ d\mu=\lim_{n\to\infty}\int f_n\ d\mu+\lim_{n\to\infty}\int g_n\ d\mu=\int f\ d\mu+\int g\ d\mu$$
        \item $f,g$ messbar\newline
        Beachte, dass $\displaystyle\int f\ d\mu$ und $\displaystyle\int g\ d\mu$ wohldefiniert sind, da $f,g\in\mathcal{L}(\mu)$ und schreibe
        \begin{align*}
           \int f\ d\mu+\int g\ d\mu&=\int f^+\ d\mu-\int f^-\ d\mu+\int g^+\ d\mu -\int g^-\ d\mu\\
           &=\left(\int f^+\ d\mu+\int g^+\ d\mu\right)-\left(\int f^-\ d\mu+\int g^-\ d\mu\right)\in\overline\R
        \end{align*}
        Damit ist $\displaystyle\left(\int f^+\ d\mu+\int g^+\ d\mu\right)=\left(\int f^-\ d\mu+\int g^-\ d\mu\right)=\infty$ nicht m\"oglich. Sei also o.B.d.A. $\displaystyle\left(\int f^+\ d\mu+\int g^+\ d\mu\right)<\infty$ (der andere Fall folgt \"ahnlich). Es sind $f^+,g^+\geq0$ und mit II. gilt 
        $$\int f^+\ d\mu+\int g^+\ d\mu=\int(f^++g^+)\ d\mu\in[0,\infty)$$
        Mit Proposition 5.15 folgt $|f^++g^+|=f^++g^+<\infty$ a.e. Definiere also f\"ur $\omega\in\Omega$
        \begin{align*}
            (f+g)(\omega):=
            \begin{cases}
                f(\omega)+g(\omega)&\text{ falls }f^+(\omega)+g^+(\omega)<\infty\\
                0&\text{ sonst}
            \end{cases}
        \end{align*}
        Dann ist $(f+g)$ wohldefiniert, messbar und $(f+g)=f+g$ a.e. Au\ss{}erdem gilt per Konstruktion $(f+g)^+\leq(f^++g^+)$ und mit (iii) folgt
        $$\int (f+g)^+\ d\mu\leq\int (f^++g^+)\ d\mu=\int f^+\ d\mu+\int g^+\ d\mu<\infty$$
        Damit gilt $(f+g)\in\mathcal{L}(\mu)$. Au\ss{}erdem gilt
        $$(f+g)^+-(f+g)^-=(f+g)\overset{a.e.}{=}f+g=f^+-f^-+g^+-g^-$$
        und damit 
        $$(f+g)^++f^-+g^-\overset{a.e.}{=}(f+g)^-+f^++g^+$$
        Mit Proposition 5.14 folgt 
        $$\int(f+g)^++f^-+g^-\ d\mu=\int (f+g)^-+f^++g^+\ d\mu$$
        und mit II. schlie\ss{}lich
        $$\int(f+g)^+\ d\mu+\int f^-\ d\mu+\int g^-\ d\mu=\int (f+g)^-\ d\mu+\int f^+  d\mu+\int g^+\ d\mu$$
        und schlie\ss{}lich
        $$\int (f+g)\ d\mu=\int f\ d\mu+\int g\ d\mu$$
     \end{enumerate}
     \item 
     \begin{enumerate}[label=\Roman*.]
        \item $f\geq0$ einfach\newline
        Hier gilt $\alpha\cdot f =\alpha\cdot \sum_{i=1}^n\beta_i\cdot\ind{B_i}=\sum_{i=1}^n(\alpha\cdot\beta_i)\cdot\ind{B_i}$ und damit
        $$\int (\alpha\cdot f)\ d\mu=\sum_{i=1}^n(\alpha\cdot\beta_i)\cdot\mu(B_i)=\alpha\cdot\sum_{i=1}^n\beta_i\cdot\ind{B_i}=\alpha\cdot\int f\ d\mu$$
        \item $f\geq0$ messbar, $\alpha\geq 0$\newline
        W\"ahle $f_n$ einfach, sodass $0\leq f_n\uparrow f$. Dann gilt $0\leq(\alpha\cdot f_n)\uparrow (\alpha\cdot f)$ und 
        \begin{align*}
            \int (\alpha\cdot f)\ d\mu&=\lim_{n\to\infty}\int (\alpha\cdot f_n)\ d\mu\\&\overset{\text{I.}}{=}\lim_{n\to\infty}\alpha\cdot\int f_n\ d\mu\\&=\alpha\cdot\lim_{n\to\infty}\int f_n\ d\mu\\&=\alpha\cdot\int f\ d\mu
        \end{align*}
        \item $f$ messbar, $\alpha\in\R$\newline
        Sei zuerst $\alpha\geq0$. Dann ist $(\alpha\cdot f)$ wohldefiniert, da $\alpha\neq\pm\infty$ und 
        $$(\alpha\cdot f)=(\alpha\cdot f^+)-(\alpha \cdot f^-)=\alpha\cdot(f^+-f^-)$$
        Mit II. gilt 
        $$\int (\alpha\cdot f^+)\ d\mu=\alpha\cdot\int f^+\ d\mu\ \text{ und  }\ \int (\alpha\cdot f^-)\ d\mu=\alpha\cdot\int f^-\ d\mu$$
        Da $f\in\mathcal{L}(\mu)$, muss eines der beiden Integrale endlich sein und $(\alpha\cdot f)\in\mathcal{L}(\mu)$. Damit ist
        $$\int (\alpha\cdot f^+)\ d\mu-\int (\alpha\cdot f^-)\ d\mu$$
        wohldefiniert und mit (i) folgt
        \begin{align*}
            \int(\alpha\cdot f)\ d\mu&=\int(\alpha\cdot f^+)-(\alpha\cdot f^-)\ d\mu\\
            &=\int (\alpha\cdot f^+)\ d\mu-\int(\alpha\cdot f^-)\ d\mu\\
            &=\alpha\cdot \int f^+\ d\mu-\alpha\cdot\int f^-\ d\mu\\
            &=\alpha\cdot\left(\int f^+\ d\mu-\int f^-\ d\mu\right)\\
            &=\alpha\cdot\int(f^+-f^-)\ d\mu\\
            &=\alpha\cdot\int f\ d\mu
        \end{align*}
        Sei nun $\alpha <0$. Dann ist $(\alpha\cdot f)$ wohldefiniert und messbar und 
        $$(\alpha \cdot f)=(-\alpha)(-f)$$
        wobei $(-\alpha)>0$ und $(-f)\in\mathcal{L}(\mu)$. Dammit folgt (siehe oben)
        \begin{align*}
            \int (\alpha\cdot f)\ d\mu&=\int(-\alpha)(-f)\ d\mu\\
            &=(-\alpha)\int (-f)\ d\mu\\
            &=(-\alpha)\int (f^--f^+)\ d\mu\\
            &=(-1)\cdot\alpha\cdot\left(\int f^-\ d\mu-\int f^+\ d\mu\right)\\
            &=(-1)^2\cdot\alpha\cdot\left(\int f^+\ d\mu-\int f^-\ d\mu\right)\\
            &=\alpha\cdot\int f\ d\mu
        \end{align*}
     \end{enumerate}
     \item Betrachte hier $\max(f,g):=g\cdot\ind{\{f\leq g\}}+f\cdot\ind{\{f>g\}}$ messbar mit $f\leq\max(f,g)$ und $g=\max(f,g)$ a.e. Damit folgt $\displaystyle\int \max(f,g)\ d\mu=\int g\ d\mu$.
     \begin{enumerate}[label=\Roman*.]
        \item $f$ nicht-negativ\newline
        Hier ist $0\leq f\leq \max(f,g)$. W\"ahle nun einfache Funktionen $f_n,m_n,n\geq1$, sodass $0\leq f_n\uparrow f$ und $0\leq m_n\uparrow\max(f,g)$. Dann gilt f\"ur alle $n\geq1$
        $$0\leq f_n\leq \max(f,g)=\lim_{n\to\infty}m_n$$
        und mit Lemma 5.4 folgt 
        $$\forall n\geq 1:\int f_n\ d\mu\leq\lim_{n\to\infty}\int m_n\ d\mu=\int \max(f,g)\ d\mu=\int g\ d\mu$$
        und damit
        $$\int f\ d\mu=\lim_{n\to\infty}\int f_n\ d\mu\leq\int g\ d\mu$$
        \item $f$ messbar\newline
        Trvial, wenn $\displaystyle\int g^+\ d\mu=\infty$ oder $\displaystyle\int f^-\ d\mu=-\infty$. Seien also $f^-,g^+\in\mathcal{L}^1(\mu)$. Wegen $f\leq\max(f,g)$ folgt $f^+\leq[\max(f,g)]^+$ und mit I. gilt 
        $$\int f^+\ d\mu\leq \int [\max(f,g)]^+\ d\mu<\infty$$
        Wegen $f\leq\max(f,g)$ gilt auch $f^-\geq[\max(f,g)]^-$ und mit I. folgt
        $$\infty>\int f^-\ d\mu\geq\int [\max(f,g)]^-\ d\mu$$
        Damit folgt 
        $$\int f\ d\mu\leq\int [\max(f,g)]^+\ d\mu-\int [\max(f,g)]^-\ d\mu=\int \max(f,g)\ d\mu=\int g\ d\mu$$
        \qed
     \end{enumerate}
 \end{enumerate}
 
 \paragraph{5.17. Proposition:}Sei $f$ messbar. Dann sind folgende Aussagen \"aquivalent:
 \begin{enumerate}[label=(\roman*)]
     \item $f\in\mathcal{L}^1(\mu)$
     \item $|f|\in\mathcal{L}^1(\mu)$
     \item $\exists g\in\mathcal{L}^1(\mu):|f|\leq g$ a.e.
     \item $\exists h_1,h_2\in\mathcal{L}^1(\mu):h_1,h_2\geq0,f=h_1-h_2$
 \end{enumerate}
 
 \paragraph{Beweis:}
 \underline{(i)$\implies$(ii):} $f=f^+-f^-,|f|=f^++f^-$\newline
 \underline{(ii)$\implies$(iii):} $g:=f$ \newline
 \underline{(iii)$\implies$(iv):} $|f|\leq g\implies 0\leq f^+,f^-\leq g$ und mit Monotonie $\displaystyle\int f^+\ d\mu,\int f^-\ d\mu\leq \int g\ d\mu<\infty$. Es gilt also $f^+,f^-\in\mathcal{L}^1(\mu)$ mit $f^+,f^-\geq0$ \newline
 \underline{(iv)$\implies$(i):} $\displaystyle\int f\ d\mu=\int (h_1-h_2)\ d\mu=\int h_1\ d\mu-\int h_2\ d\mu\in\R$, da $0\leq\displaystyle\int h_1\ d\mu,\int h_2\ d\mu<\infty$ \qed
 
 \paragraph{5.18. Korollar:}Seien $f,g\in\mathcal{L}^1(\mu)$. Dann gilt auch $\max(f,g),\min(f,g)\in\mathcal{L}^1(\mu)$.
 
 % Hier habe ich einen etwas eleganteren Beweis verwendet
 
 \paragraph{Beweis:}Es gilt f\"ur $x,y\in\R$ (leicht nachzupr\"ufen)
 $$\max(x,y)=\dfrac{x+y+|x-y|}{2}$$
 und damit (zweimal Dreiecksungleichung und Monotonie)
 \begin{align*}
     \int |\max(f,g)|\ d\mu=\int \left|\dfrac{f+g+|f-g|}{2}\right|\ d\mu
     \leq\dfrac{1}{2}\int2(|f|+|g|)\ d\mu=\int |f|\ d\mu+\int |g|\ d\mu<\infty 
 \end{align*}
 Die Aussage f\"ur das Minimum folgt mit $\min(x,y)=-\max(-x,-y)$. \qed
 
 \paragraph{5.19. Korollar (Dreiecksungleichung f\"ur Integrale):}F\"ur $f\in\mathcal{L}(\mu)$ gilt
 $$\left|\int f\ d\mu\right|\leq\int |f|\ d\mu$$
 
 \paragraph{Beweis:}Es gilt $-|f|\leq f\leq |f|$ und mit der Monotonie $\displaystyle-\int |f|\ d\mu\leq\int f\ d\mu\leq \int |f|\ d\mu$. Die Aussage folgt aus $-y\leq x\leq y\implies |x|\leq y$. \qed
 
 % Bin mir mit der Notation noch nicht sicher
 
\paragraph{5.20. Definition:}Sei $(\Omega,\A,\mu)$ ein Ma\ss{}raum, $(\Omega',\A')$ ein messbarer Raum und $f:(\Omega,\A)\to\nobreak(\Omega',\A')$ eine messbare Abbildung. Dann ist das Bildma\ss{} (Englisch \textit{pushforward}) von $\mu$ unter $f$ definiert als
$$(f\#\mu)(A'):=\mu\left(f^{-1}(A')\right)\text{ f\"ur alle }A'\in\A'$$
 
 \paragraph{5.21. Proposition:}Das Bildma\ss{} ist ein Ma\ss{} auf $(\Omega',\A')$ mit $(f\#\mu)(\Omega')=\mu(\Omega)$. Insbesondere gilt
 \begin{align*}
     \mu\text{ endlich }&\iff(f\#\mu)\text{ endlich}\\
     \mu\text{ Wahrscheinlichkeitsma\ss{} }&\iff(f\#\mu)\text{ Wahrscheinlichkeitsma\ss{}}\\
 \end{align*}
 
 \paragraph{Beweis:}\"Ubung!
 
 \paragraph{5.22. (Transformationssatz):}Sei $(\Omega,\A,\mu)$ ein Ma\ss{}raumm $(\Omega',\A')$ ein messbarer Raum. Seien $f:(\Omega,\A)\to(\Omega',\A')$ und $g:(\Omega',\A')\to(\overline\R,\cB(\overline\R))$ messbar und $(f\#\mu)$ das entsprechende Bildma\ss{} auf $(\Omega',\A')$. Zusammenfassend
 $$\left(\Omega,\A,\mu\right)\nto{f}{}\left(\Omega',\A',(f\#\mu)\right)\nto{g}{}\left(\overline\R,\cB(\overline\R)\right)$$
 Dann gilt
 $$\int_{\Omega}(g\circ f)\ d\mu=\int_{\Omega'}g\ d(f\#\mu)$$
 
 \paragraph{Beweis:}
 \begin{enumerate}[label=\Roman*.]
     \item $g=\ind{A'},A'\in\A'$ Indikatorfunktion\newline
     Hier ist 
     \begin{align*}
         (g\circ f)(\omega)=\ind{A'}(f(\omega))=
         \begin{cases}
             1&\text{ falls }\omega\in f^{-1}(A')\\
             0&\text{ sonst}
         \end{cases}
         =\ind{f^{-1}(A')}(\omega)
     \end{align*}
     Damit folgt
     \begin{align*}
         \int_{\Omega}(g\circ f)\ d\mu&=\int_{\Omega}\ind{f^{-1}(A')}\ d\mu\\
         &=\mu\left(f^{-1}(A')\right)=(f\#\mu)(A')\\
         &=\int_{\Omega'}\ind{A'}\ d(f\#\mu)=\int_{\Omega'}g\ d(f\#\mu)
     \end{align*}
     \item $g=\sum_{i=1}^n\gamma_i\cdot\ind{G_i'},G_i'\in\A',i=1\hdots,n$ einfach\newline
     Hier gilt (wie in I.) 
     $$(g\circ f)(\omega)=\sum_{i=1}^n\gamma_i\cdot\ind{f^{-1}(G_i')}(\omega)$$
     und damit 
     \begin{align*}
         \int_{\Omega}(g\circ f)\ d\mu&=\int_{\Omega}\sum_{i=1}^n\gamma_i\cdot\ind{f^{-1}(G_i')}\ d\mu\\
         &=\sum_{i=1}^n\gamma_i\cdot\int_{\Omega}\ind{f^{-1}(G_i')}\ d\mu\\
         &=\sum_{i=1}^n\gamma_i\cdot\int_{\Omega}\left(\ind{G_i'}\circ f\right)\ d\mu\\
         &\overset{\text{I.}}{=}\sum_{i=1}^n\gamma_i\cdot\int_{\Omega'}\ind{G_i'}\ d(f\#\mu)\\
         &=\int_{\Omega'}\sum_{i=1}^n\gamma_i\cdot\ind{G_i'}\ d(f\#\mu)\\
         &=\int_{\Omega'}g\ d(f\#\mu)
     \end{align*}
     \item $g\geq 0$ nicht-negativ, messbar\newline
     W\"ahle $g_n:\Omega'\to\R,n\geq1$ einfach mit $0\leq g_n\uparrow g$. Dann ist f\"ur alle $n\geq1$ die Zusammensetzung $(g_n\circ f)$ einfach und messbar und insbesondere $0\leq (g_n\circ f)\uparrow (g\circ f)$. Es folgt
     $$\int_\Omega (g\circ f)\ d\mu=\lim_{n\to\infty}\int_\Omega(g_n\circ f)\ d\mu\overset{\text{II.}}{=}\lim_{n\to\infty}\int_{\Omega'}g_n\ d(f\#\mu)=\int_{\Omega'}g\ d(f\#\mu)$$
     \item $g$ allgemein (messbar)\newline
     Schreibe $g=g^+-g^-$ und beachte, dass $(g\circ f)^+=(g^+\circ f)$ und $(g\circ f)^-=(g^-\circ f)$. Es folgt
     \begin{align*}
         \int_\Omega(g\circ f)\ d\mu&=\int_\Omega(g\circ f)^+-(g\circ f)^-\ d\mu\\
         &=\int_\Omega(g^+\circ f)\ d \mu-\int_\Omega(g^-\circ f)\ d\mu\\
         &\overset{\text{III.}}{=}\int_{\Omega'} g^+\ d(f\#\mu)-\int_{\Omega'}g^-\ d(f\#\mu)
         =\int_{\Omega'}g\ d(f\#\mu)
     \end{align*}
     \qed
 \end{enumerate}
 
 \paragraph{5.23. Lemma:}Seien $f_n,n\geq1$ nicht-negativ und messbar, sodass $0\leq f_1\leq\hdots\leq\displaystyle\lim_{n\to\infty}f_n$. Sei $g$ einfach, sodass $0\leq g\leq\displaystyle\lim_{n\to\infty}f_n$. Dann gilt
 $$\int g\ d\mu\leq\lim_{n\to\infty}\int f_n\ d\mu$$
 
 \paragraph{Beweis:} folgt sofort aus Lemma 5.4. \qed
 
 \paragraph{5.24. Satz:}Seien $f_n,n\geq1$ nicht-negativ und messbar, sodass $0\leq f_1\leq\hdots\leq\displaystyle\lim_{n\to\infty}f_n$. Dann gilt 
 $$\lim_{n\to\infty}\int f_n\ d\mu=\int \left(\lim_{n\to\infty} f_n\right)\ d\mu$$
 
 \paragraph{Beweis:}Es gilt $\forall n\geq1:f_n\leq\displaystyle\lim_{n\to\infty}f_n$ mit $\displaystyle\lim_{n\to\infty}f_n\geq0$ messbar. Damit folgt $\displaystyle\lim_{n\to\infty}f_n\in\mathcal{L}(\mu)$ und damit
 $$\forall n\geq 1:\int f_n\ d\mu\leq\int\left(\lim_{n\to\infty}f_n\right)\ d\mu$$
 Es folgt 
 $$\lim_{n\to\infty}\int f_n\ d\mu\leq\int\left(\lim_{n\to\infty} f_n\right)\ d\mu$$
 W\"ahle nun $g_k,k\geq1$, sodass $\displaystyle0\leq g_k\uparrow \lim_{n\to\infty}f_n$. Mit Lemma 5.23 folgt 
 $$\forall k\geq1:\int g_k\ d\mu\leq\lim_{n\to\infty}\int f_n\ d\mu$$
 und damit 
 $$\lim_{k\to\infty}\int g_k\ d\mu\leq\lim_{n\to\infty}\int f_n\ d\mu$$
 Aber per Definition des Integrals gilt
 $$\int\left(\lim_{n\to\infty}f_n\right)\ d\mu=\lim_{k\to\infty}\int g_k\ d\mu$$
 und damit folgt 
 $$\lim_{n\to\infty}\int f_n\ d\mu\geq\int\left(\lim_{n\to\infty} f_n\right)\ d\mu$$
 \qed
 
 \paragraph{5.25. Satz:}Seien $f_n,n\geq1$ und $g$ messbare Funktionen, sodass 
 $$g\leq f_1\leq\hdots\leq\lim_{n\to\infty}f_n$$
 Falls $\displaystyle\int g^-\ d\mu<\infty$, dann gilt
 $$\lim_{n\to\infty}\int f_n\ d\mu=\int \left(\lim_{n\to\infty}f_n\right)\ d\mu$$
 
 \paragraph{Beweis:} Es gilt $g\leq f_n$ und damit $f_n^-\leq g^-$ f\"ur alle $n\geq1$. Mit der Monotonie folgt
 \begin{align*}
     \forall n\geq1:\int f_n^-\ d\mu\leq \int g^-\ d\mu<\infty
 \end{align*}
 Au\ss{}erdem gilt $\displaystyle\left(\lim_{n\to\infty}f_n\right)^-\leq g^-$ und mit der Monotonie
$$\int \left(\lim_{n\to\infty}f_n\right)^-\ d\mu\leq \int g^-\ d\mu$$
Also sind $\displaystyle f_n,\left(\lim_{n\to\infty}f_n\right)\in\mathcal{L}(\mu)$. 
 \begin{enumerate}[label=\Roman*.]
     \item Sei $\displaystyle\lim_{n\to\infty}\int f_n\ d\mu=\infty$\newline
     Mit der Monotonie gilt $\displaystyle\int f_n\ d\mu\leq \int\left(\lim_{n\to\infty}f_n\right)\ d\mu$ f\"ur alle $n\geq1$ und damit 
     $$\lim_{n\to\infty}\int f_n\ d\mu\leq \int\left(\lim_{n\to\infty}f_n\right)\ d\mu$$
     Nun ist $\displaystyle\lim_{n\to\infty}\int f_n\ d\mu=\infty$ und es gilt 
     $$\lim_{n\to\infty}\int f_n\ d\mu=\int\left(\lim_{n\to\infty}f_n\right)\ d\mu=\infty$$
     \item Sei $\displaystyle\lim_{n\to\infty}\int f_n\ d\mu<\infty$\newline
     Laut Annahme gilt mit der Monotonie f\"ur alle $n\geq1$
     $$\int g\ d\mu\leq\int f_n\ d\mu\leq\lim_{n\to\infty}\int f_n\ d\mu<\infty$$
     Daher gilt $\displaystyle g,f_n,\left(\lim_{n\to\infty}f_n\right)\in\mathcal{L}^1(\mu)$ und mit Proposition 5.15 gilt $\displaystyle g,f_n,\left(\lim_{n\to\infty}f_n\right)\in\R$ a.e. Da es hier nur um die Werte der Integrale geht, seien also alle Funktionen reellwertig in $\Omega$. Dann gilt f\"ur alle $n\geq1$, dass $0\leq(f_n-g)\uparrow \displaystyle\lim_{n\to\infty}(f_n-g)$ und mit Satz 5.24 folgt
     $$\lim_{n\to\infty}\int (f_n-g)\ d\mu=\int\left(\lim_{n\to\infty}(f_n-g)\right)\ d\mu$$ 
     Die Aussage folgt schlie\ss{}lich aus der Linearit\"at des Integrals. \qed
 \end{enumerate}
 
 \paragraph{5.26. Proposition:}Sei $f:[a,b]\to\R$ messbar. Falls $f$ auf $[a,b]$ Riemann-integrierbar ist, dann ist $f$ auch Lebesgue-integrierbar (bzgl. dem Lebesgue-Ma\ss{} $\lambda=\operatorname{vol}$ auf $([a,b],\cB([a,b]))$) und
 $$\int_a^b f(x)\ dx=\int\displaylimits_{[a,b]}f\ d\lambda$$
 
 \paragraph{Beweis:}Seien $U_n,O_n,n\geq1$ Unter- bzw. Obersummen von $f$, sodass
 $$\lim_{n\to\infty}U_n=\lim_{n\to\infty}O_n=\int_a^b f(x)\ dx$$
 per Konstruktion des Riemann-Integrals. Insbesondere sind $U_n,O_n<\infty$ f\"ur alle $n\geq1$ und jede Unter- bzw. Obersumme entspricht dem Integral einer Treppenfunktion $u_n$ bzw $o_n$ f\"ur $n\geq1$, sodass $u_1\leq u_2\leq\hdots\leq f\leq\hdots\leq o_2\leq o_1$. 
 auf $[a,b]$. Als Treppenfunktionen sind $u_n,o_n$ insbesondere einfach und damit messbar. Es folgt (einfach zu pr\"ufen)
 $$U_n=\int u_n\ d\lambda\text{ und }O_n=\int o_n\ d\lambda$$ 
 Da $f$ Riemman-integrierbar auf $[a,b]$ ist folgt $|U_1|,|O_1|<\infty$ und mit der Monotonie folgt
 $$\forall n\geq1:U_n\leq\int\displaylimits_{[a,b]}f\ d\lambda\leq O_n$$
 Die Aussage folgt nun aus $\displaystyle\int_a^b f(x)\ dx=\lim_{n\to\infty}U_n\leq\int\displaylimits_{[a,b]}f\ d\lambda\leq\lim_{n\to\infty}O_n=\int_a^b f(x)\ dx$. \qed
 
 \paragraph{Bemerkung:}Sei $f$ Riemann-integrierbar auf $[a,b]$, aber nicht unbedingt messbar. Dann gilt f\"ur $f^*:=\displaystyle\lim_{n\to\infty}o_n$
 \begin{itemize}
     \item $f^*:[a,b]\to\R$ ist messbar.
     \item $\displaystyle\int_a^b f(x)\ dx=\int\displaylimits_{[a,b]}f\ d\lambda$
     \item $\left\{x\in[a,b]:f(x)\neq f^*(x)\right\}\subseteq N\in\borel$ mit $\mu(N)=0$
 \end{itemize}
 
 \paragraph{Korollar 5.27:} Sei $f:\R\to[0,\infty)$ messbar und (uneigentlich) Riemann-integrierbar auf $[0,\infty)$. Dann ist $f$ auch Lebesgue-integrierbar auf $[0,\infty)$ und die Integralbegriffe stimmen \"uberein.
 
 \paragraph{Beweis:}Laut Annahme gilt
 $$\lim_{t\to\infty}\int_0^tf(x)\ dx<\infty$$
 und insbesondere ist $f$ damit Riemann-integrierbar auf $[0,t]$ f\"ur alle $t>0$. Es folgt
 \begin{align*}
     \int_0^\infty f(x)\ dx&=\lim_{t\to\infty}\int_0^tf(x)\ dx\\&\overset{5.26}{=}\lim_{t\to\infty}\int\displaylimits_{[0,t]}f\ d\lambda\\
     &=\lim_{t\to\infty}\int f\cdot\ind{[0,t]}\ d\lambda\\
     &\overset{5.24}{=}\int \lim_{t\to\infty}\left(f\cdot\ind{[0,t]}\right)\ d\lambda\\
     &=\int\displaylimits_{[0,\infty)}f\ d\lambda
 \end{align*} 
 \qed
 
 \paragraph{Bemerkung:}Die Bedingung $f\geq0$ ist notwendig! (Gegenbeispiel $\sin (x)/x$)
 
 \paragraph{5.28. Beispiel:} % Hier war denke ich das sin x / x Beispiel
 
 \paragraph{5.29. Proposition:}Sei $f:(\Omega,\A)\to(\R,\cB(\overline\R))$ nicht-negativ und messbar. Definiere f\"ur $A\in\A$ 
 $$\nu(A):=\int\displaylimits_A f\ d\mu$$
 Dann ist $\nu:\A\to[0,\infty]$ ein Ma\ss{} auf $(\Omega,\A)$ und f\"ur $f\in\mathcal{L}^1(\nu)$ gilt
 $$\int g\ d\nu=\int fg\ d\mu$$
 
 \paragraph{Beweis:}
 \begin{enumerate}[label=\Roman*.]
     \item Zeige zun\"achst, dass $\nu$ ein Ma\ss{} ist.\newline
     \begin{enumerate}[label=(\roman*)]
        \item $\displaystyle f\geq0\implies f\cdot\ind{A}\geq0\implies\int_A f\ d\mu\in[0,\infty]$, womit $\nu:\A\to[0,\infty]$ wohldefiniert ist.
        \item $\nu(\emptyset)=\displaystyle\int_\emptyset f\ d\mu=\int f\cdot\ind{\emptyset}\ d\mu=\int 0 \ d\mu=0$
        \item Seien $A_n\in\A,n\geq1$ disjunkt. Dann gilt
        \begin{align*}
            \nu\left(\bigcup_{n\geq1}A_n\right)&=\int\displaylimits_{\bigcup_{n\geq1}A_n}f\ d\mu=\int f\cdot\ind{\bigcup_{n\geq1}A_n}\ d\mu\\
            &=\int f\cdot\left(\lim_{N\to\infty}\sum_{n=1}^N\ind{A_n}\right)\ d\mu\\
            &\overset{5.24}{=}\lim_{N\to\infty}\int f\cdot\sum_{n=1}^N\ind{A_n}\ d\mu\\
            &=\sum_{n\geq1}\int_{A_n} f\ d\mu=\sum_{n\geq1}\nu(A_n)
        \end{align*}
     \end{enumerate}
     \item $g=\ind{A},A\in\A$ Indikatorfunktion\newline
     Hier gilt
     $$\int g\ d\nu=\int \ind{A}\ d\nu=\nu(A)=\int\displaylimits_{A} f\ d\mu=\int f\cdot\ind{A}\ d\mu=\int fg\ d\mu$$
     \item $g=\sum_{i=1}^n\gamma_i\cdot\ind{G_i}$ einfach\newline
     Hier gilt
     \begin{align*}
         \int g\ d\nu=\sum_{i=1}^n\gamma_i\cdot\nu(G_i)\overset{\text{II.}}{=}\sum_{i=1}^n\gamma_i\cdot\left(\int\displaylimits_{G_i}f\ d\mu\right)=\int f\cdot\left(\sum_{i=1}^n\gamma_i\cdot\ind{G_i}\right)\ d\mu=\int fg\ d\mu
     \end{align*}
     \item $g\geq0$ nicht-negativ, messbar\newline
     W\"ahle $g_n,n\geq1$ einfach mit $0\leq g_n\uparrow g$. Wegen III. gilt $\displaystyle\int g_n\ d\nu=\int fg_n\ d\mu$ f\"ur alle $n\geq1$. Au\ss{}erdem gilt wegen $f\geq0$, dass $0\leq fg_n\uparrow fg$ und damit
     $$\int g \ d\nu=\lim_{n\to\infty}\int g_n\ d\nu=\lim_{n\to\infty}\int fg_n\ d\mu\overset{5.24}{=}\int \lim_{n\to\infty}fg_n\ d\mu=\int fg\ d\mu$$
     \item $g$ messbar\newline
     Sei $g=g^+-g^-$. Wegen IV. gilt $\displaystyle\int g^+\ d\nu=\int fg^+\ d\mu$ und $\displaystyle\int g^-\ d\nu=\int fg^-\ d\mu$. Damit folgt
     $$\int g\ d\nu=\int g^+\ d\nu-\int g^-\ d\nu=\int f(g^+-g^-)\ d\mu=\int fg\ d\mu$$
     \qed
 \end{enumerate}
 
 \paragraph{Bemerkung:}$\nu$ erbt alle Nullmengen von $\mu$ (kann jedoch auch zus\"atzliche Nullmengen haben). Es gilt also 
 $$\mu(A)=0\implies\nu(A)=0$$
 
 \paragraph{5.30. Definition:}F\"ur $f$ und $\nu$ wie in Proposition 5.29 nennt man $f$ die Dichte von $\nu$ bez\"uglich $\mu$. Kurz $f=\dfrac{d\nu}{d\mu}$. Ist $X$ eine reellwertige Zufallsvariable und $\Pp(X\in A)=\nu(A)$, dann nennt man $f$ auch die Dichte von $X$ bez\"uglich $\mu$. 
 
 \paragraph{Bemerkung:}In Proposition 5.29 sind ein Ma\ss{} $\mu$ und eine Dichte $f$ gegeben. Falls zwei Ma\ss{}e $\mu,\nu$ gegeben sind, liefert der Satz von Radon\textendash Nikodym (cf. Wahrscheinlichkeitstheorie 2) Bedingungen an $\mu$ und $\nu$ f\"ur die Existenz einer Dichte $f$.
 
 \chapter*{6. Ungleichungen}
 \addcontentsline{toc}{chapter}{6. Ungleichungen}

Sei in diesem Kapitel $\pspace$ ein Wahrscheinlichkeitsraum, und $X,Y:(\Omega,\A)\to(\overline\R, \cB(\overline\R))$ Zufallsvariablen.


\section*{Markov-Ungleichung}
\addcontentsline{toc}{section}{Markov-Ungleichung}

 
 \paragraph{6.1. Satz (Markov-Ungleichung):}Sei $X\geq0$ und $c>0$. Dann gilt
 $$\Pp(X\geq c)\leq c^{-1}\cdot\E X$$
 
 \paragraph{Beweis:}
 \begin{align*}
     \E X=\int\displaylimits_\Omega X\ d\Pp&=\int\displaylimits_{X\geq c}X\ d\Pp+\int\displaylimits_{X<c}X\ d\Pp\\
     &\geq\int\displaylimits_{X\geq c}X\ d\Pp\\
     &\geq \int\displaylimits_{X\geq c}c\ d\Pp=c\cdot \Pp(X\geq c)
 \end{align*}
\qed

\paragraph{6.2. Korollar:}Sei $X\geq 0$ und $c>0$. Dann gilt sogar
$$\Pp(X\geq c)\leq c^{-1}\cdot\E[X\cdot\ind{X\geq c}]$$

\paragraph{Beweis:}wie 6.1. \qed

\paragraph{6.3. Korollar (Chebyshev-Ungleichung):}Sei $X\in\mathcal{L}^1(\Pp)$ und $c>0$. Dann gilt
$$\Pp\left(|X-\E X|\geq c\right)\leq c^{-2}\cdot\Var(X)$$
 
 \paragraph{Beweis:} Mit der Markov-Ungleichung folgt
 $$\Pp\left(|X-\E X|\geq c\right)=\Pp\left((X-\E X)^2\geq c^2\right)\overset{6.1}{\leq}c^{-2}\cdot\E[(X-\E X)^2]=c^{-2}\cdot\Var(X)$$
 
 \paragraph{6.4. Korollar (Chernoff-Schranke):}Sei $X$ eine reellwertige Zufallsvariable und $c>0$. Dann gilt
 $$\Pp(X\geq c)\leq\inf_{t>0}e^{-tc}\cdot M_X(t)$$
 wobei $M_X(t)=\E[e^{tX}]$ die momenterzeugende Funktion von $X$ ist.
 
 \paragraph{Beweis:} Sei $t>0$. Dann gilt mit der Markov-Ungleichung
 $$\Pp(X\geq c)=\Pp\left(e^{tX}\geq e^{tc}\right)\overset{6.1}{\leq}e^{-tc}\cdot M_X(t)$$
 Diese Ungleichung gilt f\"ur alle $t>0$. Daher k\"onnen wir das Infimum nehmen und die Chernoff-Schranke folgt. \qed
 
 \section*{Konvexit\"at und Jensen-Ungleichung}
\addcontentsline{toc}{section}{Konvexit\"at und Jensen-Ungleichung}
 
 \paragraph{6.5. Definition:}Sei $(a,b)\subseteq\R$ ein nicht-leeres Intervall. Dann ist eine Abbildung $f:(a,b)\to\R$ konvex, falls gilt
 $$\forall x,y\in(a,b),\forall\alpha\in(0,1):f(\alpha x+(1-\alpha)y)\leq\alpha f(x)+(1-\alpha)f(y)$$
 
 \paragraph{6.6. Lemma:}Sei $f:(a,b)\to\R$. Dann ist $f$ genau dann konvex, wenn gilt
 $$\forall s,t,u:a<s<t<u<b:\dfrac{f(t)-f(s)}{t-s}\leq\dfrac{f(u)-f(t)}{u-t}$$
 
 \paragraph{Beweis:}
 \begin{enumerate}[label=\Roman*.]
     \item Sei zun\"achst $f$ konvex (nach Definition 6.5)\newline
     Setze $\alpha:=\dfrac{u-t}{u-s}$. Dann ist $1-\alpha=\dfrac{t-s}{u-s}$ und $\alpha s+(1-\alpha)u=t$. Aus der Konvexit\"at von $f$ folgt
     \begin{gather}
         f(\alpha s+(1-\alpha)u)=f(t)\leq\alpha f(s)+(1-\alpha)f(u)
     \end{gather}
     Damit gilt
     $$f(t)-f(s)\leq(1-\alpha)[f(u)-f(s)]=\dfrac{t-s}{u-s}[f(u)-f(s)]$$
     und
     $$\dfrac{f(t)-f(s)}{t-s}\leq\dfrac{f(u)-f(s)}{u-s}$$
     Aus (1) folgt ebenfalls
     $$f(t)-f(u)\leq\alpha[f(s)-f(u)]$$
     und damit 
     $$\dfrac{f(u)-f(t)}{u-t}\geq\dfrac{f(u)-f(s)}{u-s}$$
     \item Sei nun die Bedingung aus Lemma 6.6 erf\"ullt\newline
     Falls $x<y$, setze 
     $$s:=x,\ t:=\alpha x+(1-\alpha)y,\ u:= y$$ 
     Dann ist
     $$\alpha = \dfrac{u-t}{u-s},\ 1-\alpha=\dfrac{t-s}{u-s}$$
     Laut Annahme gilt
     $$f(t)\leq f(s)+\dfrac{t-s}{u-s}[f(u)-f(s)]=\alpha f(x)+(1-\alpha)f(y)$$
     Falls $x>y$, setze
     $$s:=y,\ t:=\alpha x+(1-\alpha)y,\ u:=x$$
     Dann ist
     $$\alpha=\dfrac{t-s}{u-s},\ 1-\alpha=\dfrac{u-t}{u-s}$$
     und die Aussage folgt wie oben aus der Annahme. \qed
 \end{enumerate}
 
 \paragraph{6.7. Lemma:}Sei $f:(a,b)\to\R$. $f$ ist genau dann konvex, wenn gilt
 $$\forall x\in(a,b),\exists\gamma\in\R,\forall y\in(a,b):f(y)\geq f(x)+\gamma(y-x)$$
 
 \paragraph{Beweis:}
 \begin{enumerate}[label=\Roman*.]
     \item Sei $f$ konvex (nach Definition 6.5)\newline
     W\"ahle $a<x_-<x<b$ und setze
     $$\gamma:=\inf_{y\in(x,b)}\dfrac{f(y)-f(x)}{y-x}\geq\dfrac{f(x)-f(x_-)}{x-x_-}>-\infty$$
     F\"ur $x=y$ ist die Aussage trivial. F\"ur $x<y$ ist 
     $$\dfrac{f(y)-f(x)}{y-x}\geq\gamma$$
     sodass $f(y)\geq f(x)+\gamma(y-x)$. F\"ur $x>y$ ist 
     $$\dfrac{f(x)-f(y)}{x-y}\leq \gamma$$
    da $\gamma\geq \dfrac{f(x)-f(x_-)}{x-x_-}$ f\"ur alle $x_-\in(a,x)$, sodass auch $f(y) \geq\nobreak f(x)+\gamma(y-x)$.
     \item Sei nun die Bedingung aus Lemma 6.7 erf\"ullt\newline
     W\"ahle $a<s<t<u$. Aus der Annahme folgt $f(s)\geq f(t)+\gamma(s-t)$ und damit 
     $$\gamma\geq\dfrac{f(t)-f(s)}{t-s}$$
     Ebenfalls folgt aus der Annahme $f(u)\geq f(t)+\gamma(u-t)$ und damit 
     $$\gamma\leq\dfrac{f(u)-f(t)}{u-t}$$
     Die Konvexit\"at von $f$ folgt mit Lemma 6.6.\qed
 \end{enumerate}
 
 \paragraph{6.8. Korollar:}Sei $f:(a,b)\to\R$ differenzierbar auf $(a,b)$. Dann gilt
 $$f\text{ konvex}\iff f'\text{ monoton nicht-fallend}$$
 
 \paragraph{Beweis:}
 \begin{enumerate}[label=\Roman*.]
     \item Sei $f$ konvex (nach Definition 6.5)\newline
     Seien $a<s<u<b$ und w\"ahle $s_+,t_-,t_+,u_-$ so, dass
     $$s<s_+<t_-<t_+<u_-<u$$
     Mit Lemma 6.6 folgt
     $$\dfrac{f(s_+)-f(s)}{s_+-s}\leq\dfrac{f(t_-)-f(s_+)}{t_--s_+}\leq\dfrac{f(t_+)-f(t_-)}{t_+-t_-}\leq\dfrac{f(u_-)-f(t_+)}{u_--t_+}\leq\dfrac{f(u)-f(u_-)}{u-u_-}$$
     und 
     $$f'(s)=\lim_{s_-\searrow s}\dfrac{f(s_+)-f(s)}{s_+-s}\leq\dfrac{f(t_+)-f(t_-)}{t_+-t_-}\leq\lim_{u_-\nearrow u}\dfrac{f(u)-f(u_-)}{u-u_-}=f'(u)$$
     Also folgt f\"ur alle $u,s\in(a,b)$ mit $s<u$, dass $f'(s)\leq f'(u)$.
     \item Sei $f'$ monoton nicht-fallend\newline
     Seien $a<s<t<u<b$. Es gilt (Fundamentalsatz der Analysis, Mittelwertsatz)
     $$f(t)=f(s)+\int_s^t f(z)\ dz\leq f(s)+(t-s)f'(t)$$
     und damit 
     $$\dfrac{f(t)-f(s)}{t-s}\leq f'(t)$$
     Ebenfalls gilt (wie oben)
     $$f(u)=f(t)+\int_t^uf(z)\ dz\geq f(t)+(u-t)f'(t)$$
     und damit 
     $$\dfrac{f(u)-f(u)}{u-t}\geq f'(t)$$
     Die Aussage folgt mit Lemma 6.6. \qed
 \end{enumerate}
 
 \paragraph{6.9. Lemma:}Ist $f$ konvex auf $(a,b)$, dann ist $f$ stetig auf $(a,b)$.
 
 \paragraph{Beweis:}Sei $x\in(a,b)$. F\"ur $s<y<x$ ist mit Lemma 6.7
 $$f(y)\geq f(x)+\gamma (y-x)$$
 und 
 $$f(y)\leq \alpha f(x)+(1-\alpha)f(s)$$
 f\"ur $\alpha=\dfrac{y-s}{x-s}$. Damit gilt $\displaystyle\lim_{y\nearrow x}=f(x)$. Analog folgt auch $\lim_{y\searrow x}f(y)=f(x)$. \qed
 
 \paragraph{6.10. Satz (Jensen-Ungleichung):}Sei $X\in\mathcal{L}^1(\Pp)$ mit $X:(\Omega,\A)\to((a,b),\cB((a,b)))$, mit $(a,b)\subseteq\R$. Ist $f:(a,b)\to\R$ konvex, dann gilt
 $$f(\E X)\leq \E f(X)$$
 
 \paragraph{Beweis:}Es gilt $\E X\in (a,b)$ (Monotonie, $\Pp(\Omega)=1$). Mit Lemma 6.7 gilt
 $$f(X)\geq f(\E X)+(X-\E X)\cdot \gamma=:Z$$
 Es gilt $Z\in\mathcal{L}^1(\Pp)$ (leicht nachzupr\"ufen) und 
 $$\E Z=\E[f(\E X)]+\gamma\cdot \E[X-\E X]=f(\E X)$$
 Da $f(X)\geq Z$, gilt $[f(X)]^-\leq Z^-\in\mathcal{L}^1(\Pp)$ und damit $f(X)\in\mathcal{L}(\Pp)$. Mit der Monotonie folgt 
 $$f(\E X)=\E[f(\E X)]=\int Z\ d\Pp\leq\int f(X)\ d\Pp=\E f(X)$$
 \qed
 
\section*{H\"older-Ljapunov-Minkowski}
\addcontentsline{toc}{section}{H\"older-Ljapunov-Minkowski}

\paragraph{6.11. Lemma (Young-Ungleichung):} Es sei $p\in(1,\infty)$ und $\frac{1}{p}+\frac{1}{q}=1$. Für $a,b\geq0$ gilt
$$ab\leq\dfrac{a^p}{p}+\dfrac{b^q}{q}$$
mit Gleichheit genau dann, wenn $a^p=b^q$.

\paragraph{Beweis:}Falls $a=0$ oder $b=0$ gilt die Ungleichung trivial. Es gelte also $a,b>0$. Setze $t:=p^{-1}$ und damit $1-t=q^{-1}$. Da $x\mapsto \log x$ konkav ist gilt
$$\log\left(ta^p+(1-t)b^q\right)\geq t\log\left(a^p\right)+(1-t)\log\left(b^q\right)=\log(ab)$$
und die Ungleichung folgt mit Anwendung von $\exp$ auf beiden Seiten. \qed

\paragraph{6.12. Satz (H\"older-Ungleichung):} Sei nun $(\Omega,\A,\mu)$ ein Ma\ss{}raum und $f,g\in\mathcal{L}(\mu)$. Sei $p\in(1,\infty)$ und $q$ der konjugierte (duale) Index zu $p$, i.e. $\dfrac{1}{p}+\dfrac{1}{q}=1$. Dann gilt 
$$\int |fg|\ d\mu\leq\left(\int |f|^p\ d\mu\right)^{1/p}\cdot\left(\int |g|^q\ d\mu\right)^{1/q}$$

\paragraph{Beweis:}
\begin{enumerate}[label=\Roman*.]
    \item Fall: $\displaystyle\int |f|^p\ d\mu=0$ oder $\displaystyle\int |g|^q\ d\mu=0$\newline
    Sei o.B.d.A. der erste Fall zutreffend. Dann gilt $|f|=0$ a.e. und (mit der Konvention $0\cdot\infty=0$) folgt $|fg|=0$ a.e. und die Aussage ist trivial.
    \item Fall: $\displaystyle\int |f|^p\ d\mu, \displaystyle\int |g|^q\ d\mu>0$\newline
    Die Aussage ist trivial, falls eines der Integrale unendlich ist. Es seien also beide Integrale reellwertig. Setze
    $$A:=\dfrac{|f|^p}{\displaystyle\int|f|^p\ d\mu},\ B:=\dfrac{|g|^q}{\displaystyle\int|g|^q\ d\mu}$$
    Mit Lemma 6.11 gilt 
    $$\dfrac{|fg|}{\displaystyle\left(\int|f|^p\ d\mu\right)^{1/p}\displaystyle\left(\int|g|^q\ d\mu\right)^{1/q}}\leq\dfrac{|f|^p}{p\cdot\left(\displaystyle\int |f|^p\ d\mu\right)}+\dfrac{|g|^q}{q\cdot\left(\displaystyle\int |g|^q\ d\mu\right)}$$
    Die Ungleichung folgt mit der Monotonie. \qed
\end{enumerate}

\paragraph{Bemerkung:}Die Ungleichung h\"alt auch f\"ur $p=1$ und $p=\infty$ mit entsprechenden konjugierten Indizes $q=\infty$ und $q=1$. Au\ss{}erdem sei f\"ur $p\in(0,\infty)$
$$\mathcal{L}^p(\Omega,\A,\mu):=\left\{f:(\Omega,\A)\to(\R,\borel):\int |f|^p\ d\mu<\infty\right\}$$

\paragraph{6.13. Korollar (Cauchy\textendash Schwarz-Ungleichung):}
$$\int |fg|\ d\mu\leq\left(\int |f|^{1/2}\ d\mu\right)^{1/2}\left(\int |g|^{1/2}\ d\mu\right)^{1/2}$$

\paragraph{Bemerkung:}F\"ur Zufallsvariablen $X,Y\in\mathcal{L}^2(\Pp)$ mit $\Var(X),\Var(Y)>0$ definiere den Korrelationskoeffizienten
$$\rho_{X,Y}:=\dfrac{\E\left[(X-\E X)(Y-\E Y)\right]}{\sqrt{\Var(X)\cdot\Var(Y)}}$$
Dieser ist wegen Korollar 6.13 und Satz 6.12 wohldefiniert und es gilt $\rho_{X,Y}\in[-1,1]$.

\paragraph{6.14. Korollar (Ljapunov-Ungleichung):}Betrachte einen endlichen Ma\ss{}raum $(\Omega,\A,\mu)$ mit $\mu(\Omega)=1$ und eine messbare Abbildung $f:(\Omega,\A)\to(\overline\R,\cB(\overline\R))$. F\"ur $1\leq p\leq q<\infty$ gilt
$$\left(\int |f|^p\ d\mu\right)^{1/p}\leq\left(\int |f|^q\ d\mu\right)^{1/q}$$
 
 \paragraph{Beweis:}Setze $A:=|f|^p, B:=1, a:=q/p, b:=q/(q-p)=a/(a-1)$. Dann gilt $\frac{1}{a}+\frac{1}{b}=1$ und mit Satz 6.12 folgt
 $$\int |f|^p\ d\mu=\int |AB|\ d\mu\leq\left(\int |A|^a\ d\mu\right)^{1/a}\cdot\left(\int |B|^b\ d\mu\right)^{1/b}=\left(\int |f|^q\ d\mu\right)^{p/q}$$
 \qed
 
 \paragraph{Bemerkung:} Die Ungleichung l\"asst sich nat\"urlich auf beliebige endliche Ma\ss{}r\"aume erweitern. In der englischsprachigen Literatur wird unter der Ljapunov-Ungleichung oft ein Korollar der H\"older-Ungleichung (Log-Konvexit\"at von $L^p$) angegeben, siehe z.B. Problem 3.12  und Problem 3.13 aus Teschl, G. (2024) \textit{Topics in Real Analysis}., p. 83. 
 
 \paragraph{6.15. Satz (Minkowski-Ungleichung):}F\"ur $p\in[1,\infty)$ gilt
 $$\left(\int|f+g|^p\ d\mu\right)^{1/p}\leq\left(\int |f|^p\ d\mu\right)^{1/p}+\left(\int |g|^p\ d\mu\right)^{1/p}$$
 
 \paragraph{Beweis:}Der Beweis ist trivial, falls $\displaystyle\int |f+g|^p\ d\mu=0$ oder einer der Summanden auf der rechten Seite unendlich ist. Sei also $\displaystyle\int |f+g|^p\ d\mu>0$ und $f,g\in\mathcal{L}^p(\mu)$. Der Fall $p=1$ folgt aus der Dreiecksungleichung f\"ur Betrag und Monotonie. Sei also $p\in(0,\infty)$. Es gilt $|f+g|^p=|f+g|\cdot|f+g|^{p-1}\leq|f|\cdot|f+g|^{p-1}+|g|\cdot|f+g|^{p-1}$. Mit der Monotonie folgt
 $$\int|f+g|\ d\mu\leq\int |f|\cdot|f+g|^{p-1}\ d\mu+\int |g|\cdot|f+g|^{p-1}\ d\mu$$
 Wende nun die H\"older-Ungleichung mit $q:=p/(p-1)$ an. Dann gilt
 \begin{align*}
     \int|f+g|^p\ d\mu
     \leq& \left(\int |f|^p\ d\mu\right)^{1/p}\cdot\left(\int |f+g|^{p-1\cdot\frac{p}{p-1}}\ d\mu\right)^{\frac{p-1}{p}}\\&+\left(\int |g|^p\ d\mu\right)^{1/p}\cdot\left(\int |f+g|^{p-1\cdot\frac{p}{p-1}}\ d\mu\right)^{\frac{p-1}{p}}\\
     =&\left(\int |f+g|^p\ d\mu\right)^{\frac{p-1}{p}}\left[\left(\int |f|^p\ d\mu\right)^{1/p}+\left(\int |g|^p\ d\mu\right)^{1/p}\right]
 \end{align*}
 
 \qed
 
 \paragraph{Bemerkung:}F\"ur $p\in[1,\infty)$ ist die Abbildung
 $$\Vert f\Vert_p:=\left(\int |f|^p\ d\mu\right)^{1/p}$$
eine Halbnorm auf $\mathcal{L}^p(\mu)$. F\"ur den Quotientenraum $L^p(\mu)$ bez\"uglich der \"Aquivalenzrelation $f\sim g\iff f=g$ a.e. bildet $\Vert\cdot \Vert_p$ eine Norm.
 
\chapter*{7. Unabh\"angigkeit}
\addcontentsline{toc}{chapter}{7. Unabh\"angigkeit}

In diesem Kapitel sei $\pspace$ ein Wahrscheinlichkeitsraum und $X,Y:(\Omega,\A)\to(\Omega',\A')$ Zufallsvariablen. Sei au\ss{}erdem $I\neq\emptyset$ eine beliebige Indexmenge. 
 
\section*{Unabh\"angigkeit von Ereignissen und Zufallsvariablen}
\addcontentsline{toc}{section}{Unabh\"angigkeit von Ereignissen und Zufallsvariablen}
 
\paragraph{7.1. Definition:}Ereignisse $A_i\in\A_i,i\in I$ sind unabh\"angig, falls gilt
$$\forall J\subseteq I\text{ mit }|J|<\infty:\Pp\left(\bigcap_{j\in J}A_j\right)=\prod_{j\in J}\Pp(A_j)$$ 
Kurz $A_i,i\in I$ u.a.

\paragraph{7.2. Beispiel:}Paarweise Unabh\"angigkeit impliziert nicht unbedingt Unabh\"angigkeit. Betrachte zum Beispiel $X,Y\in \mathcal{U}\{0,1\}$ unabh\"angig diskret-gleichverteilt und setze $Z:=(X+Y)\bmod 2$
 
\paragraph{Bemerkung:}Betrachte unabh\"angige Eregnisse $A,B$. Dann sind auch folgende Ereignisse u.a.:
$$A,B^c\ \ \ \ A^c,B\ \ \ \ A^c,B^c\ \ \ \ A,\emptyset\ \ \ \ A,\Omega\ \ \ \ \text{etc.}$$
Also ist jedes Ereignis in $\sigma(\{A\})=\{\emptyset,\Omega,A,A^c\}$ von jedem Ereignis in $\sigma(\{B\})=\{\emptyset,\Omega,B,B^c\}$ u.a.

\paragraph{7.3. Definition:}Familien von Ereignissen $\G_i\subseteq\A,i\in I$ sind unabh\"angig, wenn f\"ur jede Auswahl von Mengen $G_i\in\G_i$ die entsprechenden Ereignisse unabh\"angig sind. Insbesondere folgt damit aus der Unabh\"angigkeit eines Mengensystems auch die Unabh\"angigkeit aller gr\"oberen Mengensysteme, i.e.
$$\forall i\in I:\G_i,i\in I\text{ u.a. und }\mathcal{F}_i\subseteq\G_i\implies \mathcal{F}_i,i\in I\text{ u.a.}$$

\paragraph{7.4. Definition:}Zufallsvariablen $X_i,i\in I$ sind unabh\"angig, wenn die $\sigma$-Algebren $\sigma(X_i),i\in I$ unabh\"angig sind.

\paragraph{7.5. Proposition:}Seien $(X_i,\mathcal{X}_i)$ und $(Y_i,\mathcal{Y}_i)$ messbare R\"aume und $X_i:(\Omega,\A)\to(X_i,\mathcal{X}_i)$ unabh\"angige Zufallsvariablen f\"ur $i\in I$. Seien au\ss{}erdem $g_i:(X_i,\mathcal{X}_i)\to(Y_i,\mathcal{Y}_i)$ messbare Abbildungen f\"ur $i\in I$. Dann sind auch die Zufallsvariablen $(g_i\circ X_i):(\Omega,\A)\to(Y_i,\mathcal{Y}_i), i\in I$ unabh\"angig. 

\paragraph{Beweis:}F\"ur $i\in I$ gilt
\begin{align*}
    \sigma(g_i\circ X_i)&=\sigma\left(\left\{(g_i\circ X_i)^{-1}(A_i):A_i\in\mathcal{Y}_i\right\}\right)\\
    &=\sigma\left(\left\{X_i^{-1}\left(g_i^{-1}(A_i)\right):A_i\in\mathcal{Y}_i\right\}\right)\\
    &\subseteq\sigma\left(\left\{X_i^{-1}\left(B_i\right):B_i\in\mathcal{X}_i\right\}\right)
\end{align*}
wobei die letze Inklusion aus der Messbarkeit von $g_i$ folgt. Die Aussage folgt aus Definitionen 7.3 und 7.4. \qed

\paragraph{7.6. Proposition:}Betrachte unabh\"angige Zufallsvariablen $X,Y:(\Omega,\A)\to(\overline\R,\cB(\overline\R))$, sodass $X,Y\in\mathcal{L}^1(\Pp)$. Dann gilt $XY\in\mathcal{L}^1(\Pp)$ und $\E[XY]=(\E X)\cdot (\E Y)$.

\paragraph{Beweis:}
\begin{enumerate}[label=\Roman*.]
    \item $X,Y$ Indikatorfunktionen\newline
    Seien hier $X=\ind{A},Y=\ind{B}$ mit $A,B\in\A$. Dann gilt $A=\{X=1\},B=\{Y=1\}$ und $A,B$ sind unabh\"angig. Es gilt $XY=\ind{A\cap B}\in\mathcal{L}^1(\Pp)$ (einfache \"Uberlegung) und 
    $$\E [XY]=\int \ind{A\cap B}\ d\Pp=\Pp(A\cap B)=\Pp(A)\cdot \Pp(B)=(\E X)\cdot(\E Y)$$
    \item $X,Y$ einfache Funktionen\newline
    Seien $X=\displaystyle\sum_{i=1}^n\alpha_i\cdot\ind{A_i},Y=\sum_{j=1}^m\beta_j\cdot\ind{B_j}$ mit $A_i$ disjunkt und $\alpha_i$ alle verschieden f\"ur $i=1,\hdots,n$ und $B_j$ disjunkt und $\beta_j$ alle verschieden f\"ur $j=1\hdots,m$. Dann sind $A_i=\{X=\alpha_i\}$ und $B_j=\{Y=\beta_j\}$ u.a. f\"ur $i=1,\hdots,n$ und $j=1\hdots,m$. Au\ss{}erdem ist $XY$ wieder einfach und damit $XY\in\mathcal{L}^1(\Pp)$ und es gilt
    \begin{align*}
        \E[XY]&=\int \sum_{i=1}^n\sum_{j=1}^m\alpha_i\beta_j\cdot\ind{A_i}\ind{B_j}\ d\Pp\\
        &=\sum_{i=1}^n\sum_{j=1}^m\alpha_i\beta_j\cdot\Pp(A_i\cap B_j)\\
        &=\sum_{i=1}^n\sum_{j=1}^m\alpha_i\beta_j\cdot\Pp(nA_i)\Pp(B_j)\\
        &=\left(\sum_{i=1}^n\alpha_i\cdot\Pp(A_i)\right)\cdot \left(\sum_{j=1}^m\beta_j\cdot\Pp(B_j)\right)=(\E X)\cdot(\E Y)
    \end{align*}
    \item $X,Y$ nicht negativ, messbar\newline
    W\"ahle einfache Funktionen $X_n,Y_n,n\geq1$, sodass $0\leq X_n\uparrow X$ und $0\leq Y_n\uparrow Y$. Dann gilt auch $0\leq X_nY_n\uparrow XY$. F\"ur $X_n,Y_n,n\geq1$ wie \"ublich ist $X_n$ eine messbare Funktion von $X$ und $Y_n$ eine messbare Funktion von $Y$. Da $XY\geq0$, gilt $XY\in\mathcal{L}(\Pp)$ und $\E[XY]$ ist wohldefiniert in $\overline\R$. Es gilt
    $$\E [XY]=\lim_{n\to\infty}\E[X_nY_n]=\lim_{n\to\infty}(\E X_n)\cdot(\E Y_n)=(\E X)\cdot(\E Y)$$
    und damit $XY\in\mathcal{L}^1(\Pp)$. 
    \item $X,Y$ messbar\newline
    Schreibe $X=X^+-X^-$ und $Y=Y^+-Y^-$. Mit Proposition sind $X^-,Y^-$ u.a., $X^+,Y^+$ u.a., $X^-,Y^+$ u.a. und $X^+,Y^-$ u.a. Ebenfalls gilt
    $$|XY|\leq|X^+Y^+-X^-Y^++X^-Y^--X^+Y^-|\leq X^+Y^++X^-Y^++X^-Y^-+X^+Y^-$$
    wobei alle Summanden auf der rechten Seite integrierbar sind. Damit gilt $XY\in\mathcal{L}^1(\Pp)$.
    \begin{align*}
        \E[XY]&=\E[X^+Y^+]+\E[X^-Y^-]-\E[X^-Y^+]-\E[X^+Y^-]\\
        &=(\E X^+)\cdot(\E Y^+)+(\E X^-)\cdot(\E Y^-)-(\E X^-)\cdot(\E Y^+)-(\E X^+)\cdot(\E Y^-)\\
        &=(\E X)\cdot(\E Y)
    \end{align*}
    \qed
\end{enumerate}

\section*{Borel\textendash Cantelli Lemmata}
\addcontentsline{toc}{section}{Borel\textendash Cantelli Lemmata}

\paragraph{7.7. Definition:}Seien $A_n\subseteq\Omega, n\geq1$. Definiere
$$\limsup_{n\to\infty}A_n:=\bigcap_{n\geq1}\bigcup_{m\geq n}A_m\text{ und }\liminf_{n\to\infty}A_n:=\bigcup_{n\geq1}\bigcap_{m\geq n}A_m$$
Intuitiv macht diese Definition Sinn, da f\"ur $\displaystyle\limsup_{n\to\infty}\ind{A_n}=\ind{M}$ gilt, dass $M=\displaystyle\limsup_{n\to\infty}A_n$ und \"ahnliches f\"ur den $\liminf$. Kurz $\displaystyle\limsup_{n\to\infty}A_n=\{\omega\in\Omega:\omega\in A_n\text{ unendlich oft}\}$ (englisch: $A_n$ \textit{infinitely often}) und $\displaystyle\liminf_{n\to\infty}A_n=\{\omega\in\Omega:\omega\in A_n \text{ letzendlich}\}$ (englisch: $A_n$ \textit{eventually}). 

\paragraph{Bemerkung:}Es gilt
\begin{itemize}
    \item $\displaystyle\left(\limsup_{n\to\infty}A_n\right)^c=\liminf_{n\to\infty}A_n^c$ (De Morgan)
    \item $\displaystyle\liminf_{n\to\infty}A_n\subseteq\limsup_{n\to\infty}A_n$
    \item $A_n$ messbar f\"ur $n\geq1\implies\displaystyle\limsup_{n\to\infty}A_n,\liminf_{n\to\infty}A_n$ messbar
\end{itemize}

\paragraph{7.8. Lemma (I. Borel\textendash Cantelli Lemma)}Betrachte einen allgeminen Ma\ss{}raum $(\Omega,\A,\mu)$. Seien $A_n\in\A,n\geq1$, sodass $\displaystyle\sum_{n\geq1}\mu(A_n)<\infty$. Dann folgt $\mu\left(\displaystyle\limsup_{n\to\infty}A_n\right)=0$.

\paragraph{Beweis:}Da $\mu(A_n)\in[0,\infty)$ f\"ur alle $n\geq1$, ist die Reihe $\displaystyle\sum_{n\geq1}\mu(A_n)$ absolut konvergent. Ebenfalls gilt 
$$\bigcup_{m\geq1}A_m\supseteq\bigcup_{m\geq2}A_m\supseteq\hdots\supseteq\bigcap_{n\geq1}\bigcup_{m\geq n}A_m=\limsup_{n\to\infty}A_n$$
und $\displaystyle\mu\left(\bigcup_{m\geq1}A_m\right)\leq\sum_{m\geq1}\mu(A_m)<\infty$. Mit der Stetigkeit von oben folgt
\begin{align*}
    \mu\left(\limsup_{n\to\infty}A_n\right)&=\mu\left(\bigcap_{n\geq1}\bigcup_{m\geq n}A_m\right)\\
    &=\lim_{n\to\infty}\mu\left(\bigcup_{m\geq n}A_n\right)\\
    &\leq\lim_{n\to\infty}\sum_{m\geq n}\mu(A_n)=0
\end{align*}
wobei die letze Gleichung aus der (bedingten) Konvergenz folgt (Cauchy-Folge). \qed

\paragraph{7.9. Lemma (II. Borel\textendash Cantelli Lemma)}Betrachte einen Wahrscheinlichkeitsraum $\pspace$ und $A_n\in\A,n\geq1$ unabh\"angig, sodass $\displaystyle\sum_{n\geq1}\Pp(A_n)=\infty$. Dann ist $\displaystyle\Pp\left(\limsup_{n\to\infty}A_n\right)=1$. 

\paragraph{Beweis:}Betrachte die Ungleichung (Beweis z.B. mit Mittelwertsatz oder 1. und 2. Ableitung)
$$1-x\leq e^{-x}\text{ f\"ur alle }x\geq0$$
Laut Annahme gilt f\"ur alle $N\geq1:\displaystyle\sum_{n\geq N}\Pp(A_n)=\infty$ (einfache \"Uberlegung). Zeige nun $\displaystyle\Pp\left(\liminf_{n\to\infty}A_n^c\right)=\nobreak0$.
\begin{align*}
    \Pp\left(\liminf_{n\to\infty}A_n^c\right)&=\lim_{n\to\infty}\Pp\left(\bigcap_{m\geq n}A_m^c\right)=\lim_{n\to\infty}\prod_{m\geq n}\Pp(A_m^c)\\
    &=\lim_{n\to\infty}\prod_{m\geq n}(1-\Pp(A_m))\leq\lim_{n\to\infty}\prod_{m\geq n}e^{-\Pp(A_m)}\\
    &=\lim_{n\to\infty}\exp\left(-\sum_{m\geq n}\Pp(A_m)\right)=\exp\left(-\lim_{n\to\infty}\sum_{m\geq n}\Pp(A_m)\right)=0
\end{align*}
\qed

\paragraph{7.10. Korollar:}Seien $X_n:(\Omega,\A)\to(\overline\R,\cB(\overline\R)),n\geq1$ nicht-negative Zufallsvariablen und sei 
$$\forall\eps>0:\sum_{n\geq1}\Pp(X_n>\eps)<\infty$$
Dann gilt $\displaystyle\Pp\left(\lim_{n\to\infty}X_n=0\right)$.

\paragraph{Beweis:}Da $X_n\geq0$ f\"ur $n\geq1$ gilt
$$\left\{\lim_{n\to\infty}X_n=0\right\}=\left\{\limsup_{n\to\infty}X_n=0\right\}=\bigcap_{k\geq1}\left\{\limsup_{n\to\infty}X_n\leq\frac{1}{k}\right\}$$
Aus der Voraussetzung folgt nun 
$$\forall k\geq1:\sum_{n\geq1}\Pp\left(X_n> k^{-1}\right)<\infty$$
und mit Lemma 7.8 gilt $\displaystyle\Pp\left(\limsup_{n\to\infty}\{X_n>k^{-1}\}\right)=0$. Es gilt
\begin{align*}
    1=\Pp\left(\liminf_{n\to\infty}\{X_n\leq k^{-1}\}\right)\leq\Pp\left(\limsup_{n\to\infty}X_n\leq k^{-1}\right)\leq1
\end{align*}
und damit $\displaystyle\Pp\left(\limsup_{n\to\infty}X_n>k^{-1}\right)=0$ f\"ur alle $k\geq1$. Schlie\ss{}lich folgt
$$\Pp\left(\bigcap_{k\geq1}\left\{\limsup_{n\to\infty}X_n\leq k^{-1}\right\}\right)=1-\Pp\left(\bigcup_{k\geq1}\left\{\limsup_{n\to\infty}X_n> k^{-1}\right\}\right)\geq 1-\sum_{k\geq1}\Pp\left(\limsup_{n\to\infty}X_n> k^{-1}\right)=1$$
\qed

\paragraph{7.11. Lemma:}Sei $\M\subseteq\A$ ein $\pi$-System und $A\in\A$. Sind $\M$ und $\{A\}$ unabh\"angig, dann sind auch $\sigma(\M)$ und $\{A\}$ unabh\"angig. 

\paragraph{Beweis:}Trivial, falls $\Pp(A)\in\{0,1\}$. Sei also $\Pp(A)\in(0,1)$ und zeige $\forall M\in\sigma(\M):\Pp(A\cap M)=\Pp(A)\cdot\Pp(M)$. Setze daf\"ur 
$$\Pp(\ \cdot\ | A):=\dfrac{\Pp(\ \cdot \ \cap A)}{\Pp(A)}$$
f\"ur $\Pp(A)\neq0$ (laut Annahme erf\"ullt). Zeige also $\Pp(M|A)=\Pp(M)$ f\"ur alle $M\in\sigma(\M)$. $\Pp(\ \cdot \ |A)$ ist ein Wahrscheinlichkeitsma\ss{} auf $(\Omega,\A)$ und es gilt laut Annahme
$$\forall M\in\M:\Pp(M|A)=\Pp(M)$$
Mit dem $\lambda\textendash\pi$-Theorem folgt, dass $\Pp(\ \cdot\ |A)$ und $\Pp(\cdot)$ auch auf $\sigma(\M)$ \"ubereinstimmen m\"ussen. \qed

\paragraph{7.12. Satz:}Sei f\"ur $i\in I$, $\M_i\subseteq\A$ ein $\pi$-System. Dann gilt
$$\M_i,i\in I\text{ u.a.}\iff\sigma(\M_i),i\in I\text{ u.a.}$$

\paragraph{Beweis:}Die Richtung $\impliedby$ ist trivial. Sei also $J\subseteq I,J=\{1,\hdots,k\}$ endlich und zeige, dass $\sigma(\M_j),j\in J$ unabh\"angig sind. 
\begin{enumerate}[label=\arabic*.]
    \item Schritt: W\"ahle $M_j\in\M_j$ beliebig f\"ur $j=2, \hdots,k$. Laut Annahme sind $\M_1$ und $\{M_2\cap\hdots\cap M_k\}$ unabh\"angig, sodass mit Lemma 7.11. auch $\sigma(\M_1)$ und $\{M_2\cap\hdots\cap M_k\}$ unabh\"angig sind. Es folgt, dass $\sigma(\M_1),\M_2,\hdots,\M_k$ unabh\"angig sind.
    \item Schritt: W\"ahle $M_1\in\sigma(\M_1)$ und $M_j\in\M_j,j=3,\hdots,k$. Wegen dem 1. Schritt sind $\M_2$ und $\{M_1\cap M_3\cap\hdots\cap M_k\}$ unabh\"angig. Mit Lemma 7.11 folgt, dass $\sigma(\M_2)$ und $\{M_1\cap M_3\cap\hdots\cap M_k\}$ unabh\"angig sind und damit $\sigma(\M_2),\sigma(\M_1),\M_3,\hdots,\M_k$ unabh\"angig sind. 
\end{enumerate}
Nach $k-2$ weiteren Schritten ist die Unabh\"angigkeit von $\sigma(\M_1),\hdots,\sigma(\M_k)$ bewiesen. \qed

\paragraph{7.13. Korollar:}Seien $X_i:(\Omega,\A)\to(\overline\R,\cB(\overline\R)),i\in I$ Zufallsvariablen. Dann sind die $X_i,i\in I$ genau dann unabh\"angig, wenn gilt
$$\forall J\subseteq I,|J|<\infty:\forall t_j\in\overline\R,j\in J:\Pp(X_j\leq t_j,j\in J)=\prod_{j\in J}\Pp(X_j\leq t_j)$$
Falls $X_i,i\in I$  reelwertig sind, k\"onnen die $t_j\in\R$ gew\"ahlt werden. 

\paragraph{Beweis:}Die Richtung $\implies$ ist trivial. Es gilt (cf. Kapitel 3) $\cB(\overline\R)=\sigma(\mathcal{K})=\sigma\left(\{[-\infty,t]:t\in\overline\R\}\right)$. F\"ur $i\in I$ setze $\M_i:=\left\{X_i^{-1}\left([-\infty,t]\right):t\in\overline\R\right\}$. Mit Proposition 3.4 gilt $\sigma(\M_i)=\sigma(X_i)$ f\"ur alle $i\in I$. Da der Durchschnitt zweier abgeschlossener Intervalle wieder ein abgeschlossenes Intervall ist, ist $\M_i$ f\"ur $i\in I$ ein $\pi$-System. Laut Annahme sind $\M_i,i\in I$ unabh\"angig. Mit Satz 7.12 folgt die Unabh\"angigkeit der $\sigma(X_i),i\in I$ und damit per Definition die Unabh\"angigkeit der $X_i,i\in I$. \qed

\paragraph{7.14. Proposition:}Seien $X_i:(\Omega,\A)\to(\Omega',\A'),i\in I$ unabh\"angige Zufallsvariablen. F\"ur eine Partition $I=K\cup L$ mit $K,L\neq\emptyset$ sind auch $\sigma(X_k,k\in K)$ und $\sigma(X_\ell,\ell\in L)$ unabh\"angig. 

\paragraph{Beweis:}Mit Lemma 3.5 gilt
$$\sigma(X_k,k\in K)=\sigma\left(\bigcap_{j\in J}\{X_j\in A_j'\}:J\subseteq K,|J|<\infty,A_j'\in\A' \text{ f\"ur }j\in J\right)=\sigma(\mathcal{E}_1)$$
und $\sigma(X_\ell,\ell\in L)=\sigma(\mathcal{E}_2)$, wobei $\mathcal{E}_1,\mathcal{E}_2$ $\pi$-Systeme sind (einfach nachzupr\"ufen). Au\ss{}erdem sind $\mathcal{E}_1,\mathcal{E}_2$ unabh\"angig, denn f\"ur $\displaystyle A=\bigcap_{j\in J}\{X_j\in A_j'\}\in \mathcal{E}_1$ f\"ur $J\subseteq K$ endlich und $\displaystyle B=\bigcap_{m\in M}\{X_m\in A_m'\}$ f\"ur $M\subseteq L$ endlich gilt
\begin{align*}
    \Pp(A\cap B)&=\Pp\left(\bigcap_{j\in J\cup M}\{X_j\in A_j'\}\right)\overset{J\cup M\subseteq I}{=}\prod_{j\in J\cup M}\Pp(X_j\in A_j')\\
    &=\Pp\left(\bigcap_{j\in J}\{X_j\in A_j'\}\right)\cdot\Pp\left(\bigcap_{m\in M}\{X_m\in A_m'\}\right)=\Pp(A)\cdot\Pp(B)
\end{align*}
da $K,L$ disjunkt sind und damit auch alle endlichen Teilmengen $J,M$ disjunkt sind. Mit Satz 7.12 folgt nun, dass $\sigma(\mathcal{E}_1)$ und $\sigma(\mathcal{E}_2)$ unabh\"angig sind und damit die Aussage. \qed

\section*{Asymptotische $\sigma$-Algebra}
\addcontentsline{toc}{section}{Asymptotische $\sigma$-Algebra}

\paragraph{7.15. Definition:}Seien $X_n,n\geq1$ Zufallsvariablen. Setze 
$$\cB_n:=\sigma(X_1,\hdots,X_n)\text{ und }\mathcal{T}_n:=\sigma(X_{n+1},X_{n+2},\hdots)$$
Dann gilt $\cB_n\subseteq\cB_{n+1}$ und $\mathcal{T}_n\supseteq\mathcal{T}_{n+1}$ f\"ur alle $n\geq1$. Definiere weiters
$$\cB:=\bigcup_{n\geq1}\cB_n\text{ und }\mathcal{T}_\infty:=\bigcap_{n\geq1}\mathcal{T}_n$$
Dann sind $\sigma(\cB)$ und $\mathcal{T}_\infty$ $\sigma$-Algebren und mann nennt $\mathcal{T}_\infty$ die asymptotische $\sigma$-Algebra (englisch \textit{tail $\sigma$-algebra}) der $X_n,n\geq1$. 

\paragraph{7.16. Satz (0\textendash 1-Gesetz von Kolmogorov):}Betrachte unabh\"angige Zufallsvariablen $X_n,n\geq1$ sowie deren asymptotische $\sigma$-Algebra $\mathcal{T}_\infty$. Dann gilt $\forall A\in\mathcal{T}_\infty:\Pp(A)\in\{0,1\}$.

\paragraph{Beweis:}Mit Lemma 7.14 sind $\cB_n$ und $\mathcal{T}_n$ unbah\"angig f\"ur alle $n\geq1$. Da $\mathcal{T}_\infty\subseteq\mathcal{T}_n$ sind $\cB_n$ und $\mathcal{T}_\infty$ f\"ur alle $n\geq1$ unabh\"angig. Damit sind auch $\cB$ und $\mathcal{T}_\infty$ unabh\"angig. Da $\cB$ ein $\pi$-System ist (leicht nachzupr\"ufen), folgt mit Satz 7.12, dass $\sigma(\cB)$ und $\mathcal{T}_\infty$ unabh\"angig sind. Nun ist
$$\mathcal{T}_\infty\subseteq\mathcal{T}_n=\sigma(X_{n+1},X_{n+2},\hdots)\subseteq\sigma(X_1,X_2,\hdots)=\sigma(\cB)$$
und damit sind $\mathcal{T}_\infty$ und $\mathcal{T}_\infty$ unabh\"angig und 
$$\forall A\in\mathcal{T}_\infty:\Pp(A)=\Pp(A\cap A)=(\Pp(A))^2$$
und damit $\Pp(A)\in\{0,1\}$. \qed

\paragraph{7.17. Beispiel (Ereignisse aus $\mathcal{T}$):}Betrachte unabh\"angige, rellwertige Zufallsvariablen $X_n,n\geq\nobreak1$.
\begin{enumerate}[label=(\roman*)]
    \item $\left\{\displaystyle\limsup_{n\to\infty}X_n\geq c\right\},\left\{\displaystyle\liminf_{n\to\infty}X_n\geq c\right\}\in\mathcal{T}_\infty$ f\"ur $c\in\R$.\newline
    Es gilt
    $$\left\{\displaystyle\limsup_{n\to\infty}X_n\geq c\right\}=\left\{\limsup_{\substack{n\to\infty\\n\geq N}}X_n\geq c\right\}\in\mathcal{T}_N$$
    f\"ur alle $N\geq1$ und damit $\left\{\displaystyle\limsup_{n\to\infty}X_n\geq c\right\}\in\displaystyle\bigcap_{N\geq1}\mathcal{T}_N=\mathcal{T}_\infty$. \"Ahnliches gilt f\"ur den $\liminf$. 
    \item $\displaystyle\left\{\lim_{n\to\infty}X_n\in\R\right\}\in\mathcal{T}_\infty$ folgt aus Proposition 3.18.
    \item $\left\{\displaystyle\limsup_{n\to\infty}\frac{1}{n}\sum_{i=1}^nX_n > c\right\},\left\{\displaystyle\liminf_{n\to\infty}\frac{1}{n}\sum_{i=1}^nX_n > c\right\}\in\mathcal{T}_\infty$\newline
    Sei $\omega\in\left\{\displaystyle\limsup_{n\to\infty}\frac{1}{n}\sum_{i=1}^nX_n > c\right\}$. Dann gibt es eine Teilfolge $n'$, sodass 
    $$\lim_{n'\to\infty}\frac{1}{n'}\sum_{i=1}^{n'}X_i(\omega)>c$$
    Aber $\forall N\geq1$ gilt
    $$\frac{1}{n'}\sum_{i=1}^{n'}X_i(\omega)=\frac{1}{n'}\sum_{\substack{i=1\\i\leq N}}^{n'}X_i(\omega)+\frac{1}{n'}\sum_{\substack{i=1\\i> N}}^{n'}X_i(\omega)$$
    wobei der erste Summand f\"ur $n'\to\infty$ gegen 0 konvergiert (einfache \"Uberlegung). Damit gilt
    $$\lim_{n'\to\infty}\frac{1}{n'}\sum_{\substack{i=1\\i> N}}^{n'}X_i(\omega)=c'>c$$
    f\"ur alle $N\geq1$ und es folgt
    $$\omega\in \left\{\displaystyle\limsup_{n\to\infty}\frac{1}{n}\sum_{\substack{i=1\\i>N}}^nX_n > c\right\}\in\mathcal{T}_N$$
    f\"ur alle $N\geq1$. \"Ahnliches gilt f\"ur den $\liminf$. 
    \item $\left\{\displaystyle\frac{1}{n}\sum_{i=1}^nX_n \text{ konvergiert in }\overline\R\right\}\in\mathcal{T}_\infty$ folgt ebenfalls aus Proposition 3.18. 
\end{enumerate}


\end{document}
